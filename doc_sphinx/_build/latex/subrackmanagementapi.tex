%% Generated by Sphinx.
\def\sphinxdocclass{report}
\documentclass[letterpaper,10pt,english]{sphinxmanual}
\ifdefined\pdfpxdimen
   \let\sphinxpxdimen\pdfpxdimen\else\newdimen\sphinxpxdimen
\fi \sphinxpxdimen=.75bp\relax
\ifdefined\pdfimageresolution
    \pdfimageresolution= \numexpr \dimexpr1in\relax/\sphinxpxdimen\relax
\fi
%% let collapsible pdf bookmarks panel have high depth per default
\PassOptionsToPackage{bookmarksdepth=5}{hyperref}

\PassOptionsToPackage{booktabs}{sphinx}
\PassOptionsToPackage{colorrows}{sphinx}

\PassOptionsToPackage{warn}{textcomp}
\usepackage[utf8]{inputenc}
\ifdefined\DeclareUnicodeCharacter
% support both utf8 and utf8x syntaxes
  \ifdefined\DeclareUnicodeCharacterAsOptional
    \def\sphinxDUC#1{\DeclareUnicodeCharacter{"#1}}
  \else
    \let\sphinxDUC\DeclareUnicodeCharacter
  \fi
  \sphinxDUC{00A0}{\nobreakspace}
  \sphinxDUC{2500}{\sphinxunichar{2500}}
  \sphinxDUC{2502}{\sphinxunichar{2502}}
  \sphinxDUC{2514}{\sphinxunichar{2514}}
  \sphinxDUC{251C}{\sphinxunichar{251C}}
  \sphinxDUC{2572}{\textbackslash}
\fi
\usepackage{cmap}
\usepackage[T1]{fontenc}
\usepackage{amsmath,amssymb,amstext}
\usepackage{babel}



\usepackage{tgtermes}
\usepackage{tgheros}
\renewcommand{\ttdefault}{txtt}



\usepackage[Bjarne]{fncychap}
\usepackage{sphinx}

\fvset{fontsize=auto}
\usepackage{geometry}


% Include hyperref last.
\usepackage{hyperref}
% Fix anchor placement for figures with captions.
\usepackage{hypcap}% it must be loaded after hyperref.
% Set up styles of URL: it should be placed after hyperref.
\urlstyle{same}

\addto\captionsenglish{\renewcommand{\contentsname}{Contents:}}

\usepackage{sphinxmessages}
\setcounter{tocdepth}{1}



\title{Subrack Management API}
\date{Jan 08, 2024}
\release{1.0}
\author{Simone Chiarucci (INAF\sphinxhyphen{}OAA)}
\newcommand{\sphinxlogo}{\vbox{}}
\renewcommand{\releasename}{Release}
\makeindex
\begin{document}

\ifdefined\shorthandoff
  \ifnum\catcode`\=\string=\active\shorthandoff{=}\fi
  \ifnum\catcode`\"=\active\shorthandoff{"}\fi
\fi

\pagestyle{empty}
\sphinxmaketitle
\pagestyle{plain}
\sphinxtableofcontents
\pagestyle{normal}
\phantomsection\label{\detokenize{index::doc}}


\sphinxstepscope


\chapter{Subrack management API Documentation}
\label{\detokenize{apidocs:module-backplane}}\label{\detokenize{apidocs:subrack-management-api-documentation}}\label{\detokenize{apidocs::doc}}\index{module@\spxentry{module}!backplane@\spxentry{backplane}}\index{backplane@\spxentry{backplane}!module@\spxentry{module}}\index{BackplaneInvalidParameter@\spxentry{BackplaneInvalidParameter}}

\begin{fulllineitems}
\phantomsection\label{\detokenize{apidocs:backplane.BackplaneInvalidParameter}}
\pysigstartsignatures
\pysigline{\sphinxbfcode{\sphinxupquote{exception\DUrole{w}{ }}}\sphinxcode{\sphinxupquote{backplane.}}\sphinxbfcode{\sphinxupquote{BackplaneInvalidParameter}}}
\pysigstopsignatures
\sphinxAtStartPar
Exception class for invalid parameters provided to a function or class method.

\end{fulllineitems}

\index{twos\_comp() (in module backplane)@\spxentry{twos\_comp()}\spxextra{in module backplane}}

\begin{fulllineitems}
\phantomsection\label{\detokenize{apidocs:backplane.twos_comp}}
\pysigstartsignatures
\pysiglinewithargsret{\sphinxcode{\sphinxupquote{backplane.}}\sphinxbfcode{\sphinxupquote{twos\_comp}}}{\sphinxparam{\DUrole{n}{val}}\sphinxparamcomma \sphinxparam{\DUrole{n}{bits}}}{}
\pysigstopsignatures
\sphinxAtStartPar
Compute the two’s complement of an integer value.
\begin{quote}\begin{description}
\sphinxlineitem{Parameters}\begin{itemize}
\item {} 
\sphinxAtStartPar
\sphinxstyleliteralstrong{\sphinxupquote{val}} (\sphinxstyleliteralemphasis{\sphinxupquote{int}}) \textendash{} The integer value.

\item {} 
\sphinxAtStartPar
\sphinxstyleliteralstrong{\sphinxupquote{bits}} (\sphinxstyleliteralemphasis{\sphinxupquote{int}}) \textendash{} The number of bits representing the integer.

\end{itemize}

\sphinxlineitem{Returns}
\sphinxAtStartPar
The two’s complement of the input value.

\sphinxlineitem{Return type}
\sphinxAtStartPar
int

\end{description}\end{quote}

\end{fulllineitems}

\index{module@\spxentry{module}!management@\spxentry{management}}\index{management@\spxentry{management}!module@\spxentry{module}}\index{MANAGEMENT (class in management)@\spxentry{MANAGEMENT}\spxextra{class in management}}\phantomsection\label{\detokenize{apidocs:module-management}}

\begin{fulllineitems}
\phantomsection\label{\detokenize{apidocs:management.MANAGEMENT}}
\pysigstartsignatures
\pysiglinewithargsret{\sphinxbfcode{\sphinxupquote{class\DUrole{w}{ }}}\sphinxcode{\sphinxupquote{management.}}\sphinxbfcode{\sphinxupquote{MANAGEMENT}}}{\sphinxparam{\DUrole{o}{**}\DUrole{n}{kwargs}}}{}
\pysigstopsignatures
\sphinxAtStartPar
Class representing a MANAGEMENT instance.
\index{checkLoad() (management.MANAGEMENT method)@\spxentry{checkLoad()}\spxextra{management.MANAGEMENT method}}

\begin{fulllineitems}
\phantomsection\label{\detokenize{apidocs:management.MANAGEMENT.checkLoad}}
\pysigstartsignatures
\pysiglinewithargsret{\sphinxbfcode{\sphinxupquote{checkLoad}}}{}{}
\pysigstopsignatures
\sphinxAtStartPar
Check if the registers are loaded.
\begin{quote}\begin{description}
\sphinxlineitem{Raises}
\sphinxAtStartPar
\sphinxstyleliteralstrong{\sphinxupquote{NameError}} \textendash{} If registers are not loaded or the board is not connected.

\end{description}\end{quote}

\end{fulllineitems}

\index{disconnect() (management.MANAGEMENT method)@\spxentry{disconnect()}\spxextra{management.MANAGEMENT method}}

\begin{fulllineitems}
\phantomsection\label{\detokenize{apidocs:management.MANAGEMENT.disconnect}}
\pysigstartsignatures
\pysiglinewithargsret{\sphinxbfcode{\sphinxupquote{disconnect}}}{}{}
\pysigstopsignatures
\sphinxAtStartPar
Disconnect from the network.

\sphinxAtStartPar
This function closes the network connection and sets the state to “Unconnected”.
\begin{quote}\begin{description}
\sphinxlineitem{Raises}
\sphinxAtStartPar
\sphinxstyleliteralstrong{\sphinxupquote{None}} \textendash{} 

\end{description}\end{quote}

\end{fulllineitems}

\index{find\_register() (management.MANAGEMENT method)@\spxentry{find\_register()}\spxextra{management.MANAGEMENT method}}

\begin{fulllineitems}
\phantomsection\label{\detokenize{apidocs:management.MANAGEMENT.find_register}}
\pysigstartsignatures
\pysiglinewithargsret{\sphinxbfcode{\sphinxupquote{find\_register}}}{\sphinxparam{\DUrole{n}{register\_name}}\sphinxparamcomma \sphinxparam{\DUrole{n}{display}\DUrole{o}{=}\DUrole{default_value}{False}}}{}
\pysigstopsignatures
\sphinxAtStartPar
Find register information for provided search string.
:param register\_name: Regular expression to search against
:param display: True to output result to console
:return: List of found registers

\end{fulllineitems}

\index{find\_register\_names() (management.MANAGEMENT method)@\spxentry{find\_register\_names()}\spxextra{management.MANAGEMENT method}}

\begin{fulllineitems}
\phantomsection\label{\detokenize{apidocs:management.MANAGEMENT.find_register_names}}
\pysigstartsignatures
\pysiglinewithargsret{\sphinxbfcode{\sphinxupquote{find\_register\_names}}}{\sphinxparam{\DUrole{n}{register\_name}}\sphinxparamcomma \sphinxparam{\DUrole{n}{display}\DUrole{o}{=}\DUrole{default_value}{False}}}{}
\pysigstopsignatures
\sphinxAtStartPar
Find register names for the provided search string.
\begin{quote}\begin{description}
\sphinxlineitem{Parameters}\begin{itemize}
\item {} 
\sphinxAtStartPar
\sphinxstyleliteralstrong{\sphinxupquote{register\_name}} (\sphinxstyleliteralemphasis{\sphinxupquote{str}}) \textendash{} Regular expression to search against.

\item {} 
\sphinxAtStartPar
\sphinxstyleliteralstrong{\sphinxupquote{display}} (\sphinxstyleliteralemphasis{\sphinxupquote{bool}}\sphinxstyleliteralemphasis{\sphinxupquote{, }}\sphinxstyleliteralemphasis{\sphinxupquote{optional}}) \textendash{} True to output result to console. Defaults to False.

\end{itemize}

\sphinxlineitem{Returns}
\sphinxAtStartPar
List of found register names.

\sphinxlineitem{Return type}
\sphinxAtStartPar
list

\end{description}\end{quote}

\end{fulllineitems}

\index{get\_bios() (management.MANAGEMENT method)@\spxentry{get\_bios()}\spxextra{management.MANAGEMENT method}}

\begin{fulllineitems}
\phantomsection\label{\detokenize{apidocs:management.MANAGEMENT.get_bios}}
\pysigstartsignatures
\pysiglinewithargsret{\sphinxbfcode{\sphinxupquote{get\_bios}}}{}{}
\pysigstopsignatures
\sphinxAtStartPar
Generate a BIOS string based on specific register values.
\begin{quote}\begin{description}
\sphinxlineitem{Returns}
\sphinxAtStartPar
BIOS information string.

\sphinxlineitem{Return type}
\sphinxAtStartPar
str

\end{description}\end{quote}

\end{fulllineitems}

\index{get\_board() (management.MANAGEMENT method)@\spxentry{get\_board()}\spxextra{management.MANAGEMENT method}}

\begin{fulllineitems}
\phantomsection\label{\detokenize{apidocs:management.MANAGEMENT.get_board}}
\pysigstartsignatures
\pysiglinewithargsret{\sphinxbfcode{\sphinxupquote{get\_board}}}{\sphinxparam{\DUrole{n}{ext\_info\_offset}\DUrole{o}{=}\DUrole{default_value}{16}}}{}
\pysigstopsignatures
\sphinxAtStartPar
Get the board name from the extended info string.
\begin{quote}\begin{description}
\sphinxlineitem{Parameters}
\sphinxAtStartPar
\sphinxstyleliteralstrong{\sphinxupquote{ext\_info\_offset}} (\sphinxstyleliteralemphasis{\sphinxupquote{int}}\sphinxstyleliteralemphasis{\sphinxupquote{, }}\sphinxstyleliteralemphasis{\sphinxupquote{optional}}) \textendash{} Offset for extended info. Defaults to 0x10.

\sphinxlineitem{Returns}
\sphinxAtStartPar
The extracted board name.

\sphinxlineitem{Return type}
\sphinxAtStartPar
str

\sphinxlineitem{Raises}
\sphinxAtStartPar
\sphinxstyleliteralstrong{\sphinxupquote{NameError}} \textendash{} If the field BOARD doesn’t exist in the extended info.

\end{description}\end{quote}

\end{fulllineitems}

\index{get\_board\_info() (management.MANAGEMENT method)@\spxentry{get\_board\_info()}\spxextra{management.MANAGEMENT method}}

\begin{fulllineitems}
\phantomsection\label{\detokenize{apidocs:management.MANAGEMENT.get_board_info}}
\pysigstartsignatures
\pysiglinewithargsret{\sphinxbfcode{\sphinxupquote{get\_board\_info}}}{}{}
\pysigstopsignatures
\sphinxAtStartPar
Retrieve information about the board.
\begin{quote}\begin{description}
\sphinxlineitem{Returns}
\sphinxAtStartPar
Board information as a dictionary.

\sphinxlineitem{Return type}
\sphinxAtStartPar
dict

\end{description}\end{quote}

\end{fulllineitems}

\index{get\_extended\_info() (management.MANAGEMENT method)@\spxentry{get\_extended\_info()}\spxextra{management.MANAGEMENT method}}

\begin{fulllineitems}
\phantomsection\label{\detokenize{apidocs:management.MANAGEMENT.get_extended_info}}
\pysigstartsignatures
\pysiglinewithargsret{\sphinxbfcode{\sphinxupquote{get\_extended\_info}}}{\sphinxparam{\DUrole{n}{ext\_info\_offset}\DUrole{o}{=}\DUrole{default_value}{16}}}{}
\pysigstopsignatures
\sphinxAtStartPar
Get the extended info string from the board.

\end{fulllineitems}

\index{get\_mac() (management.MANAGEMENT method)@\spxentry{get\_mac()}\spxextra{management.MANAGEMENT method}}

\begin{fulllineitems}
\phantomsection\label{\detokenize{apidocs:management.MANAGEMENT.get_mac}}
\pysigstartsignatures
\pysiglinewithargsret{\sphinxbfcode{\sphinxupquote{get\_mac}}}{}{}
\pysigstopsignatures
\sphinxAtStartPar
Retrieve the MAC address and format it as a string.
\begin{quote}\begin{description}
\sphinxlineitem{Returns}
\sphinxAtStartPar
MAC address string.

\sphinxlineitem{Return type}
\sphinxAtStartPar
str

\end{description}\end{quote}

\end{fulllineitems}

\index{get\_register\_name\_by\_address() (management.MANAGEMENT method)@\spxentry{get\_register\_name\_by\_address()}\spxextra{management.MANAGEMENT method}}

\begin{fulllineitems}
\phantomsection\label{\detokenize{apidocs:management.MANAGEMENT.get_register_name_by_address}}
\pysigstartsignatures
\pysiglinewithargsret{\sphinxbfcode{\sphinxupquote{get\_register\_name\_by\_address}}}{\sphinxparam{\DUrole{n}{add}}}{}
\pysigstopsignatures
\sphinxAtStartPar
Get register name by address.
\begin{quote}\begin{description}
\sphinxlineitem{Parameters}
\sphinxAtStartPar
\sphinxstyleliteralstrong{\sphinxupquote{add}} (\sphinxstyleliteralemphasis{\sphinxupquote{int}}) \textendash{} Register address.

\sphinxlineitem{Returns}
\sphinxAtStartPar
Register name or “Unknown register address” if not found.

\sphinxlineitem{Return type}
\sphinxAtStartPar
str

\end{description}\end{quote}

\end{fulllineitems}

\index{get\_xml\_from\_board() (management.MANAGEMENT method)@\spxentry{get\_xml\_from\_board()}\spxextra{management.MANAGEMENT method}}

\begin{fulllineitems}
\phantomsection\label{\detokenize{apidocs:management.MANAGEMENT.get_xml_from_board}}
\pysigstartsignatures
\pysiglinewithargsret{\sphinxbfcode{\sphinxupquote{get\_xml\_from\_board}}}{\sphinxparam{\DUrole{n}{xml\_map\_offset}}}{}
\pysigstopsignatures
\sphinxAtStartPar
Get XML data from the board.

\end{fulllineitems}

\index{list\_register\_names() (management.MANAGEMENT method)@\spxentry{list\_register\_names()}\spxextra{management.MANAGEMENT method}}

\begin{fulllineitems}
\phantomsection\label{\detokenize{apidocs:management.MANAGEMENT.list_register_names}}
\pysigstartsignatures
\pysiglinewithargsret{\sphinxbfcode{\sphinxupquote{list\_register\_names}}}{}{}
\pysigstopsignatures
\sphinxAtStartPar
List register names.
\begin{quote}\begin{description}
\sphinxlineitem{Raises}
\sphinxAtStartPar
\sphinxstyleliteralstrong{\sphinxupquote{None}} \textendash{} 

\end{description}\end{quote}

\end{fulllineitems}

\index{load\_firmware\_blocking() (management.MANAGEMENT method)@\spxentry{load\_firmware\_blocking()}\spxextra{management.MANAGEMENT method}}

\begin{fulllineitems}
\phantomsection\label{\detokenize{apidocs:management.MANAGEMENT.load_firmware_blocking}}
\pysigstartsignatures
\pysiglinewithargsret{\sphinxbfcode{\sphinxupquote{load\_firmware\_blocking}}}{\sphinxparam{\DUrole{n}{Device}}\sphinxparamcomma \sphinxparam{\DUrole{n}{path\_to\_xml\_file}\DUrole{o}{=}\DUrole{default_value}{\textquotesingle{}\textquotesingle{}}}\sphinxparamcomma \sphinxparam{\DUrole{n}{xml\_map\_offset}\DUrole{o}{=}\DUrole{default_value}{8}}}{}
\pysigstopsignatures
\sphinxAtStartPar
Load firmware blocking.

\end{fulllineitems}

\index{pll\_calib() (management.MANAGEMENT method)@\spxentry{pll\_calib()}\spxextra{management.MANAGEMENT method}}

\begin{fulllineitems}
\phantomsection\label{\detokenize{apidocs:management.MANAGEMENT.pll_calib}}
\pysigstartsignatures
\pysiglinewithargsret{\sphinxbfcode{\sphinxupquote{pll\_calib}}}{}{}
\pysigstopsignatures
\sphinxAtStartPar
Calibrate PLL.

\sphinxAtStartPar
This function performs calibration by writing specific values to SPI addresses.
\begin{quote}\begin{description}
\sphinxlineitem{Raises}
\sphinxAtStartPar
\sphinxstyleliteralstrong{\sphinxupquote{None}} \textendash{} 

\end{description}\end{quote}

\end{fulllineitems}

\index{pll\_dumpcfg() (management.MANAGEMENT method)@\spxentry{pll\_dumpcfg()}\spxextra{management.MANAGEMENT method}}

\begin{fulllineitems}
\phantomsection\label{\detokenize{apidocs:management.MANAGEMENT.pll_dumpcfg}}
\pysigstartsignatures
\pysiglinewithargsret{\sphinxbfcode{\sphinxupquote{pll\_dumpcfg}}}{\sphinxparam{\DUrole{n}{cfg\_filename}}}{}
\pysigstopsignatures
\sphinxAtStartPar
Dump PLL configuration to a file.
\begin{quote}\begin{description}
\sphinxlineitem{Parameters}
\sphinxAtStartPar
\sphinxstyleliteralstrong{\sphinxupquote{cfg\_filename}} (\sphinxstyleliteralemphasis{\sphinxupquote{str}}) \textendash{} The path to the configuration file.

\end{description}\end{quote}

\end{fulllineitems}

\index{pll\_ioupdate() (management.MANAGEMENT method)@\spxentry{pll\_ioupdate()}\spxextra{management.MANAGEMENT method}}

\begin{fulllineitems}
\phantomsection\label{\detokenize{apidocs:management.MANAGEMENT.pll_ioupdate}}
\pysigstartsignatures
\pysiglinewithargsret{\sphinxbfcode{\sphinxupquote{pll\_ioupdate}}}{}{}
\pysigstopsignatures
\sphinxAtStartPar
Update PLL IO.

\sphinxAtStartPar
This function updates PLL IO by writing a specific value to the SPI address.
\begin{quote}\begin{description}
\sphinxlineitem{Raises}
\sphinxAtStartPar
\sphinxstyleliteralstrong{\sphinxupquote{None}} \textendash{} 

\end{description}\end{quote}

\end{fulllineitems}

\index{pll\_ldcfg() (management.MANAGEMENT method)@\spxentry{pll\_ldcfg()}\spxextra{management.MANAGEMENT method}}

\begin{fulllineitems}
\phantomsection\label{\detokenize{apidocs:management.MANAGEMENT.pll_ldcfg}}
\pysigstartsignatures
\pysiglinewithargsret{\sphinxbfcode{\sphinxupquote{pll\_ldcfg}}}{\sphinxparam{\DUrole{n}{cfg\_filename}}}{}
\pysigstopsignatures
\sphinxAtStartPar
Load PLL configuration from a file.
\begin{quote}\begin{description}
\sphinxlineitem{Parameters}
\sphinxAtStartPar
\sphinxstyleliteralstrong{\sphinxupquote{cfg\_filename}} (\sphinxstyleliteralemphasis{\sphinxupquote{str}}) \textendash{} The path to the configuration file.

\end{description}\end{quote}

\end{fulllineitems}

\index{pll\_read\_with\_update() (management.MANAGEMENT method)@\spxentry{pll\_read\_with\_update()}\spxextra{management.MANAGEMENT method}}

\begin{fulllineitems}
\phantomsection\label{\detokenize{apidocs:management.MANAGEMENT.pll_read_with_update}}
\pysigstartsignatures
\pysiglinewithargsret{\sphinxbfcode{\sphinxupquote{pll\_read\_with\_update}}}{\sphinxparam{\DUrole{n}{address}}}{}
\pysigstopsignatures
\sphinxAtStartPar
Read from SPI with PLL update.
\begin{quote}\begin{description}
\sphinxlineitem{Parameters}
\sphinxAtStartPar
\sphinxstyleliteralstrong{\sphinxupquote{address}} (\sphinxstyleliteralemphasis{\sphinxupquote{int}}) \textendash{} The address to read from.

\sphinxlineitem{Returns}
\sphinxAtStartPar
The read value.

\sphinxlineitem{Return type}
\sphinxAtStartPar
int

\end{description}\end{quote}

\end{fulllineitems}

\index{pll\_write\_with\_update() (management.MANAGEMENT method)@\spxentry{pll\_write\_with\_update()}\spxextra{management.MANAGEMENT method}}

\begin{fulllineitems}
\phantomsection\label{\detokenize{apidocs:management.MANAGEMENT.pll_write_with_update}}
\pysigstartsignatures
\pysiglinewithargsret{\sphinxbfcode{\sphinxupquote{pll\_write\_with\_update}}}{\sphinxparam{\DUrole{n}{address}}\sphinxparamcomma \sphinxparam{\DUrole{n}{value}}}{}
\pysigstopsignatures
\sphinxAtStartPar
Write to SPI with PLL update.
\begin{quote}\begin{description}
\sphinxlineitem{Parameters}\begin{itemize}
\item {} 
\sphinxAtStartPar
\sphinxstyleliteralstrong{\sphinxupquote{address}} (\sphinxstyleliteralemphasis{\sphinxupquote{int}}) \textendash{} The address to write to.

\item {} 
\sphinxAtStartPar
\sphinxstyleliteralstrong{\sphinxupquote{value}} (\sphinxstyleliteralemphasis{\sphinxupquote{int}}) \textendash{} The value to write.

\end{itemize}

\end{description}\end{quote}

\end{fulllineitems}

\index{read\_register() (management.MANAGEMENT method)@\spxentry{read\_register()}\spxextra{management.MANAGEMENT method}}

\begin{fulllineitems}
\phantomsection\label{\detokenize{apidocs:management.MANAGEMENT.read_register}}
\pysigstartsignatures
\pysiglinewithargsret{\sphinxbfcode{\sphinxupquote{read\_register}}}{\sphinxparam{\DUrole{n}{register}}\sphinxparamcomma \sphinxparam{\DUrole{n}{n}\DUrole{o}{=}\DUrole{default_value}{1}}\sphinxparamcomma \sphinxparam{\DUrole{n}{offset}\DUrole{o}{=}\DUrole{default_value}{0}}\sphinxparamcomma \sphinxparam{\DUrole{n}{device}\DUrole{o}{=}\DUrole{default_value}{None}}}{}
\pysigstopsignatures
\sphinxAtStartPar
Get register value
:param register: Register name
:param n: Number of words to read
:param offset: Memory address offset to read from
:param device: Device/node can be explicitly specified
:return: Values

\end{fulllineitems}

\index{read\_spi() (management.MANAGEMENT method)@\spxentry{read\_spi()}\spxextra{management.MANAGEMENT method}}

\begin{fulllineitems}
\phantomsection\label{\detokenize{apidocs:management.MANAGEMENT.read_spi}}
\pysigstartsignatures
\pysiglinewithargsret{\sphinxbfcode{\sphinxupquote{read\_spi}}}{\sphinxparam{\DUrole{n}{address}}}{}
\pysigstopsignatures
\sphinxAtStartPar
Read SPI data from the specified address.
\begin{quote}\begin{description}
\sphinxlineitem{Parameters}
\sphinxAtStartPar
\sphinxstyleliteralstrong{\sphinxupquote{address}} (\sphinxstyleliteralemphasis{\sphinxupquote{int}}) \textendash{} The address to read from.

\sphinxlineitem{Returns}
\sphinxAtStartPar
The read value.

\sphinxlineitem{Return type}
\sphinxAtStartPar
int

\end{description}\end{quote}

\end{fulllineitems}

\index{write\_register() (management.MANAGEMENT method)@\spxentry{write\_register()}\spxextra{management.MANAGEMENT method}}

\begin{fulllineitems}
\phantomsection\label{\detokenize{apidocs:management.MANAGEMENT.write_register}}
\pysigstartsignatures
\pysiglinewithargsret{\sphinxbfcode{\sphinxupquote{write\_register}}}{\sphinxparam{\DUrole{n}{register}}\sphinxparamcomma \sphinxparam{\DUrole{n}{values}}\sphinxparamcomma \sphinxparam{\DUrole{n}{offset}\DUrole{o}{=}\DUrole{default_value}{0}}\sphinxparamcomma \sphinxparam{\DUrole{n}{device}\DUrole{o}{=}\DUrole{default_value}{None}}}{}
\pysigstopsignatures
\sphinxAtStartPar
Set register value
:param register: Register name
:param values: Values to write
:param offset: Memory address offset to write to
:param device: Device/node can be explicitly specified

\end{fulllineitems}

\index{write\_spi() (management.MANAGEMENT method)@\spxentry{write\_spi()}\spxextra{management.MANAGEMENT method}}

\begin{fulllineitems}
\phantomsection\label{\detokenize{apidocs:management.MANAGEMENT.write_spi}}
\pysigstartsignatures
\pysiglinewithargsret{\sphinxbfcode{\sphinxupquote{write\_spi}}}{\sphinxparam{\DUrole{n}{address}}\sphinxparamcomma \sphinxparam{\DUrole{n}{value}}}{}
\pysigstopsignatures
\sphinxAtStartPar
Write SPI data to the specified address.
\begin{quote}\begin{description}
\sphinxlineitem{Parameters}\begin{itemize}
\item {} 
\sphinxAtStartPar
\sphinxstyleliteralstrong{\sphinxupquote{address}} (\sphinxstyleliteralemphasis{\sphinxupquote{int}}) \textendash{} The address to write to.

\item {} 
\sphinxAtStartPar
\sphinxstyleliteralstrong{\sphinxupquote{value}} (\sphinxstyleliteralemphasis{\sphinxupquote{int}}) \textendash{} The value to write.

\end{itemize}

\end{description}\end{quote}

\end{fulllineitems}


\end{fulllineitems}

\index{filter\_list\_by\_level() (in module management)@\spxentry{filter\_list\_by\_level()}\spxextra{in module management}}

\begin{fulllineitems}
\phantomsection\label{\detokenize{apidocs:management.filter_list_by_level}}
\pysigstartsignatures
\pysiglinewithargsret{\sphinxcode{\sphinxupquote{management.}}\sphinxbfcode{\sphinxupquote{filter\_list\_by\_level}}}{\sphinxparam{\DUrole{n}{reg\_name\_list}}\sphinxparamcomma \sphinxparam{\DUrole{n}{reg\_name}}}{}
\pysigstopsignatures
\sphinxAtStartPar
Filter a list of register names by level.

\end{fulllineitems}

\index{format\_num() (in module management)@\spxentry{format\_num()}\spxextra{in module management}}

\begin{fulllineitems}
\phantomsection\label{\detokenize{apidocs:management.format_num}}
\pysigstartsignatures
\pysiglinewithargsret{\sphinxcode{\sphinxupquote{management.}}\sphinxbfcode{\sphinxupquote{format\_num}}}{\sphinxparam{\DUrole{n}{num}}}{}
\pysigstopsignatures
\sphinxAtStartPar
Convert a number to a string.

\end{fulllineitems}

\index{get\_max\_width() (in module management)@\spxentry{get\_max\_width()}\spxextra{in module management}}

\begin{fulllineitems}
\phantomsection\label{\detokenize{apidocs:management.get_max_width}}
\pysigstartsignatures
\pysiglinewithargsret{\sphinxcode{\sphinxupquote{management.}}\sphinxbfcode{\sphinxupquote{get\_max\_width}}}{\sphinxparam{\DUrole{n}{table1}}\sphinxparamcomma \sphinxparam{\DUrole{n}{index1}}}{}
\pysigstopsignatures
\sphinxAtStartPar
Get the maximum width of the given column index

\end{fulllineitems}

\index{get\_shift\_from\_mask() (in module management)@\spxentry{get\_shift\_from\_mask()}\spxextra{in module management}}

\begin{fulllineitems}
\phantomsection\label{\detokenize{apidocs:management.get_shift_from_mask}}
\pysigstartsignatures
\pysiglinewithargsret{\sphinxcode{\sphinxupquote{management.}}\sphinxbfcode{\sphinxupquote{get\_shift\_from\_mask}}}{\sphinxparam{\DUrole{n}{mask}}}{}
\pysigstopsignatures
\sphinxAtStartPar
Get the shift value from a mask.

\end{fulllineitems}

\index{hexstring2ascii() (in module management)@\spxentry{hexstring2ascii()}\spxextra{in module management}}

\begin{fulllineitems}
\phantomsection\label{\detokenize{apidocs:management.hexstring2ascii}}
\pysigstartsignatures
\pysiglinewithargsret{\sphinxcode{\sphinxupquote{management.}}\sphinxbfcode{\sphinxupquote{hexstring2ascii}}}{\sphinxparam{\DUrole{n}{hexstring}}\sphinxparamcomma \sphinxparam{\DUrole{n}{xor}\DUrole{o}{=}\DUrole{default_value}{0}}}{}
\pysigstopsignatures
\sphinxAtStartPar
Convert a hexstring to an ASCII\sphinxhyphen{}String.
\begin{quote}\begin{description}
\sphinxlineitem{Parameters}\begin{itemize}
\item {} 
\sphinxAtStartPar
\sphinxstyleliteralstrong{\sphinxupquote{hexstring}} (\sphinxstyleliteralemphasis{\sphinxupquote{str}}) \textendash{} The input hex string.

\item {} 
\sphinxAtStartPar
\sphinxstyleliteralstrong{\sphinxupquote{xor}} (\sphinxstyleliteralemphasis{\sphinxupquote{int}}) \textendash{} XOR value as an integer (default is 0).

\end{itemize}

\sphinxlineitem{Returns}
\sphinxAtStartPar
The converted ASCII string.

\sphinxlineitem{Return type}
\sphinxAtStartPar
str

\end{description}\end{quote}

\end{fulllineitems}

\index{pprint\_table() (in module management)@\spxentry{pprint\_table()}\spxextra{in module management}}

\begin{fulllineitems}
\phantomsection\label{\detokenize{apidocs:management.pprint_table}}
\pysigstartsignatures
\pysiglinewithargsret{\sphinxcode{\sphinxupquote{management.}}\sphinxbfcode{\sphinxupquote{pprint\_table}}}{\sphinxparam{\DUrole{n}{table}}}{}
\pysigstopsignatures
\sphinxAtStartPar
Prints out a table of data, padded for alignment.
\begin{quote}\begin{description}
\sphinxlineitem{Parameters}
\sphinxAtStartPar
\sphinxstyleliteralstrong{\sphinxupquote{table}} (\sphinxstyleliteralemphasis{\sphinxupquote{list}}) \textendash{} The table to print. A list of lists.
Each row must have the same number of columns.

\end{description}\end{quote}

\end{fulllineitems}

\index{module@\spxentry{module}!subrack\_management\_board@\spxentry{subrack\_management\_board}}\index{subrack\_management\_board@\spxentry{subrack\_management\_board}!module@\spxentry{module}}\index{Adu\_Eth\_Ping() (in module subrack\_management\_board)@\spxentry{Adu\_Eth\_Ping()}\spxextra{in module subrack\_management\_board}}\phantomsection\label{\detokenize{apidocs:module-subrack_management_board}}

\begin{fulllineitems}
\phantomsection\label{\detokenize{apidocs:subrack_management_board.Adu_Eth_Ping}}
\pysigstartsignatures
\pysiglinewithargsret{\sphinxcode{\sphinxupquote{subrack\_management\_board.}}\sphinxbfcode{\sphinxupquote{Adu\_Eth\_Ping}}}{\sphinxparam{\DUrole{n}{ip}}\sphinxparamcomma \sphinxparam{\DUrole{n}{count}\DUrole{o}{=}\DUrole{default_value}{1}}\sphinxparamcomma \sphinxparam{\DUrole{n}{interval}\DUrole{o}{=}\DUrole{default_value}{\textquotesingle{}0.2\textquotesingle{}}}\sphinxparamcomma \sphinxparam{\DUrole{n}{size}\DUrole{o}{=}\DUrole{default_value}{8}}\sphinxparamcomma \sphinxparam{\DUrole{n}{wait}\DUrole{o}{=}\DUrole{default_value}{\textquotesingle{}1\textquotesingle{}}}}{}
\pysigstopsignatures
\sphinxAtStartPar
Perform a ping on a specified IP address.

\end{fulllineitems}

\index{SubrackExecFault@\spxentry{SubrackExecFault}}

\begin{fulllineitems}
\phantomsection\label{\detokenize{apidocs:subrack_management_board.SubrackExecFault}}
\pysigstartsignatures
\pysigline{\sphinxbfcode{\sphinxupquote{exception\DUrole{w}{ }}}\sphinxcode{\sphinxupquote{subrack\_management\_board.}}\sphinxbfcode{\sphinxupquote{SubrackExecFault}}}
\pysigstopsignatures
\sphinxAtStartPar
Define an exception which occurs when an error occur when
a function or class method  fails

\end{fulllineitems}

\index{SubrackInvalidCmd@\spxentry{SubrackInvalidCmd}}

\begin{fulllineitems}
\phantomsection\label{\detokenize{apidocs:subrack_management_board.SubrackInvalidCmd}}
\pysigstartsignatures
\pysigline{\sphinxbfcode{\sphinxupquote{exception\DUrole{w}{ }}}\sphinxcode{\sphinxupquote{subrack\_management\_board.}}\sphinxbfcode{\sphinxupquote{SubrackInvalidCmd}}}
\pysigstopsignatures
\sphinxAtStartPar
Define an exception which occurs when an invalid command is provided
to a function or class method

\end{fulllineitems}

\index{SubrackInvalidParameter@\spxentry{SubrackInvalidParameter}}

\begin{fulllineitems}
\phantomsection\label{\detokenize{apidocs:subrack_management_board.SubrackInvalidParameter}}
\pysigstartsignatures
\pysigline{\sphinxbfcode{\sphinxupquote{exception\DUrole{w}{ }}}\sphinxcode{\sphinxupquote{subrack\_management\_board.}}\sphinxbfcode{\sphinxupquote{SubrackInvalidParameter}}}
\pysigstopsignatures
\sphinxAtStartPar
Define an exception which occurs when an invalid parameter is provided
to a function or class method

\end{fulllineitems}

\index{SubrackMngBoard (class in subrack\_management\_board)@\spxentry{SubrackMngBoard}\spxextra{class in subrack\_management\_board}}

\begin{fulllineitems}
\phantomsection\label{\detokenize{apidocs:subrack_management_board.SubrackMngBoard}}
\pysigstartsignatures
\pysiglinewithargsret{\sphinxbfcode{\sphinxupquote{class\DUrole{w}{ }}}\sphinxcode{\sphinxupquote{subrack\_management\_board.}}\sphinxbfcode{\sphinxupquote{SubrackMngBoard}}}{\sphinxparam{\DUrole{o}{**}\DUrole{n}{kwargs}}}{}
\pysigstopsignatures
\sphinxAtStartPar
This class implements methods to manage and to monitor the subrack management board
\index{GetCPLDLockedPLL() (subrack\_management\_board.SubrackMngBoard method)@\spxentry{GetCPLDLockedPLL()}\spxextra{subrack\_management\_board.SubrackMngBoard method}}

\begin{fulllineitems}
\phantomsection\label{\detokenize{apidocs:subrack_management_board.SubrackMngBoard.GetCPLDLockedPLL}}
\pysigstartsignatures
\pysiglinewithargsret{\sphinxbfcode{\sphinxupquote{GetCPLDLockedPLL}}}{}{}
\pysigstopsignatures
\sphinxAtStartPar
This method get the status of the CPLD internal PLL Lock
:return locked: value of locked status, True PLL is locked, False PLL not locked

\end{fulllineitems}

\index{GetFanAlarm() (subrack\_management\_board.SubrackMngBoard method)@\spxentry{GetFanAlarm()}\spxextra{subrack\_management\_board.SubrackMngBoard method}}

\begin{fulllineitems}
\phantomsection\label{\detokenize{apidocs:subrack_management_board.SubrackMngBoard.GetFanAlarm}}
\pysigstartsignatures
\pysiglinewithargsret{\sphinxbfcode{\sphinxupquote{GetFanAlarm}}}{}{}
\pysigstopsignatures
\sphinxAtStartPar
method to get Fan Status Alarm Register of subrack
:return alarms: OK, WARN, ALARM, WARN\sphinxhyphen{}ALARM, of each Fan

\end{fulllineitems}

\index{GetFanMode() (subrack\_management\_board.SubrackMngBoard method)@\spxentry{GetFanMode()}\spxextra{subrack\_management\_board.SubrackMngBoard method}}

\begin{fulllineitems}
\phantomsection\label{\detokenize{apidocs:subrack_management_board.SubrackMngBoard.GetFanMode}}
\pysigstartsignatures
\pysiglinewithargsret{\sphinxbfcode{\sphinxupquote{GetFanMode}}}{\sphinxparam{\DUrole{n}{fan\_id}}}{}
\pysigstopsignatures
\sphinxAtStartPar
This method get the\_bkpln\_fan\_mode
:param fan\_id: id of the selected fan accepted value: 1\sphinxhyphen{}4
:return auto\_mode: functional fan mode: auto or manual
:return status: status of operation

\end{fulllineitems}

\index{GetFanPwm() (subrack\_management\_board.SubrackMngBoard method)@\spxentry{GetFanPwm()}\spxextra{subrack\_management\_board.SubrackMngBoard method}}

\begin{fulllineitems}
\phantomsection\label{\detokenize{apidocs:subrack_management_board.SubrackMngBoard.GetFanPwm}}
\pysigstartsignatures
\pysiglinewithargsret{\sphinxbfcode{\sphinxupquote{GetFanPwm}}}{\sphinxparam{\DUrole{n}{fan\_id}}}{}
\pysigstopsignatures
\sphinxAtStartPar
Retrieves the PWM (Pulse Width Modulation) percentage of a specified fan.

\sphinxAtStartPar
This method queries the fan speed and returns the PWM percentage.
\begin{quote}\begin{description}
\sphinxlineitem{Parameters}
\sphinxAtStartPar
\sphinxstyleliteralstrong{\sphinxupquote{fan\_id}} \textendash{} The fan identifier.

\sphinxlineitem{Returns}
\sphinxAtStartPar
The PWM (Pulse Width Modulation) percentage of the specified fan.

\end{description}\end{quote}

\end{fulllineitems}

\index{GetFanRpm() (subrack\_management\_board.SubrackMngBoard method)@\spxentry{GetFanRpm()}\spxextra{subrack\_management\_board.SubrackMngBoard method}}

\begin{fulllineitems}
\phantomsection\label{\detokenize{apidocs:subrack_management_board.SubrackMngBoard.GetFanRpm}}
\pysigstartsignatures
\pysiglinewithargsret{\sphinxbfcode{\sphinxupquote{GetFanRpm}}}{\sphinxparam{\DUrole{n}{fan\_id}}}{}
\pysigstopsignatures
\sphinxAtStartPar
Retrieves the RPM (Revolutions Per Minute) of a specified fan.

\sphinxAtStartPar
This method queries the fan speed and returns the RPM value.
\begin{quote}\begin{description}
\sphinxlineitem{Parameters}
\sphinxAtStartPar
\sphinxstyleliteralstrong{\sphinxupquote{fan\_id}} \textendash{} The fan identifier.

\sphinxlineitem{Returns}
\sphinxAtStartPar
The RPM (Revolutions Per Minute) of the specified fan.

\end{description}\end{quote}

\end{fulllineitems}

\index{GetFanSpeed() (subrack\_management\_board.SubrackMngBoard method)@\spxentry{GetFanSpeed()}\spxextra{subrack\_management\_board.SubrackMngBoard method}}

\begin{fulllineitems}
\phantomsection\label{\detokenize{apidocs:subrack_management_board.SubrackMngBoard.GetFanSpeed}}
\pysigstartsignatures
\pysiglinewithargsret{\sphinxbfcode{\sphinxupquote{GetFanSpeed}}}{\sphinxparam{\DUrole{n}{fan\_id}}}{}
\pysigstopsignatures
\sphinxAtStartPar
This method get the get\_bkpln\_fan\_speed
:param fan\_id: id of the selected fan accepted value: 1\sphinxhyphen{}4
:return fanrpm: fan rpm value
:return fan\_bank\_pwm: pwm value of selected fan

\end{fulllineitems}

\index{GetLockedPLL() (subrack\_management\_board.SubrackMngBoard method)@\spxentry{GetLockedPLL()}\spxextra{subrack\_management\_board.SubrackMngBoard method}}

\begin{fulllineitems}
\phantomsection\label{\detokenize{apidocs:subrack_management_board.SubrackMngBoard.GetLockedPLL}}
\pysigstartsignatures
\pysiglinewithargsret{\sphinxbfcode{\sphinxupquote{GetLockedPLL}}}{}{}
\pysigstopsignatures
\sphinxAtStartPar
This method get the status of the PLL Lock
:return locked: value of locked status, True PLL is locked, False PLL not locked

\end{fulllineitems}

\index{GetPSFanSpeed() (subrack\_management\_board.SubrackMngBoard method)@\spxentry{GetPSFanSpeed()}\spxextra{subrack\_management\_board.SubrackMngBoard method}}

\begin{fulllineitems}
\phantomsection\label{\detokenize{apidocs:subrack_management_board.SubrackMngBoard.GetPSFanSpeed}}
\pysigstartsignatures
\pysiglinewithargsret{\sphinxbfcode{\sphinxupquote{GetPSFanSpeed}}}{\sphinxparam{\DUrole{n}{ps\_id}}}{}
\pysigstopsignatures
\sphinxAtStartPar
This method get the fan speed of selected Power Supply of subrack
:param ps\_id: id of the selected power supply, accepted value: 1,2
:return fanspeed: speed of the fan

\end{fulllineitems}

\index{GetPSIout() (subrack\_management\_board.SubrackMngBoard method)@\spxentry{GetPSIout()}\spxextra{subrack\_management\_board.SubrackMngBoard method}}

\begin{fulllineitems}
\phantomsection\label{\detokenize{apidocs:subrack_management_board.SubrackMngBoard.GetPSIout}}
\pysigstartsignatures
\pysiglinewithargsret{\sphinxbfcode{\sphinxupquote{GetPSIout}}}{\sphinxparam{\DUrole{n}{ps\_id}}}{}
\pysigstopsignatures
\sphinxAtStartPar
This method get the Iout current value of selected Power Supply of subrack
:param ps\_id: id of the selected power supply, accepted value: 1,2
:return vout: value of Iout in Ampere

\end{fulllineitems}

\index{GetPSPower() (subrack\_management\_board.SubrackMngBoard method)@\spxentry{GetPSPower()}\spxextra{subrack\_management\_board.SubrackMngBoard method}}

\begin{fulllineitems}
\phantomsection\label{\detokenize{apidocs:subrack_management_board.SubrackMngBoard.GetPSPower}}
\pysigstartsignatures
\pysiglinewithargsret{\sphinxbfcode{\sphinxupquote{GetPSPower}}}{\sphinxparam{\DUrole{n}{ps\_id}}}{}
\pysigstopsignatures
\sphinxAtStartPar
This method get the Power consumption value of selected Power Supply of subrack
:param ps\_id: id of the selected power supply, accepted value: 1,2
:return power: value of power in W

\end{fulllineitems}

\index{GetPSVout() (subrack\_management\_board.SubrackMngBoard method)@\spxentry{GetPSVout()}\spxextra{subrack\_management\_board.SubrackMngBoard method}}

\begin{fulllineitems}
\phantomsection\label{\detokenize{apidocs:subrack_management_board.SubrackMngBoard.GetPSVout}}
\pysigstartsignatures
\pysiglinewithargsret{\sphinxbfcode{\sphinxupquote{GetPSVout}}}{\sphinxparam{\DUrole{n}{ps\_id}}}{}
\pysigstopsignatures
\sphinxAtStartPar
This method get the Vout voltage value of selected Power Supply of subrack
:param ps\_id: id of the selected power supply, accepted value: 1,2
:return vout: value of Vout in Volt

\end{fulllineitems}

\index{GetPingCpld() (subrack\_management\_board.SubrackMngBoard method)@\spxentry{GetPingCpld()}\spxextra{subrack\_management\_board.SubrackMngBoard method}}

\begin{fulllineitems}
\phantomsection\label{\detokenize{apidocs:subrack_management_board.SubrackMngBoard.GetPingCpld}}
\pysigstartsignatures
\pysiglinewithargsret{\sphinxbfcode{\sphinxupquote{GetPingCpld}}}{}{}
\pysigstopsignatures
\sphinxAtStartPar
Checks the connectivity to the CPLD using a ping operation.

\sphinxAtStartPar
This method checks if the CPLD is reachable via ping.
\begin{quote}\begin{description}
\sphinxlineitem{Returns}
\sphinxAtStartPar
True if the CPLD is reachable via ping, False otherwise.

\end{description}\end{quote}

\end{fulllineitems}

\index{GetPingTPM() (subrack\_management\_board.SubrackMngBoard method)@\spxentry{GetPingTPM()}\spxextra{subrack\_management\_board.SubrackMngBoard method}}

\begin{fulllineitems}
\phantomsection\label{\detokenize{apidocs:subrack_management_board.SubrackMngBoard.GetPingTPM}}
\pysigstartsignatures
\pysiglinewithargsret{\sphinxbfcode{\sphinxupquote{GetPingTPM}}}{\sphinxparam{\DUrole{n}{tpm\_slot\_id}}}{}
\pysigstopsignatures
\sphinxAtStartPar
Checks the connectivity to a TPM using a ping operation.

\sphinxAtStartPar
This method checks if the TPM in the specified slot is present, powered on,
and responds to a ping operation.
\begin{quote}\begin{description}
\sphinxlineitem{Parameters}
\sphinxAtStartPar
\sphinxstyleliteralstrong{\sphinxupquote{tpm\_slot\_id}} \textendash{} The TPM slot identifier.

\sphinxlineitem{Returns}
\sphinxAtStartPar
True if the TPM is reachable via ping, False if unreachable,
None if TPM is not present or not powered on.

\end{description}\end{quote}

\end{fulllineitems}

\index{GetPllSource() (subrack\_management\_board.SubrackMngBoard method)@\spxentry{GetPllSource()}\spxextra{subrack\_management\_board.SubrackMngBoard method}}

\begin{fulllineitems}
\phantomsection\label{\detokenize{apidocs:subrack_management_board.SubrackMngBoard.GetPllSource}}
\pysigstartsignatures
\pysiglinewithargsret{\sphinxbfcode{\sphinxupquote{GetPllSource}}}{}{}
\pysigstopsignatures
\sphinxAtStartPar
Retrieves the PLL (Phase\sphinxhyphen{}Locked Loop) source.

\sphinxAtStartPar
This method reads the PLL source from the management interface.
\begin{quote}\begin{description}
\sphinxlineitem{Returns}
\sphinxAtStartPar
“internal” if the PLL source is internal, “external” otherwise.

\end{description}\end{quote}

\end{fulllineitems}

\index{GetPowerAlarm() (subrack\_management\_board.SubrackMngBoard method)@\spxentry{GetPowerAlarm()}\spxextra{subrack\_management\_board.SubrackMngBoard method}}

\begin{fulllineitems}
\phantomsection\label{\detokenize{apidocs:subrack_management_board.SubrackMngBoard.GetPowerAlarm}}
\pysigstartsignatures
\pysiglinewithargsret{\sphinxbfcode{\sphinxupquote{GetPowerAlarm}}}{}{}
\pysigstopsignatures
\sphinxAtStartPar
method to get TPM Power consumption Alarm Register of subrack
:return alarms: status vector, OK, WARN, ALM of each TPM Voltages Alarm, for each board

\end{fulllineitems}

\index{GetSubrackTemperatures() (subrack\_management\_board.SubrackMngBoard method)@\spxentry{GetSubrackTemperatures()}\spxextra{subrack\_management\_board.SubrackMngBoard method}}

\begin{fulllineitems}
\phantomsection\label{\detokenize{apidocs:subrack_management_board.SubrackMngBoard.GetSubrackTemperatures}}
\pysigstartsignatures
\pysiglinewithargsret{\sphinxbfcode{\sphinxupquote{GetSubrackTemperatures}}}{}{}
\pysigstopsignatures
\sphinxAtStartPar
method to get temperatures from sensors placed on backplane and subrack\sphinxhyphen{}management boards
:return temp\_mng1: temperature value of management sensor 1
:return temp\_mng2: temperature value of management sensor 2
:return temp\_bck1: temperature value of backplane sensor 1
:return temp\_bck2: temperature value of backplane sensor 2

\end{fulllineitems}

\index{GetTPMCurrent() (subrack\_management\_board.SubrackMngBoard method)@\spxentry{GetTPMCurrent()}\spxextra{subrack\_management\_board.SubrackMngBoard method}}

\begin{fulllineitems}
\phantomsection\label{\detokenize{apidocs:subrack_management_board.SubrackMngBoard.GetTPMCurrent}}
\pysigstartsignatures
\pysiglinewithargsret{\sphinxbfcode{\sphinxupquote{GetTPMCurrent}}}{\sphinxparam{\DUrole{n}{tpm\_slot\_id}}\sphinxparamcomma \sphinxparam{\DUrole{n}{force}\DUrole{o}{=}\DUrole{default_value}{True}}}{}
\pysigstopsignatures
\sphinxAtStartPar
method to get current consuptin of selected tpm (providing subrack index slot of tpm)
:param tpm\_slot\_id: subrack slot index for selected TPM, accepted value 1\sphinxhyphen{}8
:param force: force the operation even if no TPM is present in selected slot

\end{fulllineitems}

\index{GetTPMGlobalStatusAlarm() (subrack\_management\_board.SubrackMngBoard method)@\spxentry{GetTPMGlobalStatusAlarm()}\spxextra{subrack\_management\_board.SubrackMngBoard method}}

\begin{fulllineitems}
\phantomsection\label{\detokenize{apidocs:subrack_management_board.SubrackMngBoard.GetTPMGlobalStatusAlarm}}
\pysigstartsignatures
\pysiglinewithargsret{\sphinxbfcode{\sphinxupquote{GetTPMGlobalStatusAlarm}}}{\sphinxparam{\DUrole{n}{tpm\_slot\_id}}\sphinxparamcomma \sphinxparam{\DUrole{n}{forceread}\DUrole{o}{=}\DUrole{default_value}{False}}}{}
\pysigstopsignatures
\sphinxAtStartPar
Deprecated method to retrieve TPM global status alarms.
\begin{quote}\begin{description}
\sphinxlineitem{Parameters}\begin{itemize}
\item {} 
\sphinxAtStartPar
\sphinxstyleliteralstrong{\sphinxupquote{tpm\_slot\_id}} \textendash{} The TPM slot identifier.

\item {} 
\sphinxAtStartPar
\sphinxstyleliteralstrong{\sphinxupquote{forceread}} \textendash{} (Optional) If True, forces a read even if deprecated.

\end{itemize}

\sphinxlineitem{Raises}
\sphinxAtStartPar
{\hyperref[\detokenize{apidocs:subrack_management_board.SubrackExecFault}]{\sphinxcrossref{\sphinxstyleliteralstrong{\sphinxupquote{SubrackExecFault}}}}} \textendash{} Indicates that the method is deprecated, and
subrack no longer accesses TPM.

\end{description}\end{quote}

\end{fulllineitems}

\index{GetTPMIP() (subrack\_management\_board.SubrackMngBoard method)@\spxentry{GetTPMIP()}\spxextra{subrack\_management\_board.SubrackMngBoard method}}

\begin{fulllineitems}
\phantomsection\label{\detokenize{apidocs:subrack_management_board.SubrackMngBoard.GetTPMIP}}
\pysigstartsignatures
\pysiglinewithargsret{\sphinxbfcode{\sphinxupquote{GetTPMIP}}}{\sphinxparam{\DUrole{n}{tpm\_slot\_id}}}{}
\pysigstopsignatures
\sphinxAtStartPar
method to manually set volatile local ip address of a TPM board present on subrack
:param tpm\_slot\_id:subrack slot index for selected TPM, accepted value 1\sphinxhyphen{}8
:return tpm\_ip\_str: tpm ip address

\end{fulllineitems}

\index{GetTPMInfo() (subrack\_management\_board.SubrackMngBoard method)@\spxentry{GetTPMInfo()}\spxextra{subrack\_management\_board.SubrackMngBoard method}}

\begin{fulllineitems}
\phantomsection\label{\detokenize{apidocs:subrack_management_board.SubrackMngBoard.GetTPMInfo}}
\pysigstartsignatures
\pysiglinewithargsret{\sphinxbfcode{\sphinxupquote{GetTPMInfo}}}{\sphinxparam{\DUrole{n}{tpm\_slot\_id}}\sphinxparamcomma \sphinxparam{\DUrole{n}{forceread}\DUrole{o}{=}\DUrole{default_value}{False}}}{}
\pysigstopsignatures
\sphinxAtStartPar
Deprecated method to retrieve TPM information.
\begin{quote}\begin{description}
\sphinxlineitem{Parameters}\begin{itemize}
\item {} 
\sphinxAtStartPar
\sphinxstyleliteralstrong{\sphinxupquote{tpm\_slot\_id}} \textendash{} The TPM slot identifier.

\item {} 
\sphinxAtStartPar
\sphinxstyleliteralstrong{\sphinxupquote{forceread}} \textendash{} (Optional) If True, forces a read even if deprecated.

\end{itemize}

\sphinxlineitem{Raises}
\sphinxAtStartPar
{\hyperref[\detokenize{apidocs:subrack_management_board.SubrackExecFault}]{\sphinxcrossref{\sphinxstyleliteralstrong{\sphinxupquote{SubrackExecFault}}}}} \textendash{} Indicates that the method is deprecated, and
subrack no longer accesses TPM.

\end{description}\end{quote}

\end{fulllineitems}

\index{GetTPMMCUTemperature() (subrack\_management\_board.SubrackMngBoard method)@\spxentry{GetTPMMCUTemperature()}\spxextra{subrack\_management\_board.SubrackMngBoard method}}

\begin{fulllineitems}
\phantomsection\label{\detokenize{apidocs:subrack_management_board.SubrackMngBoard.GetTPMMCUTemperature}}
\pysigstartsignatures
\pysiglinewithargsret{\sphinxbfcode{\sphinxupquote{GetTPMMCUTemperature}}}{\sphinxparam{\DUrole{n}{tpm\_slot\_id}}\sphinxparamcomma \sphinxparam{\DUrole{n}{forceread}\DUrole{o}{=}\DUrole{default_value}{False}}}{}
\pysigstopsignatures
\sphinxAtStartPar
Deprecated method to retrieve TPM MCU temperature.
\begin{quote}\begin{description}
\sphinxlineitem{Parameters}\begin{itemize}
\item {} 
\sphinxAtStartPar
\sphinxstyleliteralstrong{\sphinxupquote{tpm\_slot\_id}} \textendash{} The TPM slot identifier.

\item {} 
\sphinxAtStartPar
\sphinxstyleliteralstrong{\sphinxupquote{forceread}} \textendash{} (Optional) If True, forces a read even if deprecated.

\end{itemize}

\sphinxlineitem{Raises}
\sphinxAtStartPar
{\hyperref[\detokenize{apidocs:subrack_management_board.SubrackExecFault}]{\sphinxcrossref{\sphinxstyleliteralstrong{\sphinxupquote{SubrackExecFault}}}}} \textendash{} Indicates that the method is deprecated, and
subrack no longer accesses TPM.

\end{description}\end{quote}

\end{fulllineitems}

\index{GetTPMOnOffVect() (subrack\_management\_board.SubrackMngBoard method)@\spxentry{GetTPMOnOffVect()}\spxextra{subrack\_management\_board.SubrackMngBoard method}}

\begin{fulllineitems}
\phantomsection\label{\detokenize{apidocs:subrack_management_board.SubrackMngBoard.GetTPMOnOffVect}}
\pysigstartsignatures
\pysiglinewithargsret{\sphinxbfcode{\sphinxupquote{GetTPMOnOffVect}}}{}{}
\pysigstopsignatures
\sphinxAtStartPar
method to get Power On status of inserted tpm, 0 off or not present, 1 power on
:return vector of poweron status for slots, bits 7:0, bit 0 slot 1, bit 7 slot 8, 1 TPM power on, 0 no TPM
inserted or power off

\end{fulllineitems}

\index{GetTPMPower() (subrack\_management\_board.SubrackMngBoard method)@\spxentry{GetTPMPower()}\spxextra{subrack\_management\_board.SubrackMngBoard method}}

\begin{fulllineitems}
\phantomsection\label{\detokenize{apidocs:subrack_management_board.SubrackMngBoard.GetTPMPower}}
\pysigstartsignatures
\pysiglinewithargsret{\sphinxbfcode{\sphinxupquote{GetTPMPower}}}{\sphinxparam{\DUrole{n}{tpm\_slot\_id}}\sphinxparamcomma \sphinxparam{\DUrole{n}{force}\DUrole{o}{=}\DUrole{default_value}{True}}}{}
\pysigstopsignatures
\sphinxAtStartPar
method to get power consumption of selected tpm (providing subrack index slot of tpm)
:param tpm\_slot\_id: subrack slot index for selected TPM, accepted value 1\sphinxhyphen{}8
:param force force the operation even if no TPM is present in selected slot

\end{fulllineitems}

\index{GetTPMPresent() (subrack\_management\_board.SubrackMngBoard method)@\spxentry{GetTPMPresent()}\spxextra{subrack\_management\_board.SubrackMngBoard method}}

\begin{fulllineitems}
\phantomsection\label{\detokenize{apidocs:subrack_management_board.SubrackMngBoard.GetTPMPresent}}
\pysigstartsignatures
\pysiglinewithargsret{\sphinxbfcode{\sphinxupquote{GetTPMPresent}}}{\sphinxparam{\DUrole{n}{tpm\_slot\_id}\DUrole{o}{=}\DUrole{default_value}{None}}}{}
\pysigstopsignatures
\sphinxAtStartPar
brief method to get info about TPM board present on subrack
:return TpmDetected: vector of tpm positional,1 TPM detected,0 no TPM inserted,bit 7:0,bit 0 slot 1,bit 7 slot 8

\end{fulllineitems}

\index{GetTPMSupplyFault() (subrack\_management\_board.SubrackMngBoard method)@\spxentry{GetTPMSupplyFault()}\spxextra{subrack\_management\_board.SubrackMngBoard method}}

\begin{fulllineitems}
\phantomsection\label{\detokenize{apidocs:subrack_management_board.SubrackMngBoard.GetTPMSupplyFault}}
\pysigstartsignatures
\pysiglinewithargsret{\sphinxbfcode{\sphinxupquote{GetTPMSupplyFault}}}{}{}
\pysigstopsignatures
\sphinxAtStartPar
Method to get info about TPM supply fault status, 1 for each TPM in backplane slot
:return tpmsupplyfault: vector of tpm supply fault status, 1 fault, 0 no fault,bit 7:0,bit 0 slot 1,bit 7 slot 8

\end{fulllineitems}

\index{GetTPMTemperatures() (subrack\_management\_board.SubrackMngBoard method)@\spxentry{GetTPMTemperatures()}\spxextra{subrack\_management\_board.SubrackMngBoard method}}

\begin{fulllineitems}
\phantomsection\label{\detokenize{apidocs:subrack_management_board.SubrackMngBoard.GetTPMTemperatures}}
\pysigstartsignatures
\pysiglinewithargsret{\sphinxbfcode{\sphinxupquote{GetTPMTemperatures}}}{\sphinxparam{\DUrole{n}{tpm\_slot\_id}}\sphinxparamcomma \sphinxparam{\DUrole{n}{forceread}\DUrole{o}{=}\DUrole{default_value}{False}}}{}
\pysigstopsignatures
\sphinxAtStartPar
Deprecated method to retrieve TPM temperatures.
\begin{quote}\begin{description}
\sphinxlineitem{Parameters}\begin{itemize}
\item {} 
\sphinxAtStartPar
\sphinxstyleliteralstrong{\sphinxupquote{tpm\_slot\_id}} \textendash{} The TPM slot identifier.

\item {} 
\sphinxAtStartPar
\sphinxstyleliteralstrong{\sphinxupquote{forceread}} \textendash{} (Optional) If True, forces a read even if deprecated.

\end{itemize}

\sphinxlineitem{Raises}
\sphinxAtStartPar
{\hyperref[\detokenize{apidocs:subrack_management_board.SubrackExecFault}]{\sphinxcrossref{\sphinxstyleliteralstrong{\sphinxupquote{SubrackExecFault}}}}} \textendash{} Indicates that the method is deprecated, and
subrack no longer accesses TPM.

\end{description}\end{quote}

\end{fulllineitems}

\index{GetTPMVoltage() (subrack\_management\_board.SubrackMngBoard method)@\spxentry{GetTPMVoltage()}\spxextra{subrack\_management\_board.SubrackMngBoard method}}

\begin{fulllineitems}
\phantomsection\label{\detokenize{apidocs:subrack_management_board.SubrackMngBoard.GetTPMVoltage}}
\pysigstartsignatures
\pysiglinewithargsret{\sphinxbfcode{\sphinxupquote{GetTPMVoltage}}}{\sphinxparam{\DUrole{n}{tpm\_slot\_id}}\sphinxparamcomma \sphinxparam{\DUrole{n}{force}\DUrole{o}{=}\DUrole{default_value}{True}}}{}
\pysigstopsignatures
\sphinxAtStartPar
brief method to get power consuptin of selected tpm (providing subrack index slot of tpm)
:param  tpm\_slot\_id:subrack slot index for selected TPM, accepted value 1\sphinxhyphen{}8
:param force: force the operation even if no TPM is present in selected slot

\end{fulllineitems}

\index{GetTPM\_Add\_List() (subrack\_management\_board.SubrackMngBoard method)@\spxentry{GetTPM\_Add\_List()}\spxextra{subrack\_management\_board.SubrackMngBoard method}}

\begin{fulllineitems}
\phantomsection\label{\detokenize{apidocs:subrack_management_board.SubrackMngBoard.GetTPM_Add_List}}
\pysigstartsignatures
\pysiglinewithargsret{\sphinxbfcode{\sphinxupquote{GetTPM\_Add\_List}}}{}{}
\pysigstopsignatures
\sphinxAtStartPar
method to get the IP address will be assigned to each TPM board present on subrack
:return list of IP address will be assigned assigned

\end{fulllineitems}

\index{GetUPSStatus() (subrack\_management\_board.SubrackMngBoard method)@\spxentry{GetUPSStatus()}\spxextra{subrack\_management\_board.SubrackMngBoard method}}

\begin{fulllineitems}
\phantomsection\label{\detokenize{apidocs:subrack_management_board.SubrackMngBoard.GetUPSStatus}}
\pysigstartsignatures
\pysiglinewithargsret{\sphinxbfcode{\sphinxupquote{GetUPSStatus}}}{}{}
\pysigstopsignatures
\sphinxAtStartPar
Retrieves the status of the uninterruptible power supply (UPS).

\sphinxAtStartPar
This method reads data from the serial port to update UPS charge registers and
ADC (Analog\sphinxhyphen{}to\sphinxhyphen{}Digital Converter) values. It also checks for warning and alarm
conditions based on voltage levels.
\begin{quote}\begin{description}
\sphinxlineitem{Returns}
\sphinxAtStartPar
A dictionary containing the UPS status information, including:
\sphinxhyphen{} ‘warning’: True if there is a warning condition, False otherwise.
\sphinxhyphen{} ‘alarm’: True if there is an alarm condition, False otherwise.
\sphinxhyphen{} ‘charging’: True if the UPS is currently charging, False otherwise.

\end{description}\end{quote}

\end{fulllineitems}

\index{GetVoltageAlarm() (subrack\_management\_board.SubrackMngBoard method)@\spxentry{GetVoltageAlarm()}\spxextra{subrack\_management\_board.SubrackMngBoard method}}

\begin{fulllineitems}
\phantomsection\label{\detokenize{apidocs:subrack_management_board.SubrackMngBoard.GetVoltageAlarm}}
\pysigstartsignatures
\pysiglinewithargsret{\sphinxbfcode{\sphinxupquote{GetVoltageAlarm}}}{}{}
\pysigstopsignatures
\sphinxAtStartPar
method to get TPM Voltages Power supply Alarm Register of subrack
:return alarms: status vector, OK, WARN, ALM of each TPM Voltages Alarm, for each board

\end{fulllineitems}

\index{Get\_API\_version() (subrack\_management\_board.SubrackMngBoard method)@\spxentry{Get\_API\_version()}\spxextra{subrack\_management\_board.SubrackMngBoard method}}

\begin{fulllineitems}
\phantomsection\label{\detokenize{apidocs:subrack_management_board.SubrackMngBoard.Get_API_version}}
\pysigstartsignatures
\pysiglinewithargsret{\sphinxbfcode{\sphinxupquote{Get\_API\_version}}}{}{}
\pysigstopsignatures
\sphinxAtStartPar
method to get the Version of the API
:return string with API version

\end{fulllineitems}

\index{Get\_Subrack\_TimeTS() (subrack\_management\_board.SubrackMngBoard method)@\spxentry{Get\_Subrack\_TimeTS()}\spxextra{subrack\_management\_board.SubrackMngBoard method}}

\begin{fulllineitems}
\phantomsection\label{\detokenize{apidocs:subrack_management_board.SubrackMngBoard.Get_Subrack_TimeTS}}
\pysigstartsignatures
\pysiglinewithargsret{\sphinxbfcode{\sphinxupquote{Get\_Subrack\_TimeTS}}}{}{}
\pysigstopsignatures
\sphinxAtStartPar
method to get the subrack Time in timestamp format
:return time in timestamp format

\end{fulllineitems}

\index{Get\_TPM\_temperature\_vector() (subrack\_management\_board.SubrackMngBoard method)@\spxentry{Get\_TPM\_temperature\_vector()}\spxextra{subrack\_management\_board.SubrackMngBoard method}}

\begin{fulllineitems}
\phantomsection\label{\detokenize{apidocs:subrack_management_board.SubrackMngBoard.Get_TPM_temperature_vector}}
\pysigstartsignatures
\pysiglinewithargsret{\sphinxbfcode{\sphinxupquote{Get\_TPM\_temperature\_vector}}}{}{}
\pysigstopsignatures
\sphinxAtStartPar
Deprecated method to retrieve TPM temperature vector.
\begin{quote}\begin{description}
\sphinxlineitem{Raises}
\sphinxAtStartPar
{\hyperref[\detokenize{apidocs:subrack_management_board.SubrackExecFault}]{\sphinxcrossref{\sphinxstyleliteralstrong{\sphinxupquote{SubrackExecFault}}}}} \textendash{} Indicates that the method is deprecated, and
subrack no longer accesses TPM.

\end{description}\end{quote}

\end{fulllineitems}

\index{Get\_tpm\_alarms\_vector() (subrack\_management\_board.SubrackMngBoard method)@\spxentry{Get\_tpm\_alarms\_vector()}\spxextra{subrack\_management\_board.SubrackMngBoard method}}

\begin{fulllineitems}
\phantomsection\label{\detokenize{apidocs:subrack_management_board.SubrackMngBoard.Get_tpm_alarms_vector}}
\pysigstartsignatures
\pysiglinewithargsret{\sphinxbfcode{\sphinxupquote{Get\_tpm\_alarms\_vector}}}{}{}
\pysigstopsignatures
\sphinxAtStartPar
Deprecated method to retrieve TPM alarms vector.
\begin{quote}\begin{description}
\sphinxlineitem{Raises}
\sphinxAtStartPar
{\hyperref[\detokenize{apidocs:subrack_management_board.SubrackExecFault}]{\sphinxcrossref{\sphinxstyleliteralstrong{\sphinxupquote{SubrackExecFault}}}}} \textendash{} Indicates that the method is deprecated, and
subrack no longer accesses TPM.

\end{description}\end{quote}

\end{fulllineitems}

\index{Initialize() (subrack\_management\_board.SubrackMngBoard method)@\spxentry{Initialize()}\spxextra{subrack\_management\_board.SubrackMngBoard method}}

\begin{fulllineitems}
\phantomsection\label{\detokenize{apidocs:subrack_management_board.SubrackMngBoard.Initialize}}
\pysigstartsignatures
\pysiglinewithargsret{\sphinxbfcode{\sphinxupquote{Initialize}}}{\sphinxparam{\DUrole{n}{pll\_source\_internal}\DUrole{o}{=}\DUrole{default_value}{False}}}{}
\pysigstopsignatures
\sphinxAtStartPar
Initialize the Subrack.

\sphinxAtStartPar
Parameters:
\sphinxhyphen{} pll\_source\_internal (bool): Set to True to use internal PLL source.

\end{fulllineitems}

\index{PllInitialize() (subrack\_management\_board.SubrackMngBoard method)@\spxentry{PllInitialize()}\spxextra{subrack\_management\_board.SubrackMngBoard method}}

\begin{fulllineitems}
\phantomsection\label{\detokenize{apidocs:subrack_management_board.SubrackMngBoard.PllInitialize}}
\pysigstartsignatures
\pysiglinewithargsret{\sphinxbfcode{\sphinxupquote{PllInitialize}}}{\sphinxparam{\DUrole{n}{source\_internal}\DUrole{o}{=}\DUrole{default_value}{False}}\sphinxparamcomma \sphinxparam{\DUrole{n}{pll\_cfg\_file}\DUrole{o}{=}\DUrole{default_value}{None}}}{}
\pysigstopsignatures
\sphinxAtStartPar
This method initialize the PLL

\end{fulllineitems}

\index{PowerOffTPM() (subrack\_management\_board.SubrackMngBoard method)@\spxentry{PowerOffTPM()}\spxextra{subrack\_management\_board.SubrackMngBoard method}}

\begin{fulllineitems}
\phantomsection\label{\detokenize{apidocs:subrack_management_board.SubrackMngBoard.PowerOffTPM}}
\pysigstartsignatures
\pysiglinewithargsret{\sphinxbfcode{\sphinxupquote{PowerOffTPM}}}{\sphinxparam{\DUrole{n}{tpm\_slot\_id}}\sphinxparamcomma \sphinxparam{\DUrole{n}{force}\DUrole{o}{=}\DUrole{default_value}{False}}}{}
\pysigstopsignatures
\sphinxAtStartPar
method to power off selected tpm
:param  tpm\_slot\_id: subrack slot index for selected TPM, accepted value 1\sphinxhyphen{}8
:param force: force the operation even if no TPM is present in selected slot

\end{fulllineitems}

\index{SetFanMode() (subrack\_management\_board.SubrackMngBoard method)@\spxentry{SetFanMode()}\spxextra{subrack\_management\_board.SubrackMngBoard method}}

\begin{fulllineitems}
\phantomsection\label{\detokenize{apidocs:subrack_management_board.SubrackMngBoard.SetFanMode}}
\pysigstartsignatures
\pysiglinewithargsret{\sphinxbfcode{\sphinxupquote{SetFanMode}}}{\sphinxparam{\DUrole{n}{fan\_id\_blk}}\sphinxparamcomma \sphinxparam{\DUrole{n}{auto\_mode}}}{}
\pysigstopsignatures
\sphinxAtStartPar
This method set the fan mode
:param fan\_id\_blk: id of the fan cuople accepted value: 1\sphinxhyphen{}4, for fan 1,2; 3 for fan 3,4
:param auto\_mode: fan mode configuration, 1 auto(controlled by MCU), 0 manual(use SetFanSpeed method)
:note fan are coupled, passing fan\_blk\_id=1 both fan 1 and 2 will be configured at same mode,

\end{fulllineitems}

\index{SetFanSpeed() (subrack\_management\_board.SubrackMngBoard method)@\spxentry{SetFanSpeed()}\spxextra{subrack\_management\_board.SubrackMngBoard method}}

\begin{fulllineitems}
\phantomsection\label{\detokenize{apidocs:subrack_management_board.SubrackMngBoard.SetFanSpeed}}
\pysigstartsignatures
\pysiglinewithargsret{\sphinxbfcode{\sphinxupquote{SetFanSpeed}}}{\sphinxparam{\DUrole{n}{fan\_id}}\sphinxparamcomma \sphinxparam{\DUrole{n}{speed\_pwm\_perc}}}{}
\pysigstopsignatures
\sphinxAtStartPar
This method set the\_bkpln\_fan\_speed
:param fan\_id: id of the selected fan accepted value: 1\sphinxhyphen{}4
:param speed\_pwm\_perc: percentage value of fan RPM  (MIN 0=0\% \sphinxhyphen{} MAX 100=100\%)
:note settings of fan speed is possible only if fan mode is manual

\end{fulllineitems}

\index{SetPSFanSpeed() (subrack\_management\_board.SubrackMngBoard method)@\spxentry{SetPSFanSpeed()}\spxextra{subrack\_management\_board.SubrackMngBoard method}}

\begin{fulllineitems}
\phantomsection\label{\detokenize{apidocs:subrack_management_board.SubrackMngBoard.SetPSFanSpeed}}
\pysigstartsignatures
\pysiglinewithargsret{\sphinxbfcode{\sphinxupquote{SetPSFanSpeed}}}{\sphinxparam{\DUrole{n}{ps\_id}}\sphinxparamcomma \sphinxparam{\DUrole{n}{speed\_percent}}}{}
\pysigstopsignatures
\sphinxAtStartPar
This method set the fan speed of selected Power Supply of subrack
:param ps\_id: id of the selected power supply, accepted value: 1,2
:param speed\_percent: speed in percentual value from 0 to 100

\end{fulllineitems}

\index{SetTPMIP() (subrack\_management\_board.SubrackMngBoard method)@\spxentry{SetTPMIP()}\spxextra{subrack\_management\_board.SubrackMngBoard method}}

\begin{fulllineitems}
\phantomsection\label{\detokenize{apidocs:subrack_management_board.SubrackMngBoard.SetTPMIP}}
\pysigstartsignatures
\pysiglinewithargsret{\sphinxbfcode{\sphinxupquote{SetTPMIP}}}{\sphinxparam{\DUrole{n}{tpm\_slot\_id}}\sphinxparamcomma \sphinxparam{\DUrole{n}{ip}}\sphinxparamcomma \sphinxparam{\DUrole{n}{netmask}}\sphinxparamcomma \sphinxparam{\DUrole{n}{gateway}\DUrole{o}{=}\DUrole{default_value}{None}}\sphinxparamcomma \sphinxparam{\DUrole{n}{bypass\_check}\DUrole{o}{=}\DUrole{default_value}{False}}\sphinxparamcomma \sphinxparam{\DUrole{n}{timeout}\DUrole{o}{=}\DUrole{default_value}{100}}}{}
\pysigstopsignatures
\sphinxAtStartPar
method to manually set volatile local ip address of a TPM board present on subrack
:param tpm\_slot\_id: subrack slot index for selected TPM, accepted value 1\sphinxhyphen{}8
:param ip: ip address will be assigned to selected TPM
:param netmask: netmask value will be assigned to selected TPM
:return status

\end{fulllineitems}

\index{SetUPSVoltageAlarmThresholds() (subrack\_management\_board.SubrackMngBoard method)@\spxentry{SetUPSVoltageAlarmThresholds()}\spxextra{subrack\_management\_board.SubrackMngBoard method}}

\begin{fulllineitems}
\phantomsection\label{\detokenize{apidocs:subrack_management_board.SubrackMngBoard.SetUPSVoltageAlarmThresholds}}
\pysigstartsignatures
\pysiglinewithargsret{\sphinxbfcode{\sphinxupquote{SetUPSVoltageAlarmThresholds}}}{\sphinxparam{\DUrole{n}{alarm\_level}}}{}
\pysigstopsignatures
\sphinxAtStartPar
Sets the UPS voltage alarm thresholds.

\sphinxAtStartPar
This method sets the alarm level for UPS voltage, considering the current
UPS presence status.
\begin{quote}\begin{description}
\sphinxlineitem{Parameters}
\sphinxAtStartPar
\sphinxstyleliteralstrong{\sphinxupquote{alarm\_level}} \textendash{} The alarm level for UPS voltage.

\sphinxlineitem{Returns}
\sphinxAtStartPar
The number of errors encountered during the operation.

\end{description}\end{quote}

\end{fulllineitems}

\index{SetUPSVoltageWarningThresholds() (subrack\_management\_board.SubrackMngBoard method)@\spxentry{SetUPSVoltageWarningThresholds()}\spxextra{subrack\_management\_board.SubrackMngBoard method}}

\begin{fulllineitems}
\phantomsection\label{\detokenize{apidocs:subrack_management_board.SubrackMngBoard.SetUPSVoltageWarningThresholds}}
\pysigstartsignatures
\pysiglinewithargsret{\sphinxbfcode{\sphinxupquote{SetUPSVoltageWarningThresholds}}}{\sphinxparam{\DUrole{n}{warning\_level}}}{}
\pysigstopsignatures
\sphinxAtStartPar
Sets the UPS voltage warning thresholds.

\sphinxAtStartPar
This method sets the warning level for UPS voltage, considering the current
UPS presence status.
\begin{quote}\begin{description}
\sphinxlineitem{Parameters}
\sphinxAtStartPar
\sphinxstyleliteralstrong{\sphinxupquote{warning\_level}} \textendash{} The warning level for UPS voltage.

\sphinxlineitem{Returns}
\sphinxAtStartPar
The number of errors encountered during the operation.

\end{description}\end{quote}

\end{fulllineitems}

\index{SubrackInitialConfiguration() (subrack\_management\_board.SubrackMngBoard method)@\spxentry{SubrackInitialConfiguration()}\spxextra{subrack\_management\_board.SubrackMngBoard method}}

\begin{fulllineitems}
\phantomsection\label{\detokenize{apidocs:subrack_management_board.SubrackMngBoard.SubrackInitialConfiguration}}
\pysigstartsignatures
\pysiglinewithargsret{\sphinxbfcode{\sphinxupquote{SubrackInitialConfiguration}}}{}{}
\pysigstopsignatures
\sphinxAtStartPar
@brief method Initizlize the Subrack power control configuration for TPM current limit

\end{fulllineitems}

\index{all\_monitoring\_categories() (subrack\_management\_board.SubrackMngBoard method)@\spxentry{all\_monitoring\_categories()}\spxextra{subrack\_management\_board.SubrackMngBoard method}}

\begin{fulllineitems}
\phantomsection\label{\detokenize{apidocs:subrack_management_board.SubrackMngBoard.all_monitoring_categories}}
\pysigstartsignatures
\pysiglinewithargsret{\sphinxbfcode{\sphinxupquote{all\_monitoring\_categories}}}{}{}
\pysigstopsignatures
\sphinxAtStartPar
Returns a list of all monitoring point ‘categories’.
Here categories is a super\sphinxhyphen{}set of monitoring points and is
the full list of accepted strings to set\_monitoring\_point\_attr.
\begin{quote}\begin{description}
\sphinxlineitem{Returns}
\sphinxAtStartPar
list of categories

\sphinxlineitem{Return type}
\sphinxAtStartPar
list of strings

\end{description}\end{quote}

\end{fulllineitems}

\index{all\_monitoring\_points() (subrack\_management\_board.SubrackMngBoard method)@\spxentry{all\_monitoring\_points()}\spxextra{subrack\_management\_board.SubrackMngBoard method}}

\begin{fulllineitems}
\phantomsection\label{\detokenize{apidocs:subrack_management_board.SubrackMngBoard.all_monitoring_points}}
\pysigstartsignatures
\pysiglinewithargsret{\sphinxbfcode{\sphinxupquote{all\_monitoring\_points}}}{}{}
\pysigstopsignatures
\sphinxAtStartPar
Returns a list of all monitoring points by finding all leaf nodes
in the lookup dict that have a corresponding method field.

\sphinxAtStartPar
The monitoring points returned are strings produced from ‘.’ delimited
keys.
\begin{quote}\begin{description}
\sphinxlineitem{Returns}
\sphinxAtStartPar
list of monitoring points

\sphinxlineitem{Return type}
\sphinxAtStartPar
list of strings

\end{description}\end{quote}

\end{fulllineitems}

\index{bkpln\_get\_field() (subrack\_management\_board.SubrackMngBoard method)@\spxentry{bkpln\_get\_field()}\spxextra{subrack\_management\_board.SubrackMngBoard method}}

\begin{fulllineitems}
\phantomsection\label{\detokenize{apidocs:subrack_management_board.SubrackMngBoard.bkpln_get_field}}
\pysigstartsignatures
\pysiglinewithargsret{\sphinxbfcode{\sphinxupquote{bkpln\_get\_field}}}{\sphinxparam{\DUrole{n}{key}}}{}
\pysigstopsignatures
\sphinxAtStartPar
Retrieves the value of a field from the backup location (Bkpln).

\sphinxAtStartPar
This method delegates the task to the Bkpln instance to get the value of
the specified field.
\begin{quote}\begin{description}
\sphinxlineitem{Parameters}
\sphinxAtStartPar
\sphinxstyleliteralstrong{\sphinxupquote{key}} \textendash{} The key identifying the field to be retrieved.

\sphinxlineitem{Returns}
\sphinxAtStartPar
The value of the specified field.

\end{description}\end{quote}

\end{fulllineitems}

\index{bkpln\_set\_field() (subrack\_management\_board.SubrackMngBoard method)@\spxentry{bkpln\_set\_field()}\spxextra{subrack\_management\_board.SubrackMngBoard method}}

\begin{fulllineitems}
\phantomsection\label{\detokenize{apidocs:subrack_management_board.SubrackMngBoard.bkpln_set_field}}
\pysigstartsignatures
\pysiglinewithargsret{\sphinxbfcode{\sphinxupquote{bkpln\_set\_field}}}{\sphinxparam{\DUrole{n}{key}}\sphinxparamcomma \sphinxparam{\DUrole{n}{value}}\sphinxparamcomma \sphinxparam{\DUrole{n}{override\_protected}\DUrole{o}{=}\DUrole{default_value}{False}}}{}
\pysigstopsignatures
\sphinxAtStartPar
Sets a field in the backup location (Bkpln).

\sphinxAtStartPar
This method delegates the task to the Bkpln instance to set the specified
field with the given value. Optionally, it allows overriding protected fields.
\begin{quote}\begin{description}
\sphinxlineitem{Parameters}\begin{itemize}
\item {} 
\sphinxAtStartPar
\sphinxstyleliteralstrong{\sphinxupquote{key}} \textendash{} The key identifying the field to be set.

\item {} 
\sphinxAtStartPar
\sphinxstyleliteralstrong{\sphinxupquote{value}} \textendash{} The value to be set for the specified field.

\item {} 
\sphinxAtStartPar
\sphinxstyleliteralstrong{\sphinxupquote{override\_protected}} \textendash{} (Optional) If True, allows overriding protected fields.

\end{itemize}

\sphinxlineitem{Returns}
\sphinxAtStartPar
The result of setting the field.

\end{description}\end{quote}

\end{fulllineitems}

\index{get\_board\_info() (subrack\_management\_board.SubrackMngBoard method)@\spxentry{get\_board\_info()}\spxextra{subrack\_management\_board.SubrackMngBoard method}}

\begin{fulllineitems}
\phantomsection\label{\detokenize{apidocs:subrack_management_board.SubrackMngBoard.get_board_info}}
\pysigstartsignatures
\pysiglinewithargsret{\sphinxbfcode{\sphinxupquote{get\_board\_info}}}{}{}
\pysigstopsignatures
\sphinxAtStartPar
Get information about the Subrack board.

\sphinxAtStartPar
Returns:
dict: Board information.

\end{fulllineitems}

\index{get\_health\_dict() (subrack\_management\_board.SubrackMngBoard method)@\spxentry{get\_health\_dict()}\spxextra{subrack\_management\_board.SubrackMngBoard method}}

\begin{fulllineitems}
\phantomsection\label{\detokenize{apidocs:subrack_management_board.SubrackMngBoard.get_health_dict}}
\pysigstartsignatures
\pysiglinewithargsret{\sphinxbfcode{\sphinxupquote{get\_health\_dict}}}{\sphinxparam{\DUrole{o}{**}\DUrole{n}{kwargs}}}{}
\pysigstopsignatures
\sphinxAtStartPar
Returns the dictionary of SUBRACK monitoring points with the
static key only, no value

\end{fulllineitems}

\index{get\_health\_status() (subrack\_management\_board.SubrackMngBoard method)@\spxentry{get\_health\_status()}\spxextra{subrack\_management\_board.SubrackMngBoard method}}

\begin{fulllineitems}
\phantomsection\label{\detokenize{apidocs:subrack_management_board.SubrackMngBoard.get_health_status}}
\pysigstartsignatures
\pysiglinewithargsret{\sphinxbfcode{\sphinxupquote{get\_health\_status}}}{\sphinxparam{\DUrole{o}{**}\DUrole{n}{kwargs}}}{}
\pysigstopsignatures
\sphinxAtStartPar
Returns the current value of SUBRACK monitoring points
If no group argument given, current value of all monitoring points is returned.

\sphinxAtStartPar
For example:
subrack.get\_health\_status(group=’temperatures’)
would return only the health status for:

\sphinxAtStartPar
A group attribute is provided by default, see subrack\_monitoring\_point\_lookup.py.
This can be used like the below example:
subrack.get\_health\_status(group=’temperatures’)
subrack.get\_health\_status(group=’slots’)
subrack.get\_health\_status(group=’voltages’)

\end{fulllineitems}

\index{get\_health\_status\_w\_elapsed() (subrack\_management\_board.SubrackMngBoard method)@\spxentry{get\_health\_status\_w\_elapsed()}\spxextra{subrack\_management\_board.SubrackMngBoard method}}

\begin{fulllineitems}
\phantomsection\label{\detokenize{apidocs:subrack_management_board.SubrackMngBoard.get_health_status_w_elapsed}}
\pysigstartsignatures
\pysiglinewithargsret{\sphinxbfcode{\sphinxupquote{get\_health\_status\_w\_elapsed}}}{\sphinxparam{\DUrole{o}{**}\DUrole{n}{kwargs}}}{}
\pysigstopsignatures
\sphinxAtStartPar
Returns the current value of SUBRACK monitoring points
If no group argument given, current value of all monitoring points is returned.

\sphinxAtStartPar
For example:
subrack.get\_health\_status(group=’temperatures’)
would return only the health status for:

\sphinxAtStartPar
A group attribute is provided by default, see subrack\_monitoring\_point\_lookup.py.
This can be used like the below example:
subrack.get\_health\_status(group=’temperatures’)
subrack.get\_health\_status(group=’slots’)
subrack.get\_health\_status(group=’voltages’)

\end{fulllineitems}

\index{get\_subrack\_cpu\_cpld\_ip() (subrack\_management\_board.SubrackMngBoard method)@\spxentry{get\_subrack\_cpu\_cpld\_ip()}\spxextra{subrack\_management\_board.SubrackMngBoard method}}

\begin{fulllineitems}
\phantomsection\label{\detokenize{apidocs:subrack_management_board.SubrackMngBoard.get_subrack_cpu_cpld_ip}}
\pysigstartsignatures
\pysiglinewithargsret{\sphinxbfcode{\sphinxupquote{get\_subrack\_cpu\_cpld\_ip}}}{}{}
\pysigstopsignatures
\sphinxAtStartPar
SubrackInitialConfiguration
@brief method Initizlize the Subrack power control configuration for TPM current limit
:return cpu\_ip, cpld\_ip: Management CPU IP, Management CPLD IP

\end{fulllineitems}

\index{read\_tpm\_singlewire() (subrack\_management\_board.SubrackMngBoard method)@\spxentry{read\_tpm\_singlewire()}\spxextra{subrack\_management\_board.SubrackMngBoard method}}

\begin{fulllineitems}
\phantomsection\label{\detokenize{apidocs:subrack_management_board.SubrackMngBoard.read_tpm_singlewire}}
\pysigstartsignatures
\pysiglinewithargsret{\sphinxbfcode{\sphinxupquote{read\_tpm\_singlewire}}}{\sphinxparam{\DUrole{n}{tpm\_id}}\sphinxparamcomma \sphinxparam{\DUrole{n}{address}}}{}
\pysigstopsignatures
\sphinxAtStartPar
Reads a value from the single\sphinxhyphen{}wire interface of a TPM.

\sphinxAtStartPar
This method selects a TPM by psnt\_mux and reads a register value from the
specified address in the single\sphinxhyphen{}wire interface.
\begin{quote}\begin{description}
\sphinxlineitem{Parameters}\begin{itemize}
\item {} 
\sphinxAtStartPar
\sphinxstyleliteralstrong{\sphinxupquote{tpm\_id}} \textendash{} The TPM identifier.

\item {} 
\sphinxAtStartPar
\sphinxstyleliteralstrong{\sphinxupquote{address}} \textendash{} The register address in the single\sphinxhyphen{}wire interface.

\end{itemize}

\sphinxlineitem{Returns}
\sphinxAtStartPar
The value read from the specified register.

\end{description}\end{quote}

\end{fulllineitems}

\index{write\_tpm\_singlewire() (subrack\_management\_board.SubrackMngBoard method)@\spxentry{write\_tpm\_singlewire()}\spxextra{subrack\_management\_board.SubrackMngBoard method}}

\begin{fulllineitems}
\phantomsection\label{\detokenize{apidocs:subrack_management_board.SubrackMngBoard.write_tpm_singlewire}}
\pysigstartsignatures
\pysiglinewithargsret{\sphinxbfcode{\sphinxupquote{write\_tpm\_singlewire}}}{\sphinxparam{\DUrole{n}{tpm\_id}}\sphinxparamcomma \sphinxparam{\DUrole{n}{address}}\sphinxparamcomma \sphinxparam{\DUrole{n}{value}}}{}
\pysigstopsignatures
\sphinxAtStartPar
Writes a value to the single\sphinxhyphen{}wire interface of a TPM.

\sphinxAtStartPar
This method selects a TPM by psnt\_mux and writes a value to the specified
address in the single\sphinxhyphen{}wire interface.
\begin{quote}\begin{description}
\sphinxlineitem{Parameters}\begin{itemize}
\item {} 
\sphinxAtStartPar
\sphinxstyleliteralstrong{\sphinxupquote{tpm\_id}} \textendash{} The TPM identifier.

\item {} 
\sphinxAtStartPar
\sphinxstyleliteralstrong{\sphinxupquote{address}} \textendash{} The register address in the single\sphinxhyphen{}wire interface.

\item {} 
\sphinxAtStartPar
\sphinxstyleliteralstrong{\sphinxupquote{value}} \textendash{} The value to be written to the register.

\end{itemize}

\end{description}\end{quote}

\end{fulllineitems}


\end{fulllineitems}

\index{detect\_ip() (in module subrack\_management\_board)@\spxentry{detect\_ip()}\spxextra{in module subrack\_management\_board}}

\begin{fulllineitems}
\phantomsection\label{\detokenize{apidocs:subrack_management_board.detect_ip}}
\pysigstartsignatures
\pysiglinewithargsret{\sphinxcode{\sphinxupquote{subrack\_management\_board.}}\sphinxbfcode{\sphinxupquote{detect\_ip}}}{\sphinxparam{\DUrole{n}{tpm\_slot\_id}}}{}
\pysigstopsignatures
\sphinxAtStartPar
Detect the IP address of a TPM board based on its slot ID.

\end{fulllineitems}

\index{dt\_to\_timestamp() (in module subrack\_management\_board)@\spxentry{dt\_to\_timestamp()}\spxextra{in module subrack\_management\_board}}

\begin{fulllineitems}
\phantomsection\label{\detokenize{apidocs:subrack_management_board.dt_to_timestamp}}
\pysigstartsignatures
\pysiglinewithargsret{\sphinxcode{\sphinxupquote{subrack\_management\_board.}}\sphinxbfcode{\sphinxupquote{dt\_to\_timestamp}}}{\sphinxparam{\DUrole{n}{d}}}{}
\pysigstopsignatures
\sphinxAtStartPar
Convert a datetime object to a Unix timestamp.

\end{fulllineitems}

\index{exec\_cmd() (in module subrack\_management\_board)@\spxentry{exec\_cmd()}\spxextra{in module subrack\_management\_board}}

\begin{fulllineitems}
\phantomsection\label{\detokenize{apidocs:subrack_management_board.exec_cmd}}
\pysigstartsignatures
\pysiglinewithargsret{\sphinxcode{\sphinxupquote{subrack\_management\_board.}}\sphinxbfcode{\sphinxupquote{exec\_cmd}}}{\sphinxparam{\DUrole{n}{cmd}}\sphinxparamcomma \sphinxparam{\DUrole{n}{dir}\DUrole{o}{=}\DUrole{default_value}{None}}\sphinxparamcomma \sphinxparam{\DUrole{n}{verbose}\DUrole{o}{=}\DUrole{default_value}{True}}\sphinxparamcomma \sphinxparam{\DUrole{n}{exclude\_line}\DUrole{o}{=}\DUrole{default_value}{\textquotesingle{}\textquotesingle{}}}}{}
\pysigstopsignatures
\sphinxAtStartPar
Execute a shell command and capture its output.

\end{fulllineitems}

\index{flatten\_dict() (in module subrack\_management\_board)@\spxentry{flatten\_dict()}\spxextra{in module subrack\_management\_board}}

\begin{fulllineitems}
\phantomsection\label{\detokenize{apidocs:subrack_management_board.flatten_dict}}
\pysigstartsignatures
\pysiglinewithargsret{\sphinxcode{\sphinxupquote{subrack\_management\_board.}}\sphinxbfcode{\sphinxupquote{flatten\_dict}}}{\sphinxparam{\DUrole{n}{d}}\sphinxparamcomma \sphinxparam{\DUrole{n}{parent\_key}\DUrole{o}{=}\DUrole{default_value}{\textquotesingle{}\textquotesingle{}}}\sphinxparamcomma \sphinxparam{\DUrole{n}{sep}\DUrole{o}{=}\DUrole{default_value}{\textquotesingle{}\_\textquotesingle{}}}}{}
\pysigstopsignatures
\sphinxAtStartPar
Flatten a nested dictionary.

\end{fulllineitems}

\index{int2ip() (in module subrack\_management\_board)@\spxentry{int2ip()}\spxextra{in module subrack\_management\_board}}

\begin{fulllineitems}
\phantomsection\label{\detokenize{apidocs:subrack_management_board.int2ip}}
\pysigstartsignatures
\pysiglinewithargsret{\sphinxcode{\sphinxupquote{subrack\_management\_board.}}\sphinxbfcode{\sphinxupquote{int2ip}}}{\sphinxparam{\DUrole{n}{value}}}{}
\pysigstopsignatures
\sphinxAtStartPar
Convert an integer value to an IP address.

\end{fulllineitems}

\index{ipstr2hex() (in module subrack\_management\_board)@\spxentry{ipstr2hex()}\spxextra{in module subrack\_management\_board}}

\begin{fulllineitems}
\phantomsection\label{\detokenize{apidocs:subrack_management_board.ipstr2hex}}
\pysigstartsignatures
\pysiglinewithargsret{\sphinxcode{\sphinxupquote{subrack\_management\_board.}}\sphinxbfcode{\sphinxupquote{ipstr2hex}}}{\sphinxparam{\DUrole{n}{ip}}}{}
\pysigstopsignatures
\sphinxAtStartPar
Convert an IP address string to a hexadecimal representation.

\end{fulllineitems}

\index{reduce() (in module subrack\_management\_board)@\spxentry{reduce()}\spxextra{in module subrack\_management\_board}}

\begin{fulllineitems}
\phantomsection\label{\detokenize{apidocs:subrack_management_board.reduce}}
\pysigstartsignatures
\pysiglinewithargsret{\sphinxcode{\sphinxupquote{subrack\_management\_board.}}\sphinxbfcode{\sphinxupquote{reduce}}}{\sphinxparam{\DUrole{n}{function}}\sphinxparamcomma \sphinxparam{\DUrole{n}{iterable}}\sphinxoptional{\sphinxparamcomma \sphinxparam{\DUrole{n}{initial}}}}{{ $\rightarrow$ value}}
\pysigstopsignatures
\sphinxAtStartPar
Apply a function of two arguments cumulatively to the items of a sequence
or iterable, from left to right, so as to reduce the iterable to a single
value.  For example, reduce(lambda x, y: x+y, {[}1, 2, 3, 4, 5{]}) calculates
((((1+2)+3)+4)+5).  If initial is present, it is placed before the items
of the iterable in the calculation, and serves as a default when the
iterable is empty.

\end{fulllineitems}

\index{module@\spxentry{module}!subrack\_monitoring\_point\_lookup@\spxentry{subrack\_monitoring\_point\_lookup}}\index{subrack\_monitoring\_point\_lookup@\spxentry{subrack\_monitoring\_point\_lookup}!module@\spxentry{module}}\index{partial (class in subrack\_monitoring\_point\_lookup)@\spxentry{partial}\spxextra{class in subrack\_monitoring\_point\_lookup}}\phantomsection\label{\detokenize{apidocs:module-subrack_monitoring_point_lookup}}

\begin{fulllineitems}
\phantomsection\label{\detokenize{apidocs:subrack_monitoring_point_lookup.partial}}
\pysigstartsignatures
\pysiglinewithargsret{\sphinxbfcode{\sphinxupquote{class\DUrole{w}{ }}}\sphinxcode{\sphinxupquote{subrack\_monitoring\_point\_lookup.}}\sphinxbfcode{\sphinxupquote{partial}}}{\sphinxparam{\DUrole{n}{func}}\sphinxparamcomma \sphinxparam{\DUrole{o}{/}}\sphinxparamcomma \sphinxparam{\DUrole{o}{*}\DUrole{n}{args}}\sphinxparamcomma \sphinxparam{\DUrole{o}{**}\DUrole{n}{keywords}}}{}
\pysigstopsignatures
\sphinxAtStartPar
New function with partial application of the given arguments
and keywords.

\end{fulllineitems}


\sphinxstepscope


\chapter{Web Server Documentation}
\label{\detokenize{webserverdocs:module-HardwareBaseClass}}\label{\detokenize{webserverdocs:web-server-documentation}}\label{\detokenize{webserverdocs::doc}}\index{module@\spxentry{module}!HardwareBaseClass@\spxentry{HardwareBaseClass}}\index{HardwareBaseClass@\spxentry{HardwareBaseClass}!module@\spxentry{module}}
\sphinxAtStartPar
Package to implement simple device drivers.
Each hardware driver is an instance of a HardwareBaseClass
It contains a set of HardwareCommand objects, which implement
actions (with optional parameters) and
Hardware Attribute objects, which implement physical quantities.
The HardwareBaseClass is controlled using commands, or by setting
attributes (if they are RW). It is monitored by reading attributes.
Hardware objects, commands and attributes can be subclassed for specific
behaviors.
\index{GetAllAttributesCommand (class in HardwareBaseClass)@\spxentry{GetAllAttributesCommand}\spxextra{class in HardwareBaseClass}}

\begin{fulllineitems}
\phantomsection\label{\detokenize{webserverdocs:HardwareBaseClass.GetAllAttributesCommand}}
\pysigstartsignatures
\pysiglinewithargsret{\sphinxbfcode{\sphinxupquote{class\DUrole{w}{ }}}\sphinxcode{\sphinxupquote{HardwareBaseClass.}}\sphinxbfcode{\sphinxupquote{GetAllAttributesCommand}}}{\sphinxparam{\DUrole{n}{name}}\sphinxparamcomma \sphinxparam{\DUrole{n}{hardware}\DUrole{o}{=}\DUrole{default_value}{None}}\sphinxparamcomma \sphinxparam{\DUrole{n}{num\_params}\DUrole{o}{=}\DUrole{default_value}{0}}}{}
\pysigstopsignatures
\sphinxAtStartPar
command which returns a dictionary of all attributes and their
current values inthe ‘value’ field
\index{do() (HardwareBaseClass.GetAllAttributesCommand method)@\spxentry{do()}\spxextra{HardwareBaseClass.GetAllAttributesCommand method}}

\begin{fulllineitems}
\phantomsection\label{\detokenize{webserverdocs:HardwareBaseClass.GetAllAttributesCommand.do}}
\pysigstartsignatures
\pysiglinewithargsret{\sphinxbfcode{\sphinxupquote{do}}}{\sphinxparam{\DUrole{n}{params}\DUrole{o}{=}\DUrole{default_value}{None}}}{}
\pysigstopsignatures\begin{quote}\begin{description}
\sphinxlineitem{Parameters}
\sphinxAtStartPar
\sphinxstyleliteralstrong{\sphinxupquote{params}} (\sphinxstyleliteralemphasis{\sphinxupquote{list}}) \textendash{} Optional list of parameters

\sphinxlineitem{Returns}
\sphinxAtStartPar
Dictionary of returned response

\sphinxlineitem{Return type}
\sphinxAtStartPar
dict

\end{description}\end{quote}

\end{fulllineitems}


\end{fulllineitems}

\index{HardwareAttribute (class in HardwareBaseClass)@\spxentry{HardwareAttribute}\spxextra{class in HardwareBaseClass}}

\begin{fulllineitems}
\phantomsection\label{\detokenize{webserverdocs:HardwareBaseClass.HardwareAttribute}}
\pysigstartsignatures
\pysiglinewithargsret{\sphinxbfcode{\sphinxupquote{class\DUrole{w}{ }}}\sphinxcode{\sphinxupquote{HardwareBaseClass.}}\sphinxbfcode{\sphinxupquote{HardwareAttribute}}}{\sphinxparam{\DUrole{n}{name}}\sphinxparamcomma \sphinxparam{\DUrole{n}{init\_value}}\sphinxparamcomma \sphinxparam{\DUrole{n}{hardware}\DUrole{o}{=}\DUrole{default_value}{None}}\sphinxparamcomma \sphinxparam{\DUrole{n}{read\_write}\DUrole{o}{=}\DUrole{default_value}{0}}\sphinxparamcomma \sphinxparam{\DUrole{n}{num\_params}\DUrole{o}{=}\DUrole{default_value}{1}}}{}
\pysigstopsignatures
\sphinxAtStartPar
Attribute for a HardwareBaseClass
Attributes can be scalar or vector of fixed dimension, of arbitrary types
Can be read/write or read only
\index{name() (HardwareBaseClass.HardwareAttribute method)@\spxentry{name()}\spxextra{HardwareBaseClass.HardwareAttribute method}}

\begin{fulllineitems}
\phantomsection\label{\detokenize{webserverdocs:HardwareBaseClass.HardwareAttribute.name}}
\pysigstartsignatures
\pysiglinewithargsret{\sphinxbfcode{\sphinxupquote{name}}}{}{}
\pysigstopsignatures
\sphinxAtStartPar
Returns the command name
\begin{quote}\begin{description}
\sphinxlineitem{Returns}
\sphinxAtStartPar
Command name

\sphinxlineitem{Return type}
\sphinxAtStartPar
str

\end{description}\end{quote}

\end{fulllineitems}

\index{read() (HardwareBaseClass.HardwareAttribute method)@\spxentry{read()}\spxextra{HardwareBaseClass.HardwareAttribute method}}

\begin{fulllineitems}
\phantomsection\label{\detokenize{webserverdocs:HardwareBaseClass.HardwareAttribute.read}}
\pysigstartsignatures
\pysiglinewithargsret{\sphinxbfcode{\sphinxupquote{read}}}{}{}
\pysigstopsignatures
\sphinxAtStartPar
Read the attribute, formatting the response dictionary
\begin{quote}\begin{description}
\sphinxlineitem{Returns}
\sphinxAtStartPar
Dictionary of returned response

\sphinxlineitem{Return type}
\sphinxAtStartPar
dict

\end{description}\end{quote}

\end{fulllineitems}

\index{read\_value() (HardwareBaseClass.HardwareAttribute method)@\spxentry{read\_value()}\spxextra{HardwareBaseClass.HardwareAttribute method}}

\begin{fulllineitems}
\phantomsection\label{\detokenize{webserverdocs:HardwareBaseClass.HardwareAttribute.read_value}}
\pysigstartsignatures
\pysiglinewithargsret{\sphinxbfcode{\sphinxupquote{read\_value}}}{}{}
\pysigstopsignatures
\sphinxAtStartPar
Reads the actual value
To be overrided in the subclass
\begin{quote}\begin{description}
\sphinxlineitem{Returns}
\sphinxAtStartPar
Attribute true value, from real device

\end{description}\end{quote}

\end{fulllineitems}

\index{write() (HardwareBaseClass.HardwareAttribute method)@\spxentry{write()}\spxextra{HardwareBaseClass.HardwareAttribute method}}

\begin{fulllineitems}
\phantomsection\label{\detokenize{webserverdocs:HardwareBaseClass.HardwareAttribute.write}}
\pysigstartsignatures
\pysiglinewithargsret{\sphinxbfcode{\sphinxupquote{write}}}{\sphinxparam{\DUrole{n}{params}}}{}
\pysigstopsignatures
\sphinxAtStartPar
write value(s) in params to the attribute
\begin{quote}\begin{description}
\sphinxlineitem{Parameters}
\sphinxAtStartPar
\sphinxstyleliteralstrong{\sphinxupquote{params}} \textendash{} Parameters to be written

\sphinxlineitem{Returns}
\sphinxAtStartPar
dictionary for json answer

\sphinxlineitem{Return type}
\sphinxAtStartPar
dict

\end{description}\end{quote}

\end{fulllineitems}

\index{write\_value() (HardwareBaseClass.HardwareAttribute method)@\spxentry{write\_value()}\spxextra{HardwareBaseClass.HardwareAttribute method}}

\begin{fulllineitems}
\phantomsection\label{\detokenize{webserverdocs:HardwareBaseClass.HardwareAttribute.write_value}}
\pysigstartsignatures
\pysiglinewithargsret{\sphinxbfcode{\sphinxupquote{write\_value}}}{\sphinxparam{\DUrole{n}{values}}}{}
\pysigstopsignatures
\sphinxAtStartPar
Writes the actual value
To be overrided in the subclass
\begin{quote}\begin{description}
\sphinxlineitem{Parameters}
\sphinxAtStartPar
\sphinxstyleliteralstrong{\sphinxupquote{value}} \textendash{} Attribute true value, to real device

\end{description}\end{quote}

\end{fulllineitems}


\end{fulllineitems}

\index{HardwareBaseDevice (class in HardwareBaseClass)@\spxentry{HardwareBaseDevice}\spxextra{class in HardwareBaseClass}}

\begin{fulllineitems}
\phantomsection\label{\detokenize{webserverdocs:HardwareBaseClass.HardwareBaseDevice}}
\pysigstartsignatures
\pysigline{\sphinxbfcode{\sphinxupquote{class\DUrole{w}{ }}}\sphinxcode{\sphinxupquote{HardwareBaseClass.}}\sphinxbfcode{\sphinxupquote{HardwareBaseDevice}}}
\pysigstopsignatures
\sphinxAtStartPar
Server side of the hardware device.
\index{add\_attribute() (HardwareBaseClass.HardwareBaseDevice method)@\spxentry{add\_attribute()}\spxextra{HardwareBaseClass.HardwareBaseDevice method}}

\begin{fulllineitems}
\phantomsection\label{\detokenize{webserverdocs:HardwareBaseClass.HardwareBaseDevice.add_attribute}}
\pysigstartsignatures
\pysiglinewithargsret{\sphinxbfcode{\sphinxupquote{add\_attribute}}}{\sphinxparam{\DUrole{n}{attribute}}}{}
\pysigstopsignatures
\sphinxAtStartPar
Add an attribute to the command list
\begin{quote}\begin{description}
\sphinxlineitem{Parameters}
\sphinxAtStartPar
\sphinxstyleliteralstrong{\sphinxupquote{attribute}} (\sphinxcode{\sphinxupquote{HardwareBase.HardwareAttribute}}) \textendash{} Attribute object

\sphinxlineitem{Returns}
\sphinxAtStartPar
True if command canbe added, False otherwise

\sphinxlineitem{Return type}
\sphinxAtStartPar
Bool

\end{description}\end{quote}

\end{fulllineitems}

\index{add\_command() (HardwareBaseClass.HardwareBaseDevice method)@\spxentry{add\_command()}\spxextra{HardwareBaseClass.HardwareBaseDevice method}}

\begin{fulllineitems}
\phantomsection\label{\detokenize{webserverdocs:HardwareBaseClass.HardwareBaseDevice.add_command}}
\pysigstartsignatures
\pysiglinewithargsret{\sphinxbfcode{\sphinxupquote{add\_command}}}{\sphinxparam{\DUrole{n}{command}}}{}
\pysigstopsignatures
\sphinxAtStartPar
Add a command to the command list
\begin{quote}\begin{description}
\sphinxlineitem{Parameters}
\sphinxAtStartPar
\sphinxstyleliteralstrong{\sphinxupquote{command}} \textendash{} Command object

\sphinxlineitem{Returns}
\sphinxAtStartPar
True if command canbe added, False otherwise

\sphinxlineitem{Return type}
\sphinxAtStartPar
Bool

\end{description}\end{quote}

\end{fulllineitems}

\index{execute\_command() (HardwareBaseClass.HardwareBaseDevice method)@\spxentry{execute\_command()}\spxextra{HardwareBaseClass.HardwareBaseDevice method}}

\begin{fulllineitems}
\phantomsection\label{\detokenize{webserverdocs:HardwareBaseClass.HardwareBaseDevice.execute_command}}
\pysigstartsignatures
\pysiglinewithargsret{\sphinxbfcode{\sphinxupquote{execute\_command}}}{\sphinxparam{\DUrole{n}{command}}\sphinxparamcomma \sphinxparam{\DUrole{n}{params}\DUrole{o}{=}\DUrole{default_value}{None}}}{}
\pysigstopsignatures
\sphinxAtStartPar
Execute a command with optional parameters
\begin{quote}\begin{description}
\sphinxlineitem{Parameters}\begin{itemize}
\item {} 
\sphinxAtStartPar
\sphinxstyleliteralstrong{\sphinxupquote{command}} (\sphinxstyleliteralemphasis{\sphinxupquote{str}}) \textendash{} Command name

\item {} 
\sphinxAtStartPar
\sphinxstyleliteralstrong{\sphinxupquote{params}} \textendash{} Optional parameter, simple list or scalar

\end{itemize}

\sphinxlineitem{Returns}
\sphinxAtStartPar
dictionary for json answer

\end{description}\end{quote}

\end{fulllineitems}

\index{get\_attribute() (HardwareBaseClass.HardwareBaseDevice method)@\spxentry{get\_attribute()}\spxextra{HardwareBaseClass.HardwareBaseDevice method}}

\begin{fulllineitems}
\phantomsection\label{\detokenize{webserverdocs:HardwareBaseClass.HardwareBaseDevice.get_attribute}}
\pysigstartsignatures
\pysiglinewithargsret{\sphinxbfcode{\sphinxupquote{get\_attribute}}}{\sphinxparam{\DUrole{n}{attribute}}}{}
\pysigstopsignatures
\sphinxAtStartPar
Get attribute values
\begin{quote}\begin{description}
\sphinxlineitem{Parameters}
\sphinxAtStartPar
\sphinxstyleliteralstrong{\sphinxupquote{attribute}} \textendash{} Attribute name

\sphinxlineitem{Returns}
\sphinxAtStartPar
dictionary for json answer

\end{description}\end{quote}

\end{fulllineitems}

\index{set\_attribute() (HardwareBaseClass.HardwareBaseDevice method)@\spxentry{set\_attribute()}\spxextra{HardwareBaseClass.HardwareBaseDevice method}}

\begin{fulllineitems}
\phantomsection\label{\detokenize{webserverdocs:HardwareBaseClass.HardwareBaseDevice.set_attribute}}
\pysigstartsignatures
\pysiglinewithargsret{\sphinxbfcode{\sphinxupquote{set\_attribute}}}{\sphinxparam{\DUrole{n}{attribute}}\sphinxparamcomma \sphinxparam{\DUrole{n}{values}}}{}
\pysigstopsignatures
\sphinxAtStartPar
Set attribute values
\begin{quote}\begin{description}
\sphinxlineitem{Parameters}\begin{itemize}
\item {} 
\sphinxAtStartPar
\sphinxstyleliteralstrong{\sphinxupquote{attribute}} \textendash{} Attribute name

\item {} 
\sphinxAtStartPar
\sphinxstyleliteralstrong{\sphinxupquote{values}} \textendash{} Attribute values, simple list or scalar

\end{itemize}

\sphinxlineitem{Returns}
\sphinxAtStartPar
dictionary for json answer

\end{description}\end{quote}

\end{fulllineitems}


\end{fulllineitems}

\index{HardwareCommand (class in HardwareBaseClass)@\spxentry{HardwareCommand}\spxextra{class in HardwareBaseClass}}

\begin{fulllineitems}
\phantomsection\label{\detokenize{webserverdocs:HardwareBaseClass.HardwareCommand}}
\pysigstartsignatures
\pysiglinewithargsret{\sphinxbfcode{\sphinxupquote{class\DUrole{w}{ }}}\sphinxcode{\sphinxupquote{HardwareBaseClass.}}\sphinxbfcode{\sphinxupquote{HardwareCommand}}}{\sphinxparam{\DUrole{n}{name}}\sphinxparamcomma \sphinxparam{\DUrole{n}{hardware}\DUrole{o}{=}\DUrole{default_value}{None}}\sphinxparamcomma \sphinxparam{\DUrole{n}{num\_params}\DUrole{o}{=}\DUrole{default_value}{0}}}{}
\pysigstopsignatures
\sphinxAtStartPar
Command for a Hardware base class. Command has a name, has access to
its base hardware (if useful), and can check the number of parameters
\index{do() (HardwareBaseClass.HardwareCommand method)@\spxentry{do()}\spxextra{HardwareBaseClass.HardwareCommand method}}

\begin{fulllineitems}
\phantomsection\label{\detokenize{webserverdocs:HardwareBaseClass.HardwareCommand.do}}
\pysigstartsignatures
\pysiglinewithargsret{\sphinxbfcode{\sphinxupquote{do}}}{\sphinxparam{\DUrole{n}{params}\DUrole{o}{=}\DUrole{default_value}{None}}}{}
\pysigstopsignatures
\sphinxAtStartPar
Execution method. Subclasses must override this to do real work
\begin{quote}\begin{description}
\sphinxlineitem{Parameters}
\sphinxAtStartPar
\sphinxstyleliteralstrong{\sphinxupquote{params}} (\sphinxstyleliteralemphasis{\sphinxupquote{list}}) \textendash{} Optional list of parameters

\sphinxlineitem{Returns}
\sphinxAtStartPar
Dictionary of returned response

\sphinxlineitem{Return type}
\sphinxAtStartPar
dict

\end{description}\end{quote}

\end{fulllineitems}

\index{name() (HardwareBaseClass.HardwareCommand method)@\spxentry{name()}\spxextra{HardwareBaseClass.HardwareCommand method}}

\begin{fulllineitems}
\phantomsection\label{\detokenize{webserverdocs:HardwareBaseClass.HardwareCommand.name}}
\pysigstartsignatures
\pysiglinewithargsret{\sphinxbfcode{\sphinxupquote{name}}}{}{}
\pysigstopsignatures
\sphinxAtStartPar
Returns the command name
\begin{quote}\begin{description}
\sphinxlineitem{Returns}
\sphinxAtStartPar
Command name

\sphinxlineitem{Return type}
\sphinxAtStartPar
str

\end{description}\end{quote}

\end{fulllineitems}


\end{fulllineitems}

\index{ListAttributeCommand (class in HardwareBaseClass)@\spxentry{ListAttributeCommand}\spxextra{class in HardwareBaseClass}}

\begin{fulllineitems}
\phantomsection\label{\detokenize{webserverdocs:HardwareBaseClass.ListAttributeCommand}}
\pysigstartsignatures
\pysiglinewithargsret{\sphinxbfcode{\sphinxupquote{class\DUrole{w}{ }}}\sphinxcode{\sphinxupquote{HardwareBaseClass.}}\sphinxbfcode{\sphinxupquote{ListAttributeCommand}}}{\sphinxparam{\DUrole{n}{name}}\sphinxparamcomma \sphinxparam{\DUrole{n}{hardware}\DUrole{o}{=}\DUrole{default_value}{None}}\sphinxparamcomma \sphinxparam{\DUrole{n}{num\_params}\DUrole{o}{=}\DUrole{default_value}{0}}}{}
\pysigstopsignatures
\sphinxAtStartPar
Command to list names of the Hardware class commands
\index{do() (HardwareBaseClass.ListAttributeCommand method)@\spxentry{do()}\spxextra{HardwareBaseClass.ListAttributeCommand method}}

\begin{fulllineitems}
\phantomsection\label{\detokenize{webserverdocs:HardwareBaseClass.ListAttributeCommand.do}}
\pysigstartsignatures
\pysiglinewithargsret{\sphinxbfcode{\sphinxupquote{do}}}{\sphinxparam{\DUrole{n}{params}\DUrole{o}{=}\DUrole{default_value}{None}}}{}
\pysigstopsignatures\begin{quote}\begin{description}
\sphinxlineitem{Parameters}
\sphinxAtStartPar
\sphinxstyleliteralstrong{\sphinxupquote{params}} (\sphinxstyleliteralemphasis{\sphinxupquote{list}}) \textendash{} Optional list of parameters

\sphinxlineitem{Returns}
\sphinxAtStartPar
Dictionary of returned response

\sphinxlineitem{Return type}
\sphinxAtStartPar
dict

\end{description}\end{quote}

\end{fulllineitems}


\end{fulllineitems}

\index{ListCommandCommand (class in HardwareBaseClass)@\spxentry{ListCommandCommand}\spxextra{class in HardwareBaseClass}}

\begin{fulllineitems}
\phantomsection\label{\detokenize{webserverdocs:HardwareBaseClass.ListCommandCommand}}
\pysigstartsignatures
\pysiglinewithargsret{\sphinxbfcode{\sphinxupquote{class\DUrole{w}{ }}}\sphinxcode{\sphinxupquote{HardwareBaseClass.}}\sphinxbfcode{\sphinxupquote{ListCommandCommand}}}{\sphinxparam{\DUrole{n}{name}}\sphinxparamcomma \sphinxparam{\DUrole{n}{hardware}\DUrole{o}{=}\DUrole{default_value}{None}}\sphinxparamcomma \sphinxparam{\DUrole{n}{num\_params}\DUrole{o}{=}\DUrole{default_value}{0}}}{}
\pysigstopsignatures
\sphinxAtStartPar
Command to list names of the Hardware class attributes
\index{do() (HardwareBaseClass.ListCommandCommand method)@\spxentry{do()}\spxextra{HardwareBaseClass.ListCommandCommand method}}

\begin{fulllineitems}
\phantomsection\label{\detokenize{webserverdocs:HardwareBaseClass.ListCommandCommand.do}}
\pysigstartsignatures
\pysiglinewithargsret{\sphinxbfcode{\sphinxupquote{do}}}{\sphinxparam{\DUrole{n}{params}\DUrole{o}{=}\DUrole{default_value}{None}}}{}
\pysigstopsignatures\begin{quote}\begin{description}
\sphinxlineitem{Parameters}
\sphinxAtStartPar
\sphinxstyleliteralstrong{\sphinxupquote{params}} (\sphinxstyleliteralemphasis{\sphinxupquote{list}}) \textendash{} Optional list of parameters

\sphinxlineitem{Returns}
\sphinxAtStartPar
Dictionary of returned response

\sphinxlineitem{Return type}
\sphinxAtStartPar
dict

\end{description}\end{quote}

\end{fulllineitems}


\end{fulllineitems}

\index{module@\spxentry{module}!HardwareThreadedClass@\spxentry{HardwareThreadedClass}}\index{HardwareThreadedClass@\spxentry{HardwareThreadedClass}!module@\spxentry{module}}\index{AbortCommand (class in HardwareThreadedClass)@\spxentry{AbortCommand}\spxextra{class in HardwareThreadedClass}}\phantomsection\label{\detokenize{webserverdocs:module-HardwareThreadedClass}}

\begin{fulllineitems}
\phantomsection\label{\detokenize{webserverdocs:HardwareThreadedClass.AbortCommand}}
\pysigstartsignatures
\pysiglinewithargsret{\sphinxbfcode{\sphinxupquote{class\DUrole{w}{ }}}\sphinxcode{\sphinxupquote{HardwareThreadedClass.}}\sphinxbfcode{\sphinxupquote{AbortCommand}}}{\sphinxparam{\DUrole{n}{name}}\sphinxparamcomma \sphinxparam{\DUrole{n}{hardware}\DUrole{o}{=}\DUrole{default_value}{None}}\sphinxparamcomma \sphinxparam{\DUrole{n}{num\_params}\DUrole{o}{=}\DUrole{default_value}{0}}}{}
\pysigstopsignatures
\sphinxAtStartPar
Abort a thread (if specfied in the parameter) or the blocking thread
\index{do() (HardwareThreadedClass.AbortCommand method)@\spxentry{do()}\spxextra{HardwareThreadedClass.AbortCommand method}}

\begin{fulllineitems}
\phantomsection\label{\detokenize{webserverdocs:HardwareThreadedClass.AbortCommand.do}}
\pysigstartsignatures
\pysiglinewithargsret{\sphinxbfcode{\sphinxupquote{do}}}{\sphinxparam{\DUrole{n}{param}\DUrole{o}{=}\DUrole{default_value}{None}}}{}
\pysigstopsignatures\begin{quote}\begin{description}
\sphinxlineitem{Param}
\sphinxAtStartPar
Command to abort or None

\sphinxlineitem{Type}
\sphinxAtStartPar
class.ThreadedHardwareCommand

\end{description}\end{quote}

\end{fulllineitems}


\end{fulllineitems}

\index{IsCompletedCommand (class in HardwareThreadedClass)@\spxentry{IsCompletedCommand}\spxextra{class in HardwareThreadedClass}}

\begin{fulllineitems}
\phantomsection\label{\detokenize{webserverdocs:HardwareThreadedClass.IsCompletedCommand}}
\pysigstartsignatures
\pysiglinewithargsret{\sphinxbfcode{\sphinxupquote{class\DUrole{w}{ }}}\sphinxcode{\sphinxupquote{HardwareThreadedClass.}}\sphinxbfcode{\sphinxupquote{IsCompletedCommand}}}{\sphinxparam{\DUrole{n}{name}}\sphinxparamcomma \sphinxparam{\DUrole{n}{hardware}\DUrole{o}{=}\DUrole{default_value}{None}}\sphinxparamcomma \sphinxparam{\DUrole{n}{num\_params}\DUrole{o}{=}\DUrole{default_value}{0}}}{}
\pysigstopsignatures
\sphinxAtStartPar
check if a specific command thread (if specified in the parameter)
or the blocking thread is currently running
\index{do() (HardwareThreadedClass.IsCompletedCommand method)@\spxentry{do()}\spxextra{HardwareThreadedClass.IsCompletedCommand method}}

\begin{fulllineitems}
\phantomsection\label{\detokenize{webserverdocs:HardwareThreadedClass.IsCompletedCommand.do}}
\pysigstartsignatures
\pysiglinewithargsret{\sphinxbfcode{\sphinxupquote{do}}}{\sphinxparam{\DUrole{n}{param}\DUrole{o}{=}\DUrole{default_value}{\textquotesingle{}\textquotesingle{}}}}{}
\pysigstopsignatures\begin{quote}\begin{description}
\sphinxlineitem{Param}
\sphinxAtStartPar
Command to check or None

\sphinxlineitem{Type}
\sphinxAtStartPar
class.ThreadedHardwareCommand

\end{description}\end{quote}

\end{fulllineitems}


\end{fulllineitems}

\index{ThreadedHardwareCommand (class in HardwareThreadedClass)@\spxentry{ThreadedHardwareCommand}\spxextra{class in HardwareThreadedClass}}

\begin{fulllineitems}
\phantomsection\label{\detokenize{webserverdocs:HardwareThreadedClass.ThreadedHardwareCommand}}
\pysigstartsignatures
\pysiglinewithargsret{\sphinxbfcode{\sphinxupquote{class\DUrole{w}{ }}}\sphinxcode{\sphinxupquote{HardwareThreadedClass.}}\sphinxbfcode{\sphinxupquote{ThreadedHardwareCommand}}}{\sphinxparam{\DUrole{n}{name}}\sphinxparamcomma \sphinxparam{\DUrole{n}{hardware}\DUrole{o}{=}\DUrole{default_value}{None}}\sphinxparamcomma \sphinxparam{\DUrole{n}{num\_params}\DUrole{o}{=}\DUrole{default_value}{0}}\sphinxparamcomma \sphinxparam{\DUrole{n}{blocking}\DUrole{o}{=}\DUrole{default_value}{False}}}{}
\pysigstopsignatures
\sphinxAtStartPar
An hardware command which requires long time to complete.
A thread is started to execute the command. The thread must contain
frequent checkpoints, where the \_abort flag is checked and the
action stopped if this is set.
The completed() method checks for completion of the thread.
\index{abort() (HardwareThreadedClass.ThreadedHardwareCommand method)@\spxentry{abort()}\spxextra{HardwareThreadedClass.ThreadedHardwareCommand method}}

\begin{fulllineitems}
\phantomsection\label{\detokenize{webserverdocs:HardwareThreadedClass.ThreadedHardwareCommand.abort}}
\pysigstartsignatures
\pysiglinewithargsret{\sphinxbfcode{\sphinxupquote{abort}}}{}{}
\pysigstopsignatures
\sphinxAtStartPar
abort the current thread.
Just sets the abort flag, and waits for the thread to complete

\end{fulllineitems}

\index{completed() (HardwareThreadedClass.ThreadedHardwareCommand method)@\spxentry{completed()}\spxextra{HardwareThreadedClass.ThreadedHardwareCommand method}}

\begin{fulllineitems}
\phantomsection\label{\detokenize{webserverdocs:HardwareThreadedClass.ThreadedHardwareCommand.completed}}
\pysigstartsignatures
\pysiglinewithargsret{\sphinxbfcode{\sphinxupquote{completed}}}{}{}
\pysigstopsignatures
\sphinxAtStartPar
checks whether the current thread has completed
If it has, joins the thread (to free resources)
:return: Whether the thread has completed
:rtype: Bool

\end{fulllineitems}

\index{do() (HardwareThreadedClass.ThreadedHardwareCommand method)@\spxentry{do()}\spxextra{HardwareThreadedClass.ThreadedHardwareCommand method}}

\begin{fulllineitems}
\phantomsection\label{\detokenize{webserverdocs:HardwareThreadedClass.ThreadedHardwareCommand.do}}
\pysigstartsignatures
\pysiglinewithargsret{\sphinxbfcode{\sphinxupquote{do}}}{\sphinxparam{\DUrole{n}{params}}}{}
\pysigstopsignatures
\sphinxAtStartPar
Command execution.
Starts a thread and sets internal atributes to
:param params: Command parameters
:return: Dictionary of returned response
:rtype: dict

\end{fulllineitems}

\index{is\_blocking() (HardwareThreadedClass.ThreadedHardwareCommand method)@\spxentry{is\_blocking()}\spxextra{HardwareThreadedClass.ThreadedHardwareCommand method}}

\begin{fulllineitems}
\phantomsection\label{\detokenize{webserverdocs:HardwareThreadedClass.ThreadedHardwareCommand.is_blocking}}
\pysigstartsignatures
\pysiglinewithargsret{\sphinxbfcode{\sphinxupquote{is\_blocking}}}{}{}
\pysigstopsignatures\begin{quote}\begin{description}
\sphinxlineitem{Returns}
\sphinxAtStartPar
Whether the command requires blocking of the device

\sphinxlineitem{Return type}
\sphinxAtStartPar
Bool

\end{description}\end{quote}

\end{fulllineitems}

\index{thread() (HardwareThreadedClass.ThreadedHardwareCommand method)@\spxentry{thread()}\spxextra{HardwareThreadedClass.ThreadedHardwareCommand method}}

\begin{fulllineitems}
\phantomsection\label{\detokenize{webserverdocs:HardwareThreadedClass.ThreadedHardwareCommand.thread}}
\pysigstartsignatures
\pysiglinewithargsret{\sphinxbfcode{\sphinxupquote{thread}}}{\sphinxparam{\DUrole{n}{params}}}{}
\pysigstopsignatures
\sphinxAtStartPar
Execution thread. Must periodically check self.\_abort
with maximum execution time of approx 1 second between checkpoints
Terminates setting self.\_completed True.

\end{fulllineitems}


\end{fulllineitems}

\index{module@\spxentry{module}!subrack\_hardware@\spxentry{subrack\_hardware}}\index{subrack\_hardware@\spxentry{subrack\_hardware}!module@\spxentry{module}}\phantomsection\label{\detokenize{webserverdocs:module-subrack_hardware}}
\sphinxAtStartPar
LFAA SPS Subrack control board hardware driver.
\index{API\_Version (class in subrack\_hardware)@\spxentry{API\_Version}\spxextra{class in subrack\_hardware}}

\begin{fulllineitems}
\phantomsection\label{\detokenize{webserverdocs:subrack_hardware.API_Version}}
\pysigstartsignatures
\pysiglinewithargsret{\sphinxbfcode{\sphinxupquote{class\DUrole{w}{ }}}\sphinxcode{\sphinxupquote{subrack\_hardware.}}\sphinxbfcode{\sphinxupquote{API\_Version}}}{\sphinxparam{\DUrole{n}{name}}\sphinxparamcomma \sphinxparam{\DUrole{n}{init\_value}}\sphinxparamcomma \sphinxparam{\DUrole{n}{hardware}\DUrole{o}{=}\DUrole{default_value}{None}}\sphinxparamcomma \sphinxparam{\DUrole{n}{read\_write}\DUrole{o}{=}\DUrole{default_value}{0}}\sphinxparamcomma \sphinxparam{\DUrole{n}{num\_params}\DUrole{o}{=}\DUrole{default_value}{1}}}{}
\pysigstopsignatures
\sphinxAtStartPar
API version
Returns version of API
\index{read\_value() (subrack\_hardware.API\_Version method)@\spxentry{read\_value()}\spxextra{subrack\_hardware.API\_Version method}}

\begin{fulllineitems}
\phantomsection\label{\detokenize{webserverdocs:subrack_hardware.API_Version.read_value}}
\pysigstartsignatures
\pysiglinewithargsret{\sphinxbfcode{\sphinxupquote{read\_value}}}{}{}
\pysigstopsignatures
\sphinxAtStartPar
Reads the actual value
To be overrided in the subclass
\begin{quote}\begin{description}
\sphinxlineitem{Returns}
\sphinxAtStartPar
Attribute true value, from real device

\end{description}\end{quote}

\end{fulllineitems}


\end{fulllineitems}

\index{AreTpmsOnCommand (class in subrack\_hardware)@\spxentry{AreTpmsOnCommand}\spxextra{class in subrack\_hardware}}

\begin{fulllineitems}
\phantomsection\label{\detokenize{webserverdocs:subrack_hardware.AreTpmsOnCommand}}
\pysigstartsignatures
\pysiglinewithargsret{\sphinxbfcode{\sphinxupquote{class\DUrole{w}{ }}}\sphinxcode{\sphinxupquote{subrack\_hardware.}}\sphinxbfcode{\sphinxupquote{AreTpmsOnCommand}}}{\sphinxparam{\DUrole{n}{name}}\sphinxparamcomma \sphinxparam{\DUrole{n}{hardware}\DUrole{o}{=}\DUrole{default_value}{None}}\sphinxparamcomma \sphinxparam{\DUrole{n}{num\_params}\DUrole{o}{=}\DUrole{default_value}{0}}}{}
\pysigstopsignatures\index{do() (subrack\_hardware.AreTpmsOnCommand method)@\spxentry{do()}\spxextra{subrack\_hardware.AreTpmsOnCommand method}}

\begin{fulllineitems}
\phantomsection\label{\detokenize{webserverdocs:subrack_hardware.AreTpmsOnCommand.do}}
\pysigstartsignatures
\pysiglinewithargsret{\sphinxbfcode{\sphinxupquote{do}}}{\sphinxparam{\DUrole{n}{params}}}{}
\pysigstopsignatures
\sphinxAtStartPar
Execution method. Subclasses must override this to do real work
\begin{quote}\begin{description}
\sphinxlineitem{Parameters}
\sphinxAtStartPar
\sphinxstyleliteralstrong{\sphinxupquote{params}} (\sphinxstyleliteralemphasis{\sphinxupquote{list}}) \textendash{} Optional list of parameters

\sphinxlineitem{Returns}
\sphinxAtStartPar
Dictionary of returned response

\sphinxlineitem{Return type}
\sphinxAtStartPar
dict

\end{description}\end{quote}

\end{fulllineitems}


\end{fulllineitems}

\index{BackplaneTemperature (class in subrack\_hardware)@\spxentry{BackplaneTemperature}\spxextra{class in subrack\_hardware}}

\begin{fulllineitems}
\phantomsection\label{\detokenize{webserverdocs:subrack_hardware.BackplaneTemperature}}
\pysigstartsignatures
\pysiglinewithargsret{\sphinxbfcode{\sphinxupquote{class\DUrole{w}{ }}}\sphinxcode{\sphinxupquote{subrack\_hardware.}}\sphinxbfcode{\sphinxupquote{BackplaneTemperature}}}{\sphinxparam{\DUrole{n}{name}}\sphinxparamcomma \sphinxparam{\DUrole{n}{init\_value}}\sphinxparamcomma \sphinxparam{\DUrole{n}{hardware}\DUrole{o}{=}\DUrole{default_value}{None}}\sphinxparamcomma \sphinxparam{\DUrole{n}{read\_write}\DUrole{o}{=}\DUrole{default_value}{0}}\sphinxparamcomma \sphinxparam{\DUrole{n}{num\_params}\DUrole{o}{=}\DUrole{default_value}{1}}}{}
\pysigstopsignatures
\sphinxAtStartPar
Backplane temperature, in celsius
\index{read\_value() (subrack\_hardware.BackplaneTemperature method)@\spxentry{read\_value()}\spxextra{subrack\_hardware.BackplaneTemperature method}}

\begin{fulllineitems}
\phantomsection\label{\detokenize{webserverdocs:subrack_hardware.BackplaneTemperature.read_value}}
\pysigstartsignatures
\pysiglinewithargsret{\sphinxbfcode{\sphinxupquote{read\_value}}}{}{}
\pysigstopsignatures\begin{quote}\begin{description}
\sphinxlineitem{Returns}
\sphinxAtStartPar
backplane temperature, in celsius, for the two backplane halves

\sphinxlineitem{Return type}
\sphinxAtStartPar
list{[}float{]}

\end{description}\end{quote}

\end{fulllineitems}


\end{fulllineitems}

\index{BoardCurrent (class in subrack\_hardware)@\spxentry{BoardCurrent}\spxextra{class in subrack\_hardware}}

\begin{fulllineitems}
\phantomsection\label{\detokenize{webserverdocs:subrack_hardware.BoardCurrent}}
\pysigstartsignatures
\pysiglinewithargsret{\sphinxbfcode{\sphinxupquote{class\DUrole{w}{ }}}\sphinxcode{\sphinxupquote{subrack\_hardware.}}\sphinxbfcode{\sphinxupquote{BoardCurrent}}}{\sphinxparam{\DUrole{n}{name}}\sphinxparamcomma \sphinxparam{\DUrole{n}{init\_value}}\sphinxparamcomma \sphinxparam{\DUrole{n}{hardware}\DUrole{o}{=}\DUrole{default_value}{None}}\sphinxparamcomma \sphinxparam{\DUrole{n}{read\_write}\DUrole{o}{=}\DUrole{default_value}{0}}\sphinxparamcomma \sphinxparam{\DUrole{n}{num\_params}\DUrole{o}{=}\DUrole{default_value}{1}}}{}
\pysigstopsignatures\index{read\_value() (subrack\_hardware.BoardCurrent method)@\spxentry{read\_value()}\spxextra{subrack\_hardware.BoardCurrent method}}

\begin{fulllineitems}
\phantomsection\label{\detokenize{webserverdocs:subrack_hardware.BoardCurrent.read_value}}
\pysigstartsignatures
\pysiglinewithargsret{\sphinxbfcode{\sphinxupquote{read\_value}}}{}{}
\pysigstopsignatures\begin{quote}\begin{description}
\sphinxlineitem{Returns}
\sphinxAtStartPar
Total subrack current (A)

\sphinxlineitem{Return type}
\sphinxAtStartPar
float

\end{description}\end{quote}

\end{fulllineitems}


\end{fulllineitems}

\index{BoardTemperature (class in subrack\_hardware)@\spxentry{BoardTemperature}\spxextra{class in subrack\_hardware}}

\begin{fulllineitems}
\phantomsection\label{\detokenize{webserverdocs:subrack_hardware.BoardTemperature}}
\pysigstartsignatures
\pysiglinewithargsret{\sphinxbfcode{\sphinxupquote{class\DUrole{w}{ }}}\sphinxcode{\sphinxupquote{subrack\_hardware.}}\sphinxbfcode{\sphinxupquote{BoardTemperature}}}{\sphinxparam{\DUrole{n}{name}}\sphinxparamcomma \sphinxparam{\DUrole{n}{init\_value}}\sphinxparamcomma \sphinxparam{\DUrole{n}{hardware}\DUrole{o}{=}\DUrole{default_value}{None}}\sphinxparamcomma \sphinxparam{\DUrole{n}{read\_write}\DUrole{o}{=}\DUrole{default_value}{0}}\sphinxparamcomma \sphinxparam{\DUrole{n}{num\_params}\DUrole{o}{=}\DUrole{default_value}{1}}}{}
\pysigstopsignatures\index{read\_value() (subrack\_hardware.BoardTemperature method)@\spxentry{read\_value()}\spxextra{subrack\_hardware.BoardTemperature method}}

\begin{fulllineitems}
\phantomsection\label{\detokenize{webserverdocs:subrack_hardware.BoardTemperature.read_value}}
\pysigstartsignatures
\pysiglinewithargsret{\sphinxbfcode{\sphinxupquote{read\_value}}}{}{}
\pysigstopsignatures\begin{quote}\begin{description}
\sphinxlineitem{Returns}
\sphinxAtStartPar
Subrack control board temperature, in celsius, 2 values

\sphinxlineitem{Return type}
\sphinxAtStartPar
list{[}float{]}

\end{description}\end{quote}

\end{fulllineitems}


\end{fulllineitems}

\index{CPLD\_PLL\_Locked (class in subrack\_hardware)@\spxentry{CPLD\_PLL\_Locked}\spxextra{class in subrack\_hardware}}

\begin{fulllineitems}
\phantomsection\label{\detokenize{webserverdocs:subrack_hardware.CPLD_PLL_Locked}}
\pysigstartsignatures
\pysiglinewithargsret{\sphinxbfcode{\sphinxupquote{class\DUrole{w}{ }}}\sphinxcode{\sphinxupquote{subrack\_hardware.}}\sphinxbfcode{\sphinxupquote{CPLD\_PLL\_Locked}}}{\sphinxparam{\DUrole{n}{name}}\sphinxparamcomma \sphinxparam{\DUrole{n}{init\_value}}\sphinxparamcomma \sphinxparam{\DUrole{n}{hardware}\DUrole{o}{=}\DUrole{default_value}{None}}\sphinxparamcomma \sphinxparam{\DUrole{n}{read\_write}\DUrole{o}{=}\DUrole{default_value}{0}}\sphinxparamcomma \sphinxparam{\DUrole{n}{num\_params}\DUrole{o}{=}\DUrole{default_value}{1}}}{}
\pysigstopsignatures
\sphinxAtStartPar
Subrack CPLD PLL Lock status
Returns status of CPLD internal PLL lock
\index{read\_value() (subrack\_hardware.CPLD\_PLL\_Locked method)@\spxentry{read\_value()}\spxextra{subrack\_hardware.CPLD\_PLL\_Locked method}}

\begin{fulllineitems}
\phantomsection\label{\detokenize{webserverdocs:subrack_hardware.CPLD_PLL_Locked.read_value}}
\pysigstartsignatures
\pysiglinewithargsret{\sphinxbfcode{\sphinxupquote{read\_value}}}{}{}
\pysigstopsignatures
\sphinxAtStartPar
Reads the actual value
To be overrided in the subclass
\begin{quote}\begin{description}
\sphinxlineitem{Returns}
\sphinxAtStartPar
Attribute true value, from real device

\end{description}\end{quote}

\end{fulllineitems}


\end{fulllineitems}

\index{FanMode (class in subrack\_hardware)@\spxentry{FanMode}\spxextra{class in subrack\_hardware}}

\begin{fulllineitems}
\phantomsection\label{\detokenize{webserverdocs:subrack_hardware.FanMode}}
\pysigstartsignatures
\pysiglinewithargsret{\sphinxbfcode{\sphinxupquote{class\DUrole{w}{ }}}\sphinxcode{\sphinxupquote{subrack\_hardware.}}\sphinxbfcode{\sphinxupquote{FanMode}}}{\sphinxparam{\DUrole{n}{name}}\sphinxparamcomma \sphinxparam{\DUrole{n}{init\_value}}\sphinxparamcomma \sphinxparam{\DUrole{n}{hardware}\DUrole{o}{=}\DUrole{default_value}{None}}\sphinxparamcomma \sphinxparam{\DUrole{n}{read\_write}\DUrole{o}{=}\DUrole{default_value}{0}}\sphinxparamcomma \sphinxparam{\DUrole{n}{num\_params}\DUrole{o}{=}\DUrole{default_value}{1}}}{}
\pysigstopsignatures\index{read\_value() (subrack\_hardware.FanMode method)@\spxentry{read\_value()}\spxextra{subrack\_hardware.FanMode method}}

\begin{fulllineitems}
\phantomsection\label{\detokenize{webserverdocs:subrack_hardware.FanMode.read_value}}
\pysigstartsignatures
\pysiglinewithargsret{\sphinxbfcode{\sphinxupquote{read\_value}}}{}{}
\pysigstopsignatures\begin{quote}\begin{description}
\sphinxlineitem{Returns}
\sphinxAtStartPar
Subrack fan mode

\sphinxlineitem{Return type}
\sphinxAtStartPar
list{[}float{]}

\end{description}\end{quote}

\end{fulllineitems}

\index{write\_value() (subrack\_hardware.FanMode method)@\spxentry{write\_value()}\spxextra{subrack\_hardware.FanMode method}}

\begin{fulllineitems}
\phantomsection\label{\detokenize{webserverdocs:subrack_hardware.FanMode.write_value}}
\pysigstartsignatures
\pysiglinewithargsret{\sphinxbfcode{\sphinxupquote{write\_value}}}{\sphinxparam{\DUrole{n}{mode}}}{}
\pysigstopsignatures
\sphinxAtStartPar
Writes the actual value
To be overrided in the subclass
\begin{quote}\begin{description}
\sphinxlineitem{Parameters}
\sphinxAtStartPar
\sphinxstyleliteralstrong{\sphinxupquote{value}} \textendash{} Attribute true value, to real device

\end{description}\end{quote}

\end{fulllineitems}


\end{fulllineitems}

\index{FanSpeed (class in subrack\_hardware)@\spxentry{FanSpeed}\spxextra{class in subrack\_hardware}}

\begin{fulllineitems}
\phantomsection\label{\detokenize{webserverdocs:subrack_hardware.FanSpeed}}
\pysigstartsignatures
\pysiglinewithargsret{\sphinxbfcode{\sphinxupquote{class\DUrole{w}{ }}}\sphinxcode{\sphinxupquote{subrack\_hardware.}}\sphinxbfcode{\sphinxupquote{FanSpeed}}}{\sphinxparam{\DUrole{n}{name}}\sphinxparamcomma \sphinxparam{\DUrole{n}{init\_value}}\sphinxparamcomma \sphinxparam{\DUrole{n}{hardware}\DUrole{o}{=}\DUrole{default_value}{None}}\sphinxparamcomma \sphinxparam{\DUrole{n}{read\_write}\DUrole{o}{=}\DUrole{default_value}{0}}\sphinxparamcomma \sphinxparam{\DUrole{n}{num\_params}\DUrole{o}{=}\DUrole{default_value}{1}}}{}
\pysigstopsignatures\index{read\_value() (subrack\_hardware.FanSpeed method)@\spxentry{read\_value()}\spxextra{subrack\_hardware.FanSpeed method}}

\begin{fulllineitems}
\phantomsection\label{\detokenize{webserverdocs:subrack_hardware.FanSpeed.read_value}}
\pysigstartsignatures
\pysiglinewithargsret{\sphinxbfcode{\sphinxupquote{read\_value}}}{}{}
\pysigstopsignatures\begin{quote}\begin{description}
\sphinxlineitem{Returns}
\sphinxAtStartPar
Subrack fan speed, in RPM, for the 4 fans

\sphinxlineitem{Return type}
\sphinxAtStartPar
list{[}float{]}

\end{description}\end{quote}

\end{fulllineitems}


\end{fulllineitems}

\index{FanSpeedPercent (class in subrack\_hardware)@\spxentry{FanSpeedPercent}\spxextra{class in subrack\_hardware}}

\begin{fulllineitems}
\phantomsection\label{\detokenize{webserverdocs:subrack_hardware.FanSpeedPercent}}
\pysigstartsignatures
\pysiglinewithargsret{\sphinxbfcode{\sphinxupquote{class\DUrole{w}{ }}}\sphinxcode{\sphinxupquote{subrack\_hardware.}}\sphinxbfcode{\sphinxupquote{FanSpeedPercent}}}{\sphinxparam{\DUrole{n}{name}}\sphinxparamcomma \sphinxparam{\DUrole{n}{init\_value}}\sphinxparamcomma \sphinxparam{\DUrole{n}{hardware}\DUrole{o}{=}\DUrole{default_value}{None}}\sphinxparamcomma \sphinxparam{\DUrole{n}{read\_write}\DUrole{o}{=}\DUrole{default_value}{0}}\sphinxparamcomma \sphinxparam{\DUrole{n}{num\_params}\DUrole{o}{=}\DUrole{default_value}{1}}}{}
\pysigstopsignatures\index{read\_value() (subrack\_hardware.FanSpeedPercent method)@\spxentry{read\_value()}\spxextra{subrack\_hardware.FanSpeedPercent method}}

\begin{fulllineitems}
\phantomsection\label{\detokenize{webserverdocs:subrack_hardware.FanSpeedPercent.read_value}}
\pysigstartsignatures
\pysiglinewithargsret{\sphinxbfcode{\sphinxupquote{read\_value}}}{}{}
\pysigstopsignatures\begin{quote}\begin{description}
\sphinxlineitem{Returns}
\sphinxAtStartPar
Subrack fan speed, in percent of the maximum values, for the 4 fans

\sphinxlineitem{Return type}
\sphinxAtStartPar
list{[}float{]}

\end{description}\end{quote}

\end{fulllineitems}


\end{fulllineitems}

\index{GetHealthDict (class in subrack\_hardware)@\spxentry{GetHealthDict}\spxextra{class in subrack\_hardware}}

\begin{fulllineitems}
\phantomsection\label{\detokenize{webserverdocs:subrack_hardware.GetHealthDict}}
\pysigstartsignatures
\pysiglinewithargsret{\sphinxbfcode{\sphinxupquote{class\DUrole{w}{ }}}\sphinxcode{\sphinxupquote{subrack\_hardware.}}\sphinxbfcode{\sphinxupquote{GetHealthDict}}}{\sphinxparam{\DUrole{n}{name}}\sphinxparamcomma \sphinxparam{\DUrole{n}{hardware}\DUrole{o}{=}\DUrole{default_value}{None}}\sphinxparamcomma \sphinxparam{\DUrole{n}{num\_params}\DUrole{o}{=}\DUrole{default_value}{0}}}{}
\pysigstopsignatures
\sphinxAtStartPar
Return subrack health status dictionary.
\index{do() (subrack\_hardware.GetHealthDict method)@\spxentry{do()}\spxextra{subrack\_hardware.GetHealthDict method}}

\begin{fulllineitems}
\phantomsection\label{\detokenize{webserverdocs:subrack_hardware.GetHealthDict.do}}
\pysigstartsignatures
\pysiglinewithargsret{\sphinxbfcode{\sphinxupquote{do}}}{\sphinxparam{\DUrole{n}{params}}}{}
\pysigstopsignatures
\sphinxAtStartPar
Info about subrack health status
:param params: group of monitor points to report
:type params: str

\sphinxAtStartPar
:return:dictionary  of monitor points
:rtype: dict

\end{fulllineitems}


\end{fulllineitems}

\index{GetHealthStatus (class in subrack\_hardware)@\spxentry{GetHealthStatus}\spxextra{class in subrack\_hardware}}

\begin{fulllineitems}
\phantomsection\label{\detokenize{webserverdocs:subrack_hardware.GetHealthStatus}}
\pysigstartsignatures
\pysiglinewithargsret{\sphinxbfcode{\sphinxupquote{class\DUrole{w}{ }}}\sphinxcode{\sphinxupquote{subrack\_hardware.}}\sphinxbfcode{\sphinxupquote{GetHealthStatus}}}{\sphinxparam{\DUrole{n}{name}}\sphinxparamcomma \sphinxparam{\DUrole{n}{hardware}\DUrole{o}{=}\DUrole{default_value}{None}}\sphinxparamcomma \sphinxparam{\DUrole{n}{num\_params}\DUrole{o}{=}\DUrole{default_value}{0}}}{}
\pysigstopsignatures
\sphinxAtStartPar
Return info about subrack health status.
\index{do() (subrack\_hardware.GetHealthStatus method)@\spxentry{do()}\spxextra{subrack\_hardware.GetHealthStatus method}}

\begin{fulllineitems}
\phantomsection\label{\detokenize{webserverdocs:subrack_hardware.GetHealthStatus.do}}
\pysigstartsignatures
\pysiglinewithargsret{\sphinxbfcode{\sphinxupquote{do}}}{\sphinxparam{\DUrole{n}{params}}}{}
\pysigstopsignatures
\sphinxAtStartPar
Info about subrack health status
:param params: group of monitor points to report
:type params: str
\begin{quote}\begin{description}
\sphinxlineitem{Returns}
\sphinxAtStartPar
dictionary of monitor point values

\sphinxlineitem{Return type}
\sphinxAtStartPar
dict

\end{description}\end{quote}

\end{fulllineitems}


\end{fulllineitems}

\index{IsTpmOnCommand (class in subrack\_hardware)@\spxentry{IsTpmOnCommand}\spxextra{class in subrack\_hardware}}

\begin{fulllineitems}
\phantomsection\label{\detokenize{webserverdocs:subrack_hardware.IsTpmOnCommand}}
\pysigstartsignatures
\pysiglinewithargsret{\sphinxbfcode{\sphinxupquote{class\DUrole{w}{ }}}\sphinxcode{\sphinxupquote{subrack\_hardware.}}\sphinxbfcode{\sphinxupquote{IsTpmOnCommand}}}{\sphinxparam{\DUrole{n}{name}}\sphinxparamcomma \sphinxparam{\DUrole{n}{hardware}\DUrole{o}{=}\DUrole{default_value}{None}}\sphinxparamcomma \sphinxparam{\DUrole{n}{num\_params}\DUrole{o}{=}\DUrole{default_value}{0}}}{}
\pysigstopsignatures
\sphinxAtStartPar
Check TPM power status
\index{do() (subrack\_hardware.IsTpmOnCommand method)@\spxentry{do()}\spxextra{subrack\_hardware.IsTpmOnCommand method}}

\begin{fulllineitems}
\phantomsection\label{\detokenize{webserverdocs:subrack_hardware.IsTpmOnCommand.do}}
\pysigstartsignatures
\pysiglinewithargsret{\sphinxbfcode{\sphinxupquote{do}}}{\sphinxparam{\DUrole{n}{params}}}{}
\pysigstopsignatures
\sphinxAtStartPar
Check power status of TPMs
\begin{quote}\begin{description}
\sphinxlineitem{Parameters}
\sphinxAtStartPar
\sphinxstyleliteralstrong{\sphinxupquote{params}} (\sphinxstyleliteralemphasis{\sphinxupquote{str}}) \textendash{} index of TPM to check (1\sphinxhyphen{}8)

\sphinxlineitem{Returns}
\sphinxAtStartPar
dictionary with HardwareCommand response. Integer retvalue

\sphinxlineitem{Return type}
\sphinxAtStartPar
dict

\end{description}\end{quote}

\end{fulllineitems}


\end{fulllineitems}

\index{PSCurrent (class in subrack\_hardware)@\spxentry{PSCurrent}\spxextra{class in subrack\_hardware}}

\begin{fulllineitems}
\phantomsection\label{\detokenize{webserverdocs:subrack_hardware.PSCurrent}}
\pysigstartsignatures
\pysiglinewithargsret{\sphinxbfcode{\sphinxupquote{class\DUrole{w}{ }}}\sphinxcode{\sphinxupquote{subrack\_hardware.}}\sphinxbfcode{\sphinxupquote{PSCurrent}}}{\sphinxparam{\DUrole{n}{name}}\sphinxparamcomma \sphinxparam{\DUrole{n}{init\_value}}\sphinxparamcomma \sphinxparam{\DUrole{n}{hardware}\DUrole{o}{=}\DUrole{default_value}{None}}\sphinxparamcomma \sphinxparam{\DUrole{n}{read\_write}\DUrole{o}{=}\DUrole{default_value}{0}}\sphinxparamcomma \sphinxparam{\DUrole{n}{num\_params}\DUrole{o}{=}\DUrole{default_value}{1}}}{}
\pysigstopsignatures
\sphinxAtStartPar
Hardware attribute class for reading power supply currents.

\sphinxAtStartPar
Reads current values for power supplies 1 and 2 and returns a list.

\sphinxAtStartPar
Attributes:
\sphinxhyphen{} \_hardware: Reference to the SubrackMngBoard hardware object.

\sphinxAtStartPar
Methods:
\sphinxhyphen{} read\_value(): Reads power supply currents and returns a list.
\index{read\_value() (subrack\_hardware.PSCurrent method)@\spxentry{read\_value()}\spxextra{subrack\_hardware.PSCurrent method}}

\begin{fulllineitems}
\phantomsection\label{\detokenize{webserverdocs:subrack_hardware.PSCurrent.read_value}}
\pysigstartsignatures
\pysiglinewithargsret{\sphinxbfcode{\sphinxupquote{read\_value}}}{}{}
\pysigstopsignatures
\sphinxAtStartPar
Reads power supply currents.

\sphinxAtStartPar
Returns:
list: List of current values for power supplies 1 and 2.

\end{fulllineitems}


\end{fulllineitems}

\index{PSFanSpeed (class in subrack\_hardware)@\spxentry{PSFanSpeed}\spxextra{class in subrack\_hardware}}

\begin{fulllineitems}
\phantomsection\label{\detokenize{webserverdocs:subrack_hardware.PSFanSpeed}}
\pysigstartsignatures
\pysiglinewithargsret{\sphinxbfcode{\sphinxupquote{class\DUrole{w}{ }}}\sphinxcode{\sphinxupquote{subrack\_hardware.}}\sphinxbfcode{\sphinxupquote{PSFanSpeed}}}{\sphinxparam{\DUrole{n}{name}}\sphinxparamcomma \sphinxparam{\DUrole{n}{init\_value}}\sphinxparamcomma \sphinxparam{\DUrole{n}{hardware}\DUrole{o}{=}\DUrole{default_value}{None}}\sphinxparamcomma \sphinxparam{\DUrole{n}{read\_write}\DUrole{o}{=}\DUrole{default_value}{0}}\sphinxparamcomma \sphinxparam{\DUrole{n}{num\_params}\DUrole{o}{=}\DUrole{default_value}{1}}}{}
\pysigstopsignatures
\sphinxAtStartPar
Hardware attribute class for reading power supply fan speeds.

\sphinxAtStartPar
Reads fan speeds for power supplies 1 and 2 and returns a list of values.

\sphinxAtStartPar
Attributes:
\sphinxhyphen{} \_hardware: Reference to the SubrackMngBoard hardware object.

\sphinxAtStartPar
Methods:
\sphinxhyphen{} read\_value(): Reads power supply fan speeds and returns a list.
\index{read\_value() (subrack\_hardware.PSFanSpeed method)@\spxentry{read\_value()}\spxextra{subrack\_hardware.PSFanSpeed method}}

\begin{fulllineitems}
\phantomsection\label{\detokenize{webserverdocs:subrack_hardware.PSFanSpeed.read_value}}
\pysigstartsignatures
\pysiglinewithargsret{\sphinxbfcode{\sphinxupquote{read\_value}}}{}{}
\pysigstopsignatures
\sphinxAtStartPar
Reads power supply fan speeds.

\sphinxAtStartPar
Returns:
list: List of fan speeds for power supplies 1 and 2.

\end{fulllineitems}


\end{fulllineitems}

\index{PSPower (class in subrack\_hardware)@\spxentry{PSPower}\spxextra{class in subrack\_hardware}}

\begin{fulllineitems}
\phantomsection\label{\detokenize{webserverdocs:subrack_hardware.PSPower}}
\pysigstartsignatures
\pysiglinewithargsret{\sphinxbfcode{\sphinxupquote{class\DUrole{w}{ }}}\sphinxcode{\sphinxupquote{subrack\_hardware.}}\sphinxbfcode{\sphinxupquote{PSPower}}}{\sphinxparam{\DUrole{n}{name}}\sphinxparamcomma \sphinxparam{\DUrole{n}{init\_value}}\sphinxparamcomma \sphinxparam{\DUrole{n}{hardware}\DUrole{o}{=}\DUrole{default_value}{None}}\sphinxparamcomma \sphinxparam{\DUrole{n}{read\_write}\DUrole{o}{=}\DUrole{default_value}{0}}\sphinxparamcomma \sphinxparam{\DUrole{n}{num\_params}\DUrole{o}{=}\DUrole{default_value}{1}}}{}
\pysigstopsignatures
\sphinxAtStartPar
Hardware attribute class for reading power supply powers.

\sphinxAtStartPar
Reads power values for power supplies 1 and 2 and returns a list.

\sphinxAtStartPar
Attributes:
\sphinxhyphen{} \_hardware: Reference to the SubrackMngBoard hardware object.

\sphinxAtStartPar
Methods:
\sphinxhyphen{} read\_value(): Reads power supply powers and returns a list.
\index{read\_value() (subrack\_hardware.PSPower method)@\spxentry{read\_value()}\spxextra{subrack\_hardware.PSPower method}}

\begin{fulllineitems}
\phantomsection\label{\detokenize{webserverdocs:subrack_hardware.PSPower.read_value}}
\pysigstartsignatures
\pysiglinewithargsret{\sphinxbfcode{\sphinxupquote{read\_value}}}{}{}
\pysigstopsignatures
\sphinxAtStartPar
Reads power supply powers.

\sphinxAtStartPar
Returns:
list: List of power values for power supplies 1 and 2.

\end{fulllineitems}


\end{fulllineitems}

\index{PSVoltage (class in subrack\_hardware)@\spxentry{PSVoltage}\spxextra{class in subrack\_hardware}}

\begin{fulllineitems}
\phantomsection\label{\detokenize{webserverdocs:subrack_hardware.PSVoltage}}
\pysigstartsignatures
\pysiglinewithargsret{\sphinxbfcode{\sphinxupquote{class\DUrole{w}{ }}}\sphinxcode{\sphinxupquote{subrack\_hardware.}}\sphinxbfcode{\sphinxupquote{PSVoltage}}}{\sphinxparam{\DUrole{n}{name}}\sphinxparamcomma \sphinxparam{\DUrole{n}{init\_value}}\sphinxparamcomma \sphinxparam{\DUrole{n}{hardware}\DUrole{o}{=}\DUrole{default_value}{None}}\sphinxparamcomma \sphinxparam{\DUrole{n}{read\_write}\DUrole{o}{=}\DUrole{default_value}{0}}\sphinxparamcomma \sphinxparam{\DUrole{n}{num\_params}\DUrole{o}{=}\DUrole{default_value}{1}}}{}
\pysigstopsignatures
\sphinxAtStartPar
Hardware attribute class for reading power supply voltages.

\sphinxAtStartPar
Reads voltage values for power supplies 1 and 2 and returns a list.

\sphinxAtStartPar
Attributes:
\sphinxhyphen{} \_hardware: Reference to the SubrackMngBoard hardware object.

\sphinxAtStartPar
Methods:
\sphinxhyphen{} read\_value(): Reads power supply voltages and returns a list.
\index{read\_value() (subrack\_hardware.PSVoltage method)@\spxentry{read\_value()}\spxextra{subrack\_hardware.PSVoltage method}}

\begin{fulllineitems}
\phantomsection\label{\detokenize{webserverdocs:subrack_hardware.PSVoltage.read_value}}
\pysigstartsignatures
\pysiglinewithargsret{\sphinxbfcode{\sphinxupquote{read\_value}}}{}{}
\pysigstopsignatures
\sphinxAtStartPar
Reads power supply voltages.

\sphinxAtStartPar
Returns:
list: List of voltage values for power supplies 1 and 2.

\end{fulllineitems}


\end{fulllineitems}

\index{PowerDownCommand (class in subrack\_hardware)@\spxentry{PowerDownCommand}\spxextra{class in subrack\_hardware}}

\begin{fulllineitems}
\phantomsection\label{\detokenize{webserverdocs:subrack_hardware.PowerDownCommand}}
\pysigstartsignatures
\pysiglinewithargsret{\sphinxbfcode{\sphinxupquote{class\DUrole{w}{ }}}\sphinxcode{\sphinxupquote{subrack\_hardware.}}\sphinxbfcode{\sphinxupquote{PowerDownCommand}}}{\sphinxparam{\DUrole{n}{name}}\sphinxparamcomma \sphinxparam{\DUrole{n}{hardware}\DUrole{o}{=}\DUrole{default_value}{None}}\sphinxparamcomma \sphinxparam{\DUrole{n}{num\_params}\DUrole{o}{=}\DUrole{default_value}{0}}\sphinxparamcomma \sphinxparam{\DUrole{n}{blocking}\DUrole{o}{=}\DUrole{default_value}{False}}}{}
\pysigstopsignatures
\sphinxAtStartPar
Power off all TPMs
\index{thread() (subrack\_hardware.PowerDownCommand method)@\spxentry{thread()}\spxextra{subrack\_hardware.PowerDownCommand method}}

\begin{fulllineitems}
\phantomsection\label{\detokenize{webserverdocs:subrack_hardware.PowerDownCommand.thread}}
\pysigstartsignatures
\pysiglinewithargsret{\sphinxbfcode{\sphinxupquote{thread}}}{\sphinxparam{\DUrole{n}{params}}}{}
\pysigstopsignatures
\sphinxAtStartPar
Power off all TPMs
\begin{quote}\begin{description}
\sphinxlineitem{Parameters}
\sphinxAtStartPar
\sphinxstyleliteralstrong{\sphinxupquote{params}} \textendash{} unused

\end{description}\end{quote}

\end{fulllineitems}


\end{fulllineitems}

\index{PowerOffTpmCommand (class in subrack\_hardware)@\spxentry{PowerOffTpmCommand}\spxextra{class in subrack\_hardware}}

\begin{fulllineitems}
\phantomsection\label{\detokenize{webserverdocs:subrack_hardware.PowerOffTpmCommand}}
\pysigstartsignatures
\pysiglinewithargsret{\sphinxbfcode{\sphinxupquote{class\DUrole{w}{ }}}\sphinxcode{\sphinxupquote{subrack\_hardware.}}\sphinxbfcode{\sphinxupquote{PowerOffTpmCommand}}}{\sphinxparam{\DUrole{n}{name}}\sphinxparamcomma \sphinxparam{\DUrole{n}{hardware}\DUrole{o}{=}\DUrole{default_value}{None}}\sphinxparamcomma \sphinxparam{\DUrole{n}{num\_params}\DUrole{o}{=}\DUrole{default_value}{0}}\sphinxparamcomma \sphinxparam{\DUrole{n}{blocking}\DUrole{o}{=}\DUrole{default_value}{False}}}{}
\pysigstopsignatures
\sphinxAtStartPar
Power Off TPM command. Switches off a single or multiple TPM
\index{thread() (subrack\_hardware.PowerOffTpmCommand method)@\spxentry{thread()}\spxextra{subrack\_hardware.PowerOffTpmCommand method}}

\begin{fulllineitems}
\phantomsection\label{\detokenize{webserverdocs:subrack_hardware.PowerOffTpmCommand.thread}}
\pysigstartsignatures
\pysiglinewithargsret{\sphinxbfcode{\sphinxupquote{thread}}}{\sphinxparam{\DUrole{n}{tpm\_id}}}{}
\pysigstopsignatures
\sphinxAtStartPar
Power off TPMs
\begin{quote}\begin{description}
\sphinxlineitem{Parameters}
\sphinxAtStartPar
\sphinxstyleliteralstrong{\sphinxupquote{tpm\_id}} (\sphinxstyleliteralemphasis{\sphinxupquote{str}}\sphinxstyleliteralemphasis{\sphinxupquote{, }}\sphinxstyleliteralemphasis{\sphinxupquote{list}}\sphinxstyleliteralemphasis{\sphinxupquote{(}}\sphinxstyleliteralemphasis{\sphinxupquote{str}}\sphinxstyleliteralemphasis{\sphinxupquote{)}}) \textendash{} index of TPM to power on (1\sphinxhyphen{}8) or list of indexes

\sphinxlineitem{Returns}
\sphinxAtStartPar
dictionary with HardwareCommand response. List of TPM On status

\sphinxlineitem{Return type}
\sphinxAtStartPar
dict

\end{description}\end{quote}

\end{fulllineitems}


\end{fulllineitems}

\index{PowerOnTpmCommand (class in subrack\_hardware)@\spxentry{PowerOnTpmCommand}\spxextra{class in subrack\_hardware}}

\begin{fulllineitems}
\phantomsection\label{\detokenize{webserverdocs:subrack_hardware.PowerOnTpmCommand}}
\pysigstartsignatures
\pysiglinewithargsret{\sphinxbfcode{\sphinxupquote{class\DUrole{w}{ }}}\sphinxcode{\sphinxupquote{subrack\_hardware.}}\sphinxbfcode{\sphinxupquote{PowerOnTpmCommand}}}{\sphinxparam{\DUrole{n}{name}}\sphinxparamcomma \sphinxparam{\DUrole{n}{hardware}\DUrole{o}{=}\DUrole{default_value}{None}}\sphinxparamcomma \sphinxparam{\DUrole{n}{num\_params}\DUrole{o}{=}\DUrole{default_value}{0}}\sphinxparamcomma \sphinxparam{\DUrole{n}{blocking}\DUrole{o}{=}\DUrole{default_value}{False}}}{}
\pysigstopsignatures
\sphinxAtStartPar
Power On TPM command. Switches on a single or multiple TPM
\index{thread() (subrack\_hardware.PowerOnTpmCommand method)@\spxentry{thread()}\spxextra{subrack\_hardware.PowerOnTpmCommand method}}

\begin{fulllineitems}
\phantomsection\label{\detokenize{webserverdocs:subrack_hardware.PowerOnTpmCommand.thread}}
\pysigstartsignatures
\pysiglinewithargsret{\sphinxbfcode{\sphinxupquote{thread}}}{\sphinxparam{\DUrole{n}{tpm\_id}}}{}
\pysigstopsignatures
\sphinxAtStartPar
Power on TPMs
\begin{quote}\begin{description}
\sphinxlineitem{Parameters}
\sphinxAtStartPar
\sphinxstyleliteralstrong{\sphinxupquote{tpm\_id}} (\sphinxstyleliteralemphasis{\sphinxupquote{str}}\sphinxstyleliteralemphasis{\sphinxupquote{, }}\sphinxstyleliteralemphasis{\sphinxupquote{list}}\sphinxstyleliteralemphasis{\sphinxupquote{(}}\sphinxstyleliteralemphasis{\sphinxupquote{str}}\sphinxstyleliteralemphasis{\sphinxupquote{)}}) \textendash{} index of TPM to power on (1\sphinxhyphen{}8) or list of indexes

\sphinxlineitem{Returns}
\sphinxAtStartPar
dictionary with HardwareCommand response. List of TPM On status

\sphinxlineitem{Return type}
\sphinxAtStartPar
dict

\end{description}\end{quote}

\end{fulllineitems}


\end{fulllineitems}

\index{PowerUpCommand (class in subrack\_hardware)@\spxentry{PowerUpCommand}\spxextra{class in subrack\_hardware}}

\begin{fulllineitems}
\phantomsection\label{\detokenize{webserverdocs:subrack_hardware.PowerUpCommand}}
\pysigstartsignatures
\pysiglinewithargsret{\sphinxbfcode{\sphinxupquote{class\DUrole{w}{ }}}\sphinxcode{\sphinxupquote{subrack\_hardware.}}\sphinxbfcode{\sphinxupquote{PowerUpCommand}}}{\sphinxparam{\DUrole{n}{name}}\sphinxparamcomma \sphinxparam{\DUrole{n}{hardware}\DUrole{o}{=}\DUrole{default_value}{None}}\sphinxparamcomma \sphinxparam{\DUrole{n}{num\_params}\DUrole{o}{=}\DUrole{default_value}{0}}\sphinxparamcomma \sphinxparam{\DUrole{n}{blocking}\DUrole{o}{=}\DUrole{default_value}{False}}}{}
\pysigstopsignatures
\sphinxAtStartPar
Power on all TPMs
\index{thread() (subrack\_hardware.PowerUpCommand method)@\spxentry{thread()}\spxextra{subrack\_hardware.PowerUpCommand method}}

\begin{fulllineitems}
\phantomsection\label{\detokenize{webserverdocs:subrack_hardware.PowerUpCommand.thread}}
\pysigstartsignatures
\pysiglinewithargsret{\sphinxbfcode{\sphinxupquote{thread}}}{\sphinxparam{\DUrole{n}{params}}}{}
\pysigstopsignatures
\sphinxAtStartPar
Power on all TPMs
\begin{quote}\begin{description}
\sphinxlineitem{Parameters}
\sphinxAtStartPar
\sphinxstyleliteralstrong{\sphinxupquote{params}} \textendash{} unused

\end{description}\end{quote}

\end{fulllineitems}


\end{fulllineitems}

\index{SetFanMode (class in subrack\_hardware)@\spxentry{SetFanMode}\spxextra{class in subrack\_hardware}}

\begin{fulllineitems}
\phantomsection\label{\detokenize{webserverdocs:subrack_hardware.SetFanMode}}
\pysigstartsignatures
\pysiglinewithargsret{\sphinxbfcode{\sphinxupquote{class\DUrole{w}{ }}}\sphinxcode{\sphinxupquote{subrack\_hardware.}}\sphinxbfcode{\sphinxupquote{SetFanMode}}}{\sphinxparam{\DUrole{n}{name}}\sphinxparamcomma \sphinxparam{\DUrole{n}{hardware}\DUrole{o}{=}\DUrole{default_value}{None}}\sphinxparamcomma \sphinxparam{\DUrole{n}{num\_params}\DUrole{o}{=}\DUrole{default_value}{0}}}{}
\pysigstopsignatures
\sphinxAtStartPar
Set fan mode (manual, auto)
\index{do() (subrack\_hardware.SetFanMode method)@\spxentry{do()}\spxextra{subrack\_hardware.SetFanMode method}}

\begin{fulllineitems}
\phantomsection\label{\detokenize{webserverdocs:subrack_hardware.SetFanMode.do}}
\pysigstartsignatures
\pysiglinewithargsret{\sphinxbfcode{\sphinxupquote{do}}}{\sphinxparam{\DUrole{n}{params}}}{}
\pysigstopsignatures\begin{quote}\begin{description}
\sphinxlineitem{Parameters}
\sphinxAtStartPar
\sphinxstyleliteralstrong{\sphinxupquote{params}} \textendash{} {[}0{]}: Fan ID (in range 1\sphinxhyphen{}4), {[}1{]}: mode {[}MANUAL|AUTO{]}

\sphinxlineitem{Returns}
\sphinxAtStartPar
dictionary with HardwareCommand response.

\sphinxlineitem{Return type}
\sphinxAtStartPar
dict

\end{description}\end{quote}

\end{fulllineitems}


\end{fulllineitems}

\index{SetFanSpeed (class in subrack\_hardware)@\spxentry{SetFanSpeed}\spxextra{class in subrack\_hardware}}

\begin{fulllineitems}
\phantomsection\label{\detokenize{webserverdocs:subrack_hardware.SetFanSpeed}}
\pysigstartsignatures
\pysiglinewithargsret{\sphinxbfcode{\sphinxupquote{class\DUrole{w}{ }}}\sphinxcode{\sphinxupquote{subrack\_hardware.}}\sphinxbfcode{\sphinxupquote{SetFanSpeed}}}{\sphinxparam{\DUrole{n}{name}}\sphinxparamcomma \sphinxparam{\DUrole{n}{hardware}\DUrole{o}{=}\DUrole{default_value}{None}}\sphinxparamcomma \sphinxparam{\DUrole{n}{num\_params}\DUrole{o}{=}\DUrole{default_value}{0}}}{}
\pysigstopsignatures
\sphinxAtStartPar
Set cabinet fan speed (0\sphinxhyphen{}100)
\index{do() (subrack\_hardware.SetFanSpeed method)@\spxentry{do()}\spxextra{subrack\_hardware.SetFanSpeed method}}

\begin{fulllineitems}
\phantomsection\label{\detokenize{webserverdocs:subrack_hardware.SetFanSpeed.do}}
\pysigstartsignatures
\pysiglinewithargsret{\sphinxbfcode{\sphinxupquote{do}}}{\sphinxparam{\DUrole{n}{params}}}{}
\pysigstopsignatures\begin{quote}\begin{description}
\sphinxlineitem{Parameters}
\sphinxAtStartPar
\sphinxstyleliteralstrong{\sphinxupquote{params}} \textendash{} {[}0{]}: Fan ID (in range 1\sphinxhyphen{}4), {[}1{]}: speed (percentage)

\sphinxlineitem{Returns}
\sphinxAtStartPar
dictionary with HardwareCommand response. Retval is actual speed

\sphinxlineitem{Return type}
\sphinxAtStartPar
dict

\end{description}\end{quote}

\end{fulllineitems}


\end{fulllineitems}

\index{SetPSFanSpeed (class in subrack\_hardware)@\spxentry{SetPSFanSpeed}\spxextra{class in subrack\_hardware}}

\begin{fulllineitems}
\phantomsection\label{\detokenize{webserverdocs:subrack_hardware.SetPSFanSpeed}}
\pysigstartsignatures
\pysiglinewithargsret{\sphinxbfcode{\sphinxupquote{class\DUrole{w}{ }}}\sphinxcode{\sphinxupquote{subrack\_hardware.}}\sphinxbfcode{\sphinxupquote{SetPSFanSpeed}}}{\sphinxparam{\DUrole{n}{name}}\sphinxparamcomma \sphinxparam{\DUrole{n}{hardware}\DUrole{o}{=}\DUrole{default_value}{None}}\sphinxparamcomma \sphinxparam{\DUrole{n}{num\_params}\DUrole{o}{=}\DUrole{default_value}{0}}}{}
\pysigstopsignatures
\sphinxAtStartPar
Set Power Supply fan speed (0\sphinxhyphen{}100)
\index{do() (subrack\_hardware.SetPSFanSpeed method)@\spxentry{do()}\spxextra{subrack\_hardware.SetPSFanSpeed method}}

\begin{fulllineitems}
\phantomsection\label{\detokenize{webserverdocs:subrack_hardware.SetPSFanSpeed.do}}
\pysigstartsignatures
\pysiglinewithargsret{\sphinxbfcode{\sphinxupquote{do}}}{\sphinxparam{\DUrole{n}{params}}}{}
\pysigstopsignatures\begin{quote}\begin{description}
\sphinxlineitem{Parameters}
\sphinxAtStartPar
\sphinxstyleliteralstrong{\sphinxupquote{params}} \textendash{} {[}0{]}: Fan ID (in range 1\sphinxhyphen{}2), {[}1{]}: speed (percentage)

\sphinxlineitem{Returns}
\sphinxAtStartPar
dictionary with HardwareCommand response. Retval is actual speed

\sphinxlineitem{Return type}
\sphinxAtStartPar
dict

\end{description}\end{quote}

\end{fulllineitems}


\end{fulllineitems}

\index{SubrackHardware (class in subrack\_hardware)@\spxentry{SubrackHardware}\spxextra{class in subrack\_hardware}}

\begin{fulllineitems}
\phantomsection\label{\detokenize{webserverdocs:subrack_hardware.SubrackHardware}}
\pysigstartsignatures
\pysigline{\sphinxbfcode{\sphinxupquote{class\DUrole{w}{ }}}\sphinxcode{\sphinxupquote{subrack\_hardware.}}\sphinxbfcode{\sphinxupquote{SubrackHardware}}}
\pysigstopsignatures
\sphinxAtStartPar
Hardware device class representing the Subrack.

\sphinxAtStartPar
Initializes the SubrackMngBoard object and adds commands and attributes related to the Subrack.

\sphinxAtStartPar
Attributes:
\sphinxhyphen{} subrack: Reference to the SubrackMngBoard hardware object.

\sphinxAtStartPar
Methods:
\sphinxhyphen{} initialize(emulation=False): Initializes the SubrackMngBoard object.
\sphinxhyphen{} execute\_command(command, params=None): Executes the specified command.
\index{execute\_command() (subrack\_hardware.SubrackHardware method)@\spxentry{execute\_command()}\spxextra{subrack\_hardware.SubrackHardware method}}

\begin{fulllineitems}
\phantomsection\label{\detokenize{webserverdocs:subrack_hardware.SubrackHardware.execute_command}}
\pysigstartsignatures
\pysiglinewithargsret{\sphinxbfcode{\sphinxupquote{execute\_command}}}{\sphinxparam{\DUrole{n}{command}}\sphinxparamcomma \sphinxparam{\DUrole{n}{params}\DUrole{o}{=}\DUrole{default_value}{None}}}{}
\pysigstopsignatures
\sphinxAtStartPar
Executes the specified command.

\sphinxAtStartPar
Args:
\sphinxhyphen{} command (str): The command to be executed.
\sphinxhyphen{} params (dict): Optional parameters for the command.

\sphinxAtStartPar
Returns:
dict: Dictionary containing the command execution result.

\end{fulllineitems}

\index{initialize() (subrack\_hardware.SubrackHardware method)@\spxentry{initialize()}\spxextra{subrack\_hardware.SubrackHardware method}}

\begin{fulllineitems}
\phantomsection\label{\detokenize{webserverdocs:subrack_hardware.SubrackHardware.initialize}}
\pysigstartsignatures
\pysiglinewithargsret{\sphinxbfcode{\sphinxupquote{initialize}}}{\sphinxparam{\DUrole{n}{emulation}\DUrole{o}{=}\DUrole{default_value}{False}}}{}
\pysigstopsignatures
\sphinxAtStartPar
Initializes the SubrackMngBoard object.

\sphinxAtStartPar
Args:
\sphinxhyphen{} emulation (bool): Indicates whether the Subrack is in emulation mode.

\sphinxAtStartPar
Returns:
None

\end{fulllineitems}


\end{fulllineitems}

\index{Subrack\_PLL\_Locked (class in subrack\_hardware)@\spxentry{Subrack\_PLL\_Locked}\spxextra{class in subrack\_hardware}}

\begin{fulllineitems}
\phantomsection\label{\detokenize{webserverdocs:subrack_hardware.Subrack_PLL_Locked}}
\pysigstartsignatures
\pysiglinewithargsret{\sphinxbfcode{\sphinxupquote{class\DUrole{w}{ }}}\sphinxcode{\sphinxupquote{subrack\_hardware.}}\sphinxbfcode{\sphinxupquote{Subrack\_PLL\_Locked}}}{\sphinxparam{\DUrole{n}{name}}\sphinxparamcomma \sphinxparam{\DUrole{n}{init\_value}}\sphinxparamcomma \sphinxparam{\DUrole{n}{hardware}\DUrole{o}{=}\DUrole{default_value}{None}}\sphinxparamcomma \sphinxparam{\DUrole{n}{read\_write}\DUrole{o}{=}\DUrole{default_value}{0}}\sphinxparamcomma \sphinxparam{\DUrole{n}{num\_params}\DUrole{o}{=}\DUrole{default_value}{1}}}{}
\pysigstopsignatures
\sphinxAtStartPar
Subrack PLL Lock status
Returns status of Subrack PLL lock
\index{read\_value() (subrack\_hardware.Subrack\_PLL\_Locked method)@\spxentry{read\_value()}\spxextra{subrack\_hardware.Subrack\_PLL\_Locked method}}

\begin{fulllineitems}
\phantomsection\label{\detokenize{webserverdocs:subrack_hardware.Subrack_PLL_Locked.read_value}}
\pysigstartsignatures
\pysiglinewithargsret{\sphinxbfcode{\sphinxupquote{read\_value}}}{}{}
\pysigstopsignatures
\sphinxAtStartPar
Reads the actual value
To be overrided in the subclass
\begin{quote}\begin{description}
\sphinxlineitem{Returns}
\sphinxAtStartPar
Attribute true value, from real device

\end{description}\end{quote}

\end{fulllineitems}


\end{fulllineitems}

\index{Subrack\_Timestamp (class in subrack\_hardware)@\spxentry{Subrack\_Timestamp}\spxextra{class in subrack\_hardware}}

\begin{fulllineitems}
\phantomsection\label{\detokenize{webserverdocs:subrack_hardware.Subrack_Timestamp}}
\pysigstartsignatures
\pysiglinewithargsret{\sphinxbfcode{\sphinxupquote{class\DUrole{w}{ }}}\sphinxcode{\sphinxupquote{subrack\_hardware.}}\sphinxbfcode{\sphinxupquote{Subrack\_Timestamp}}}{\sphinxparam{\DUrole{n}{name}}\sphinxparamcomma \sphinxparam{\DUrole{n}{init\_value}}\sphinxparamcomma \sphinxparam{\DUrole{n}{hardware}\DUrole{o}{=}\DUrole{default_value}{None}}\sphinxparamcomma \sphinxparam{\DUrole{n}{read\_write}\DUrole{o}{=}\DUrole{default_value}{0}}\sphinxparamcomma \sphinxparam{\DUrole{n}{num\_params}\DUrole{o}{=}\DUrole{default_value}{1}}}{}
\pysigstopsignatures
\sphinxAtStartPar
Subrack Time in timestamp
Returns Time of Subrack in timrstamp format
\index{read\_value() (subrack\_hardware.Subrack\_Timestamp method)@\spxentry{read\_value()}\spxextra{subrack\_hardware.Subrack\_Timestamp method}}

\begin{fulllineitems}
\phantomsection\label{\detokenize{webserverdocs:subrack_hardware.Subrack_Timestamp.read_value}}
\pysigstartsignatures
\pysiglinewithargsret{\sphinxbfcode{\sphinxupquote{read\_value}}}{}{}
\pysigstopsignatures
\sphinxAtStartPar
Reads the actual value
To be overrided in the subclass
\begin{quote}\begin{description}
\sphinxlineitem{Returns}
\sphinxAtStartPar
Attribute true value, from real device

\end{description}\end{quote}

\end{fulllineitems}


\end{fulllineitems}

\index{TPM\_Add\_List (class in subrack\_hardware)@\spxentry{TPM\_Add\_List}\spxextra{class in subrack\_hardware}}

\begin{fulllineitems}
\phantomsection\label{\detokenize{webserverdocs:subrack_hardware.TPM_Add_List}}
\pysigstartsignatures
\pysiglinewithargsret{\sphinxbfcode{\sphinxupquote{class\DUrole{w}{ }}}\sphinxcode{\sphinxupquote{subrack\_hardware.}}\sphinxbfcode{\sphinxupquote{TPM\_Add\_List}}}{\sphinxparam{\DUrole{n}{name}}\sphinxparamcomma \sphinxparam{\DUrole{n}{init\_value}}\sphinxparamcomma \sphinxparam{\DUrole{n}{hardware}\DUrole{o}{=}\DUrole{default_value}{None}}\sphinxparamcomma \sphinxparam{\DUrole{n}{read\_write}\DUrole{o}{=}\DUrole{default_value}{0}}\sphinxparamcomma \sphinxparam{\DUrole{n}{num\_params}\DUrole{o}{=}\DUrole{default_value}{1}}}{}
\pysigstopsignatures
\sphinxAtStartPar
TPM IP Address Will Be Assigned.
Returns 8 IP address, address will be assigned to each TPM in subrack slots
\index{read\_value() (subrack\_hardware.TPM\_Add\_List method)@\spxentry{read\_value()}\spxextra{subrack\_hardware.TPM\_Add\_List method}}

\begin{fulllineitems}
\phantomsection\label{\detokenize{webserverdocs:subrack_hardware.TPM_Add_List.read_value}}
\pysigstartsignatures
\pysiglinewithargsret{\sphinxbfcode{\sphinxupquote{read\_value}}}{}{}
\pysigstopsignatures
\sphinxAtStartPar
Reads the actual value
To be overrided in the subclass
\begin{quote}\begin{description}
\sphinxlineitem{Returns}
\sphinxAtStartPar
Attribute true value, from real device

\end{description}\end{quote}

\end{fulllineitems}


\end{fulllineitems}

\index{TPM\_Temperature\_Alarms (class in subrack\_hardware)@\spxentry{TPM\_Temperature\_Alarms}\spxextra{class in subrack\_hardware}}

\begin{fulllineitems}
\phantomsection\label{\detokenize{webserverdocs:subrack_hardware.TPM_Temperature_Alarms}}
\pysigstartsignatures
\pysiglinewithargsret{\sphinxbfcode{\sphinxupquote{class\DUrole{w}{ }}}\sphinxcode{\sphinxupquote{subrack\_hardware.}}\sphinxbfcode{\sphinxupquote{TPM\_Temperature\_Alarms}}}{\sphinxparam{\DUrole{n}{name}}\sphinxparamcomma \sphinxparam{\DUrole{n}{init\_value}}\sphinxparamcomma \sphinxparam{\DUrole{n}{hardware}\DUrole{o}{=}\DUrole{default_value}{None}}\sphinxparamcomma \sphinxparam{\DUrole{n}{read\_write}\DUrole{o}{=}\DUrole{default_value}{0}}\sphinxparamcomma \sphinxparam{\DUrole{n}{num\_params}\DUrole{o}{=}\DUrole{default_value}{1}}}{}
\pysigstopsignatures
\sphinxAtStartPar
TPMs Temperature alarm status
Returns 8 temeprature alarms
\index{read\_value() (subrack\_hardware.TPM\_Temperature\_Alarms method)@\spxentry{read\_value()}\spxextra{subrack\_hardware.TPM\_Temperature\_Alarms method}}

\begin{fulllineitems}
\phantomsection\label{\detokenize{webserverdocs:subrack_hardware.TPM_Temperature_Alarms.read_value}}
\pysigstartsignatures
\pysiglinewithargsret{\sphinxbfcode{\sphinxupquote{read\_value}}}{}{}
\pysigstopsignatures
\sphinxAtStartPar
Reads the actual value
To be overrided in the subclass
\begin{quote}\begin{description}
\sphinxlineitem{Returns}
\sphinxAtStartPar
Attribute true value, from real device

\end{description}\end{quote}

\end{fulllineitems}


\end{fulllineitems}

\index{TPM\_Temperatures (class in subrack\_hardware)@\spxentry{TPM\_Temperatures}\spxextra{class in subrack\_hardware}}

\begin{fulllineitems}
\phantomsection\label{\detokenize{webserverdocs:subrack_hardware.TPM_Temperatures}}
\pysigstartsignatures
\pysiglinewithargsret{\sphinxbfcode{\sphinxupquote{class\DUrole{w}{ }}}\sphinxcode{\sphinxupquote{subrack\_hardware.}}\sphinxbfcode{\sphinxupquote{TPM\_Temperatures}}}{\sphinxparam{\DUrole{n}{name}}\sphinxparamcomma \sphinxparam{\DUrole{n}{init\_value}}\sphinxparamcomma \sphinxparam{\DUrole{n}{hardware}\DUrole{o}{=}\DUrole{default_value}{None}}\sphinxparamcomma \sphinxparam{\DUrole{n}{read\_write}\DUrole{o}{=}\DUrole{default_value}{0}}\sphinxparamcomma \sphinxparam{\DUrole{n}{num\_params}\DUrole{o}{=}\DUrole{default_value}{1}}}{}
\pysigstopsignatures
\sphinxAtStartPar
TPMs Temperature vectors
Returns 8 TPMBoard temepratures , 8 TPM FPGA1 temperatures, 8 TPM FPGA2 temperatures
\index{read\_value() (subrack\_hardware.TPM\_Temperatures method)@\spxentry{read\_value()}\spxextra{subrack\_hardware.TPM\_Temperatures method}}

\begin{fulllineitems}
\phantomsection\label{\detokenize{webserverdocs:subrack_hardware.TPM_Temperatures.read_value}}
\pysigstartsignatures
\pysiglinewithargsret{\sphinxbfcode{\sphinxupquote{read\_value}}}{}{}
\pysigstopsignatures
\sphinxAtStartPar
Reads the actual value
To be overrided in the subclass
\begin{quote}\begin{description}
\sphinxlineitem{Returns}
\sphinxAtStartPar
Attribute true value, from real device

\end{description}\end{quote}

\end{fulllineitems}


\end{fulllineitems}

\index{TPM\_Voltage\_Alarms (class in subrack\_hardware)@\spxentry{TPM\_Voltage\_Alarms}\spxextra{class in subrack\_hardware}}

\begin{fulllineitems}
\phantomsection\label{\detokenize{webserverdocs:subrack_hardware.TPM_Voltage_Alarms}}
\pysigstartsignatures
\pysiglinewithargsret{\sphinxbfcode{\sphinxupquote{class\DUrole{w}{ }}}\sphinxcode{\sphinxupquote{subrack\_hardware.}}\sphinxbfcode{\sphinxupquote{TPM\_Voltage\_Alarms}}}{\sphinxparam{\DUrole{n}{name}}\sphinxparamcomma \sphinxparam{\DUrole{n}{init\_value}}\sphinxparamcomma \sphinxparam{\DUrole{n}{hardware}\DUrole{o}{=}\DUrole{default_value}{None}}\sphinxparamcomma \sphinxparam{\DUrole{n}{read\_write}\DUrole{o}{=}\DUrole{default_value}{0}}\sphinxparamcomma \sphinxparam{\DUrole{n}{num\_params}\DUrole{o}{=}\DUrole{default_value}{1}}}{}
\pysigstopsignatures
\sphinxAtStartPar
TPMs Voltage alarm status
Returns 8 voltage alarms
\index{read\_value() (subrack\_hardware.TPM\_Voltage\_Alarms method)@\spxentry{read\_value()}\spxextra{subrack\_hardware.TPM\_Voltage\_Alarms method}}

\begin{fulllineitems}
\phantomsection\label{\detokenize{webserverdocs:subrack_hardware.TPM_Voltage_Alarms.read_value}}
\pysigstartsignatures
\pysiglinewithargsret{\sphinxbfcode{\sphinxupquote{read\_value}}}{}{}
\pysigstopsignatures
\sphinxAtStartPar
Reads the actual value
To be overrided in the subclass
\begin{quote}\begin{description}
\sphinxlineitem{Returns}
\sphinxAtStartPar
Attribute true value, from real device

\end{description}\end{quote}

\end{fulllineitems}


\end{fulllineitems}

\index{TpmCurrents (class in subrack\_hardware)@\spxentry{TpmCurrents}\spxextra{class in subrack\_hardware}}

\begin{fulllineitems}
\phantomsection\label{\detokenize{webserverdocs:subrack_hardware.TpmCurrents}}
\pysigstartsignatures
\pysiglinewithargsret{\sphinxbfcode{\sphinxupquote{class\DUrole{w}{ }}}\sphinxcode{\sphinxupquote{subrack\_hardware.}}\sphinxbfcode{\sphinxupquote{TpmCurrents}}}{\sphinxparam{\DUrole{n}{name}}\sphinxparamcomma \sphinxparam{\DUrole{n}{init\_value}}\sphinxparamcomma \sphinxparam{\DUrole{n}{hardware}\DUrole{o}{=}\DUrole{default_value}{None}}\sphinxparamcomma \sphinxparam{\DUrole{n}{read\_write}\DUrole{o}{=}\DUrole{default_value}{0}}\sphinxparamcomma \sphinxparam{\DUrole{n}{num\_params}\DUrole{o}{=}\DUrole{default_value}{1}}}{}
\pysigstopsignatures
\sphinxAtStartPar
TPM board current, in A.
Returns 8 values, 0.0 for boards not present
\index{read\_value() (subrack\_hardware.TpmCurrents method)@\spxentry{read\_value()}\spxextra{subrack\_hardware.TpmCurrents method}}

\begin{fulllineitems}
\phantomsection\label{\detokenize{webserverdocs:subrack_hardware.TpmCurrents.read_value}}
\pysigstartsignatures
\pysiglinewithargsret{\sphinxbfcode{\sphinxupquote{read\_value}}}{}{}
\pysigstopsignatures\begin{quote}\begin{description}
\sphinxlineitem{Returns}
\sphinxAtStartPar
8 values, 0.0 for boards not present

\sphinxlineitem{Rvalue}
\sphinxAtStartPar
list(float)

\end{description}\end{quote}

\end{fulllineitems}


\end{fulllineitems}

\index{TpmIPs (class in subrack\_hardware)@\spxentry{TpmIPs}\spxextra{class in subrack\_hardware}}

\begin{fulllineitems}
\phantomsection\label{\detokenize{webserverdocs:subrack_hardware.TpmIPs}}
\pysigstartsignatures
\pysiglinewithargsret{\sphinxbfcode{\sphinxupquote{class\DUrole{w}{ }}}\sphinxcode{\sphinxupquote{subrack\_hardware.}}\sphinxbfcode{\sphinxupquote{TpmIPs}}}{\sphinxparam{\DUrole{n}{name}}\sphinxparamcomma \sphinxparam{\DUrole{n}{init\_value}}\sphinxparamcomma \sphinxparam{\DUrole{n}{hardware}\DUrole{o}{=}\DUrole{default_value}{None}}\sphinxparamcomma \sphinxparam{\DUrole{n}{read\_write}\DUrole{o}{=}\DUrole{default_value}{0}}\sphinxparamcomma \sphinxparam{\DUrole{n}{num\_params}\DUrole{o}{=}\DUrole{default_value}{1}}}{}
\pysigstopsignatures
\sphinxAtStartPar
IP address of TPMs present on subrack.
Returns 8 IP address, “0” for TPMs not powered off and “\sphinxhyphen{}1” for TPMs not present
\index{read\_value() (subrack\_hardware.TpmIPs method)@\spxentry{read\_value()}\spxextra{subrack\_hardware.TpmIPs method}}

\begin{fulllineitems}
\phantomsection\label{\detokenize{webserverdocs:subrack_hardware.TpmIPs.read_value}}
\pysigstartsignatures
\pysiglinewithargsret{\sphinxbfcode{\sphinxupquote{read\_value}}}{}{}
\pysigstopsignatures\begin{quote}\begin{description}
\sphinxlineitem{Returns}
\sphinxAtStartPar
TPM IP

\sphinxlineitem{Return type}
\sphinxAtStartPar
list{[}str{]}

\end{description}\end{quote}

\end{fulllineitems}


\end{fulllineitems}

\index{TpmInfo (class in subrack\_hardware)@\spxentry{TpmInfo}\spxextra{class in subrack\_hardware}}

\begin{fulllineitems}
\phantomsection\label{\detokenize{webserverdocs:subrack_hardware.TpmInfo}}
\pysigstartsignatures
\pysiglinewithargsret{\sphinxbfcode{\sphinxupquote{class\DUrole{w}{ }}}\sphinxcode{\sphinxupquote{subrack\_hardware.}}\sphinxbfcode{\sphinxupquote{TpmInfo}}}{\sphinxparam{\DUrole{n}{name}}\sphinxparamcomma \sphinxparam{\DUrole{n}{hardware}\DUrole{o}{=}\DUrole{default_value}{None}}\sphinxparamcomma \sphinxparam{\DUrole{n}{num\_params}\DUrole{o}{=}\DUrole{default_value}{0}}}{}
\pysigstopsignatures
\sphinxAtStartPar
Return info about TPM board present on subrack.
\index{do() (subrack\_hardware.TpmInfo method)@\spxentry{do()}\spxextra{subrack\_hardware.TpmInfo method}}

\begin{fulllineitems}
\phantomsection\label{\detokenize{webserverdocs:subrack_hardware.TpmInfo.do}}
\pysigstartsignatures
\pysiglinewithargsret{\sphinxbfcode{\sphinxupquote{do}}}{\sphinxparam{\DUrole{n}{params}}}{}
\pysigstopsignatures
\sphinxAtStartPar
Info about TPM.
\begin{quote}\begin{description}
\sphinxlineitem{Parameters}
\sphinxAtStartPar
\sphinxstyleliteralstrong{\sphinxupquote{params}} (\sphinxstyleliteralemphasis{\sphinxupquote{str}}) \textendash{} index of TPM to check (1\sphinxhyphen{}8)

\sphinxlineitem{Returns}
\sphinxAtStartPar
TPM info

\sphinxlineitem{Return type}
\sphinxAtStartPar
dict

\end{description}\end{quote}

\end{fulllineitems}


\end{fulllineitems}

\index{TpmMCUTemperatures (class in subrack\_hardware)@\spxentry{TpmMCUTemperatures}\spxextra{class in subrack\_hardware}}

\begin{fulllineitems}
\phantomsection\label{\detokenize{webserverdocs:subrack_hardware.TpmMCUTemperatures}}
\pysigstartsignatures
\pysiglinewithargsret{\sphinxbfcode{\sphinxupquote{class\DUrole{w}{ }}}\sphinxcode{\sphinxupquote{subrack\_hardware.}}\sphinxbfcode{\sphinxupquote{TpmMCUTemperatures}}}{\sphinxparam{\DUrole{n}{name}}\sphinxparamcomma \sphinxparam{\DUrole{n}{init\_value}}\sphinxparamcomma \sphinxparam{\DUrole{n}{hardware}\DUrole{o}{=}\DUrole{default_value}{None}}\sphinxparamcomma \sphinxparam{\DUrole{n}{read\_write}\DUrole{o}{=}\DUrole{default_value}{0}}\sphinxparamcomma \sphinxparam{\DUrole{n}{num\_params}\DUrole{o}{=}\DUrole{default_value}{1}}}{}
\pysigstopsignatures
\sphinxAtStartPar
TPM MCU temperatures, in celsius.
Returns 8 values, 0.0 for boards not present or powered off
\index{read\_value() (subrack\_hardware.TpmMCUTemperatures method)@\spxentry{read\_value()}\spxextra{subrack\_hardware.TpmMCUTemperatures method}}

\begin{fulllineitems}
\phantomsection\label{\detokenize{webserverdocs:subrack_hardware.TpmMCUTemperatures.read_value}}
\pysigstartsignatures
\pysiglinewithargsret{\sphinxbfcode{\sphinxupquote{read\_value}}}{}{}
\pysigstopsignatures\begin{quote}\begin{description}
\sphinxlineitem{Returns}
\sphinxAtStartPar
TPM MCU temperature, in Celsius

\sphinxlineitem{Return type}
\sphinxAtStartPar
list{[}float{]}

\end{description}\end{quote}

\end{fulllineitems}


\end{fulllineitems}

\index{TpmOnOffVect (class in subrack\_hardware)@\spxentry{TpmOnOffVect}\spxextra{class in subrack\_hardware}}

\begin{fulllineitems}
\phantomsection\label{\detokenize{webserverdocs:subrack_hardware.TpmOnOffVect}}
\pysigstartsignatures
\pysiglinewithargsret{\sphinxbfcode{\sphinxupquote{class\DUrole{w}{ }}}\sphinxcode{\sphinxupquote{subrack\_hardware.}}\sphinxbfcode{\sphinxupquote{TpmOnOffVect}}}{\sphinxparam{\DUrole{n}{name}}\sphinxparamcomma \sphinxparam{\DUrole{n}{init\_value}}\sphinxparamcomma \sphinxparam{\DUrole{n}{hardware}\DUrole{o}{=}\DUrole{default_value}{None}}\sphinxparamcomma \sphinxparam{\DUrole{n}{read\_write}\DUrole{o}{=}\DUrole{default_value}{0}}\sphinxparamcomma \sphinxparam{\DUrole{n}{num\_params}\DUrole{o}{=}\DUrole{default_value}{1}}}{}
\pysigstopsignatures
\sphinxAtStartPar
TPM board power status
\index{read\_value() (subrack\_hardware.TpmOnOffVect method)@\spxentry{read\_value()}\spxextra{subrack\_hardware.TpmOnOffVect method}}

\begin{fulllineitems}
\phantomsection\label{\detokenize{webserverdocs:subrack_hardware.TpmOnOffVect.read_value}}
\pysigstartsignatures
\pysiglinewithargsret{\sphinxbfcode{\sphinxupquote{read\_value}}}{}{}
\pysigstopsignatures
\sphinxAtStartPar
Reads the actual value
To be overrided in the subclass
\begin{quote}\begin{description}
\sphinxlineitem{Returns}
\sphinxAtStartPar
Attribute true value, from real device

\end{description}\end{quote}

\end{fulllineitems}


\end{fulllineitems}

\index{TpmPowers (class in subrack\_hardware)@\spxentry{TpmPowers}\spxextra{class in subrack\_hardware}}

\begin{fulllineitems}
\phantomsection\label{\detokenize{webserverdocs:subrack_hardware.TpmPowers}}
\pysigstartsignatures
\pysiglinewithargsret{\sphinxbfcode{\sphinxupquote{class\DUrole{w}{ }}}\sphinxcode{\sphinxupquote{subrack\_hardware.}}\sphinxbfcode{\sphinxupquote{TpmPowers}}}{\sphinxparam{\DUrole{n}{name}}\sphinxparamcomma \sphinxparam{\DUrole{n}{init\_value}}\sphinxparamcomma \sphinxparam{\DUrole{n}{hardware}\DUrole{o}{=}\DUrole{default_value}{None}}\sphinxparamcomma \sphinxparam{\DUrole{n}{read\_write}\DUrole{o}{=}\DUrole{default_value}{0}}\sphinxparamcomma \sphinxparam{\DUrole{n}{num\_params}\DUrole{o}{=}\DUrole{default_value}{1}}}{}
\pysigstopsignatures
\sphinxAtStartPar
TPM board power usage (W)
\index{read\_value() (subrack\_hardware.TpmPowers method)@\spxentry{read\_value()}\spxextra{subrack\_hardware.TpmPowers method}}

\begin{fulllineitems}
\phantomsection\label{\detokenize{webserverdocs:subrack_hardware.TpmPowers.read_value}}
\pysigstartsignatures
\pysiglinewithargsret{\sphinxbfcode{\sphinxupquote{read\_value}}}{}{}
\pysigstopsignatures\begin{quote}\begin{description}
\sphinxlineitem{Returns}
\sphinxAtStartPar
8 values, 0.0 for boards not present

\sphinxlineitem{Rvalue}
\sphinxAtStartPar
list(float)

\end{description}\end{quote}

\end{fulllineitems}


\end{fulllineitems}

\index{TpmPresent (class in subrack\_hardware)@\spxentry{TpmPresent}\spxextra{class in subrack\_hardware}}

\begin{fulllineitems}
\phantomsection\label{\detokenize{webserverdocs:subrack_hardware.TpmPresent}}
\pysigstartsignatures
\pysiglinewithargsret{\sphinxbfcode{\sphinxupquote{class\DUrole{w}{ }}}\sphinxcode{\sphinxupquote{subrack\_hardware.}}\sphinxbfcode{\sphinxupquote{TpmPresent}}}{\sphinxparam{\DUrole{n}{name}}\sphinxparamcomma \sphinxparam{\DUrole{n}{init\_value}}\sphinxparamcomma \sphinxparam{\DUrole{n}{hardware}\DUrole{o}{=}\DUrole{default_value}{None}}\sphinxparamcomma \sphinxparam{\DUrole{n}{read\_write}\DUrole{o}{=}\DUrole{default_value}{0}}\sphinxparamcomma \sphinxparam{\DUrole{n}{num\_params}\DUrole{o}{=}\DUrole{default_value}{1}}}{}
\pysigstopsignatures
\sphinxAtStartPar
TPM board presence
Returns 8 values, True if board present and powered on
\index{read\_value() (subrack\_hardware.TpmPresent method)@\spxentry{read\_value()}\spxextra{subrack\_hardware.TpmPresent method}}

\begin{fulllineitems}
\phantomsection\label{\detokenize{webserverdocs:subrack_hardware.TpmPresent.read_value}}
\pysigstartsignatures
\pysiglinewithargsret{\sphinxbfcode{\sphinxupquote{read\_value}}}{}{}
\pysigstopsignatures
\sphinxAtStartPar
Reads the actual value
To be overrided in the subclass
\begin{quote}\begin{description}
\sphinxlineitem{Returns}
\sphinxAtStartPar
Attribute true value, from real device

\end{description}\end{quote}

\end{fulllineitems}


\end{fulllineitems}

\index{TpmSupplyFault (class in subrack\_hardware)@\spxentry{TpmSupplyFault}\spxextra{class in subrack\_hardware}}

\begin{fulllineitems}
\phantomsection\label{\detokenize{webserverdocs:subrack_hardware.TpmSupplyFault}}
\pysigstartsignatures
\pysiglinewithargsret{\sphinxbfcode{\sphinxupquote{class\DUrole{w}{ }}}\sphinxcode{\sphinxupquote{subrack\_hardware.}}\sphinxbfcode{\sphinxupquote{TpmSupplyFault}}}{\sphinxparam{\DUrole{n}{name}}\sphinxparamcomma \sphinxparam{\DUrole{n}{init\_value}}\sphinxparamcomma \sphinxparam{\DUrole{n}{hardware}\DUrole{o}{=}\DUrole{default_value}{None}}\sphinxparamcomma \sphinxparam{\DUrole{n}{read\_write}\DUrole{o}{=}\DUrole{default_value}{0}}\sphinxparamcomma \sphinxparam{\DUrole{n}{num\_params}\DUrole{o}{=}\DUrole{default_value}{1}}}{}
\pysigstopsignatures
\sphinxAtStartPar
TPM supply fault status
Returns 8 bool values, True if board supply fault has been triggered
\index{read\_value() (subrack\_hardware.TpmSupplyFault method)@\spxentry{read\_value()}\spxextra{subrack\_hardware.TpmSupplyFault method}}

\begin{fulllineitems}
\phantomsection\label{\detokenize{webserverdocs:subrack_hardware.TpmSupplyFault.read_value}}
\pysigstartsignatures
\pysiglinewithargsret{\sphinxbfcode{\sphinxupquote{read\_value}}}{}{}
\pysigstopsignatures
\sphinxAtStartPar
Reads the actual value
To be overrided in the subclass
\begin{quote}\begin{description}
\sphinxlineitem{Returns}
\sphinxAtStartPar
Attribute true value, from real device

\end{description}\end{quote}

\end{fulllineitems}


\end{fulllineitems}

\index{TpmVoltages (class in subrack\_hardware)@\spxentry{TpmVoltages}\spxextra{class in subrack\_hardware}}

\begin{fulllineitems}
\phantomsection\label{\detokenize{webserverdocs:subrack_hardware.TpmVoltages}}
\pysigstartsignatures
\pysiglinewithargsret{\sphinxbfcode{\sphinxupquote{class\DUrole{w}{ }}}\sphinxcode{\sphinxupquote{subrack\_hardware.}}\sphinxbfcode{\sphinxupquote{TpmVoltages}}}{\sphinxparam{\DUrole{n}{name}}\sphinxparamcomma \sphinxparam{\DUrole{n}{init\_value}}\sphinxparamcomma \sphinxparam{\DUrole{n}{hardware}\DUrole{o}{=}\DUrole{default_value}{None}}\sphinxparamcomma \sphinxparam{\DUrole{n}{read\_write}\DUrole{o}{=}\DUrole{default_value}{0}}\sphinxparamcomma \sphinxparam{\DUrole{n}{num\_params}\DUrole{o}{=}\DUrole{default_value}{1}}}{}
\pysigstopsignatures
\sphinxAtStartPar
TPM board power supply voltages (V)
\index{read\_value() (subrack\_hardware.TpmVoltages method)@\spxentry{read\_value()}\spxextra{subrack\_hardware.TpmVoltages method}}

\begin{fulllineitems}
\phantomsection\label{\detokenize{webserverdocs:subrack_hardware.TpmVoltages.read_value}}
\pysigstartsignatures
\pysiglinewithargsret{\sphinxbfcode{\sphinxupquote{read\_value}}}{}{}
\pysigstopsignatures\begin{quote}\begin{description}
\sphinxlineitem{Returns}
\sphinxAtStartPar
8 values, 0.0 for boards not present

\sphinxlineitem{Rvalue}
\sphinxAtStartPar
list(float)

\end{description}\end{quote}

\end{fulllineitems}


\end{fulllineitems}

\index{byte\_to\_bool\_array() (in module subrack\_hardware)@\spxentry{byte\_to\_bool\_array()}\spxextra{in module subrack\_hardware}}

\begin{fulllineitems}
\phantomsection\label{\detokenize{webserverdocs:subrack_hardware.byte_to_bool_array}}
\pysigstartsignatures
\pysiglinewithargsret{\sphinxcode{\sphinxupquote{subrack\_hardware.}}\sphinxbfcode{\sphinxupquote{byte\_to\_bool\_array}}}{\sphinxparam{\DUrole{n}{byte\_in}}}{}
\pysigstopsignatures
\sphinxAtStartPar
Convert a byte to a list of 8 bool. Bit 0 corresponds to list{[}0{]}
\begin{quote}\begin{description}
\sphinxlineitem{Parameters}
\sphinxAtStartPar
\sphinxstyleliteralstrong{\sphinxupquote{byte\_in}} (\sphinxstyleliteralemphasis{\sphinxupquote{int}}) \textendash{} Byte with 8 LS bits representing 8 logical values

\sphinxlineitem{Returns}
\sphinxAtStartPar
list of 8 bool values

\end{description}\end{quote}

\end{fulllineitems}

\index{ups\_status (class in subrack\_hardware)@\spxentry{ups\_status}\spxextra{class in subrack\_hardware}}

\begin{fulllineitems}
\phantomsection\label{\detokenize{webserverdocs:subrack_hardware.ups_status}}
\pysigstartsignatures
\pysiglinewithargsret{\sphinxbfcode{\sphinxupquote{class\DUrole{w}{ }}}\sphinxcode{\sphinxupquote{subrack\_hardware.}}\sphinxbfcode{\sphinxupquote{ups\_status}}}{\sphinxparam{\DUrole{n}{name}}\sphinxparamcomma \sphinxparam{\DUrole{n}{init\_value}}\sphinxparamcomma \sphinxparam{\DUrole{n}{hardware}\DUrole{o}{=}\DUrole{default_value}{None}}\sphinxparamcomma \sphinxparam{\DUrole{n}{read\_write}\DUrole{o}{=}\DUrole{default_value}{0}}\sphinxparamcomma \sphinxparam{\DUrole{n}{num\_params}\DUrole{o}{=}\DUrole{default_value}{1}}}{}
\pysigstopsignatures
\sphinxAtStartPar
UPS board status
Returns UPS Board Status: ups\_status = \{“alarm”:False,”warning”:False,”charging”:False\}
\index{read\_value() (subrack\_hardware.ups\_status method)@\spxentry{read\_value()}\spxextra{subrack\_hardware.ups\_status method}}

\begin{fulllineitems}
\phantomsection\label{\detokenize{webserverdocs:subrack_hardware.ups_status.read_value}}
\pysigstartsignatures
\pysiglinewithargsret{\sphinxbfcode{\sphinxupquote{read\_value}}}{}{}
\pysigstopsignatures
\sphinxAtStartPar
Reads the actual value
To be overrided in the subclass
\begin{quote}\begin{description}
\sphinxlineitem{Returns}
\sphinxAtStartPar
Attribute true value, from real device

\end{description}\end{quote}

\end{fulllineitems}


\end{fulllineitems}

\index{module@\spxentry{module}!web\_server@\spxentry{web\_server}}\index{web\_server@\spxentry{web\_server}!module@\spxentry{module}}\index{MyServer (class in web\_server)@\spxentry{MyServer}\spxextra{class in web\_server}}\phantomsection\label{\detokenize{webserverdocs:module-web_server}}

\begin{fulllineitems}
\phantomsection\label{\detokenize{webserverdocs:web_server.MyServer}}
\pysigstartsignatures
\pysiglinewithargsret{\sphinxbfcode{\sphinxupquote{class\DUrole{w}{ }}}\sphinxcode{\sphinxupquote{web\_server.}}\sphinxbfcode{\sphinxupquote{MyServer}}}{\sphinxparam{\DUrole{n}{request}}\sphinxparamcomma \sphinxparam{\DUrole{n}{client\_address}}\sphinxparamcomma \sphinxparam{\DUrole{n}{server}}}{}
\pysigstopsignatures
\sphinxAtStartPar
Request handler, subclassed from http package
\index{do\_GET() (web\_server.MyServer method)@\spxentry{do\_GET()}\spxextra{web\_server.MyServer method}}

\begin{fulllineitems}
\phantomsection\label{\detokenize{webserverdocs:web_server.MyServer.do_GET}}
\pysigstartsignatures
\pysiglinewithargsret{\sphinxbfcode{\sphinxupquote{do\_GET}}}{}{}
\pysigstopsignatures
\sphinxAtStartPar
Callback for GET request
Retrieves query, and splits it into a dictionary
Use mangle\_dict to convert 1\sphinxhyphen{}lists to scalars,
and numeric strings to numbers
Calls appropriate methods of the hardware class.
Formats the return dictionary to json and send it as response

\end{fulllineitems}


\end{fulllineitems}

\index{mangle\_dict() (in module web\_server)@\spxentry{mangle\_dict()}\spxextra{in module web\_server}}

\begin{fulllineitems}
\phantomsection\label{\detokenize{webserverdocs:web_server.mangle_dict}}
\pysigstartsignatures
\pysiglinewithargsret{\sphinxcode{\sphinxupquote{web\_server.}}\sphinxbfcode{\sphinxupquote{mangle\_dict}}}{\sphinxparam{\DUrole{n}{input\_dict}}}{}
\pysigstopsignatures
\sphinxAtStartPar
Takes a query dictionary from the http.getquerydict()
FOr each element keeps only the first element (list of values)
Converts it to scalar if it is a list of 1 element
If some values are numeric convert them to numbers

\end{fulllineitems}


\sphinxstepscope


\chapter{CPLD Management API Documentation}
\label{\detokenize{cplddocs:module-config_ip}}\label{\detokenize{cplddocs:cpld-management-api-documentation}}\label{\detokenize{cplddocs::doc}}\index{module@\spxentry{module}!config\_ip@\spxentry{config\_ip}}\index{config\_ip@\spxentry{config\_ip}!module@\spxentry{module}}\index{get\_mac\_from\_eep() (in module config\_ip)@\spxentry{get\_mac\_from\_eep()}\spxextra{in module config\_ip}}

\begin{fulllineitems}
\phantomsection\label{\detokenize{cplddocs:config_ip.get_mac_from_eep}}
\pysigstartsignatures
\pysiglinewithargsret{\sphinxcode{\sphinxupquote{config\_ip.}}\sphinxbfcode{\sphinxupquote{get\_mac\_from\_eep}}}{\sphinxparam{\DUrole{n}{inst}}\sphinxparamcomma \sphinxparam{\DUrole{n}{phy\_addr}\DUrole{o}{=}\DUrole{default_value}{160}}}{}
\pysigstopsignatures
\sphinxAtStartPar
Read MAC address from EEPROM.
\begin{quote}\begin{description}
\sphinxlineitem{Parameters}\begin{itemize}
\item {} 
\sphinxAtStartPar
\sphinxstyleliteralstrong{\sphinxupquote{inst}} (\sphinxstyleliteralemphasis{\sphinxupquote{instance}}) \textendash{} Instance of a class with ‘bsp’ attribute.

\item {} 
\sphinxAtStartPar
\sphinxstyleliteralstrong{\sphinxupquote{phy\_addr}} (\sphinxstyleliteralemphasis{\sphinxupquote{int}}) \textendash{} Physical address in EEPROM (default is 0xA0).

\end{itemize}

\sphinxlineitem{Returns}
\sphinxAtStartPar
List representing the MAC address.

\sphinxlineitem{Return type}
\sphinxAtStartPar
list

\end{description}\end{quote}

\end{fulllineitems}

\index{int2ip() (in module config\_ip)@\spxentry{int2ip()}\spxextra{in module config\_ip}}

\begin{fulllineitems}
\phantomsection\label{\detokenize{cplddocs:config_ip.int2ip}}
\pysigstartsignatures
\pysiglinewithargsret{\sphinxcode{\sphinxupquote{config\_ip.}}\sphinxbfcode{\sphinxupquote{int2ip}}}{\sphinxparam{\DUrole{n}{value}}}{}
\pysigstopsignatures
\sphinxAtStartPar
Convert an integer to an IP address string.
\begin{quote}\begin{description}
\sphinxlineitem{Parameters}
\sphinxAtStartPar
\sphinxstyleliteralstrong{\sphinxupquote{value}} (\sphinxstyleliteralemphasis{\sphinxupquote{int}}) \textendash{} Integer representation of the IP address.

\sphinxlineitem{Returns}
\sphinxAtStartPar
IP address string.

\sphinxlineitem{Return type}
\sphinxAtStartPar
str

\end{description}\end{quote}

\end{fulllineitems}

\index{nuple2mac() (in module config\_ip)@\spxentry{nuple2mac()}\spxextra{in module config\_ip}}

\begin{fulllineitems}
\phantomsection\label{\detokenize{cplddocs:config_ip.nuple2mac}}
\pysigstartsignatures
\pysiglinewithargsret{\sphinxcode{\sphinxupquote{config\_ip.}}\sphinxbfcode{\sphinxupquote{nuple2mac}}}{\sphinxparam{\DUrole{n}{mac}}}{}
\pysigstopsignatures
\sphinxAtStartPar
Convert a MAC address tuple to a string.
\begin{quote}\begin{description}
\sphinxlineitem{Parameters}
\sphinxAtStartPar
\sphinxstyleliteralstrong{\sphinxupquote{mac}} (\sphinxstyleliteralemphasis{\sphinxupquote{tuple}}) \textendash{} Tuple representing the MAC address.

\sphinxlineitem{Returns}
\sphinxAtStartPar
MAC address string.

\sphinxlineitem{Return type}
\sphinxAtStartPar
str

\end{description}\end{quote}

\end{fulllineitems}

\index{read\_string() (in module config\_ip)@\spxentry{read\_string()}\spxextra{in module config\_ip}}

\begin{fulllineitems}
\phantomsection\label{\detokenize{cplddocs:config_ip.read_string}}
\pysigstartsignatures
\pysiglinewithargsret{\sphinxcode{\sphinxupquote{config\_ip.}}\sphinxbfcode{\sphinxupquote{read\_string}}}{\sphinxparam{\DUrole{n}{inst}}\sphinxparamcomma \sphinxparam{\DUrole{n}{offset}}\sphinxparamcomma \sphinxparam{\DUrole{n}{max\_len}\DUrole{o}{=}\DUrole{default_value}{32}}}{}
\pysigstopsignatures
\sphinxAtStartPar
Read a string from EEPROM.
\begin{quote}\begin{description}
\sphinxlineitem{Parameters}\begin{itemize}
\item {} 
\sphinxAtStartPar
\sphinxstyleliteralstrong{\sphinxupquote{inst}} (\sphinxstyleliteralemphasis{\sphinxupquote{instance}}) \textendash{} Instance of a class with ‘bsp’ attribute.

\item {} 
\sphinxAtStartPar
\sphinxstyleliteralstrong{\sphinxupquote{offset}} (\sphinxstyleliteralemphasis{\sphinxupquote{int}}) \textendash{} Offset in EEPROM to start reading the string.

\item {} 
\sphinxAtStartPar
\sphinxstyleliteralstrong{\sphinxupquote{max\_len}} (\sphinxstyleliteralemphasis{\sphinxupquote{int}}) \textendash{} Maximum length of the string to read (default is 32).

\end{itemize}

\sphinxlineitem{Returns}
\sphinxAtStartPar
Read string from EEPROM.

\sphinxlineitem{Return type}
\sphinxAtStartPar
str

\end{description}\end{quote}

\end{fulllineitems}

\index{write\_string() (in module config\_ip)@\spxentry{write\_string()}\spxextra{in module config\_ip}}

\begin{fulllineitems}
\phantomsection\label{\detokenize{cplddocs:config_ip.write_string}}
\pysigstartsignatures
\pysiglinewithargsret{\sphinxcode{\sphinxupquote{config\_ip.}}\sphinxbfcode{\sphinxupquote{write\_string}}}{\sphinxparam{\DUrole{n}{inst}}\sphinxparamcomma \sphinxparam{\DUrole{n}{offset}}\sphinxparamcomma \sphinxparam{\DUrole{n}{string}}}{}
\pysigstopsignatures
\sphinxAtStartPar
Write a string to EEPROM.
\begin{quote}\begin{description}
\sphinxlineitem{Parameters}\begin{itemize}
\item {} 
\sphinxAtStartPar
\sphinxstyleliteralstrong{\sphinxupquote{inst}} (\sphinxstyleliteralemphasis{\sphinxupquote{instance}}) \textendash{} Instance of a class with ‘bsp’ attribute.

\item {} 
\sphinxAtStartPar
\sphinxstyleliteralstrong{\sphinxupquote{offset}} (\sphinxstyleliteralemphasis{\sphinxupquote{int}}) \textendash{} Offset in EEPROM to start writing the string.

\item {} 
\sphinxAtStartPar
\sphinxstyleliteralstrong{\sphinxupquote{string}} (\sphinxstyleliteralemphasis{\sphinxupquote{str}}) \textendash{} String to be written to EEPROM.

\end{itemize}

\end{description}\end{quote}

\end{fulllineitems}

\index{module@\spxentry{module}!cpld2mcu\_serial\_ctrl\_2@\spxentry{cpld2mcu\_serial\_ctrl\_2}}\index{cpld2mcu\_serial\_ctrl\_2@\spxentry{cpld2mcu\_serial\_ctrl\_2}!module@\spxentry{module}}\index{FlashCmd (class in cpld2mcu\_serial\_ctrl\_2)@\spxentry{FlashCmd}\spxextra{class in cpld2mcu\_serial\_ctrl\_2}}\phantomsection\label{\detokenize{cplddocs:module-cpld2mcu_serial_ctrl_2}}

\begin{fulllineitems}
\phantomsection\label{\detokenize{cplddocs:cpld2mcu_serial_ctrl_2.FlashCmd}}
\pysigstartsignatures
\pysigline{\sphinxbfcode{\sphinxupquote{class\DUrole{w}{ }}}\sphinxcode{\sphinxupquote{cpld2mcu\_serial\_ctrl\_2.}}\sphinxbfcode{\sphinxupquote{FlashCmd}}}
\pysigstopsignatures
\sphinxAtStartPar
Class containing flash commands and related constants.

\end{fulllineitems}

\index{load\_bitstream() (in module cpld2mcu\_serial\_ctrl\_2)@\spxentry{load\_bitstream()}\spxextra{in module cpld2mcu\_serial\_ctrl\_2}}

\begin{fulllineitems}
\phantomsection\label{\detokenize{cplddocs:cpld2mcu_serial_ctrl_2.load_bitstream}}
\pysigstartsignatures
\pysiglinewithargsret{\sphinxcode{\sphinxupquote{cpld2mcu\_serial\_ctrl\_2.}}\sphinxbfcode{\sphinxupquote{load\_bitstream}}}{\sphinxparam{\DUrole{n}{filename}}\sphinxparamcomma \sphinxparam{\DUrole{n}{pagesize}}}{}
\pysigstopsignatures
\sphinxAtStartPar
Load a bitstream from a file.
\begin{quote}\begin{description}
\sphinxlineitem{Parameters}\begin{itemize}
\item {} 
\sphinxAtStartPar
\sphinxstyleliteralstrong{\sphinxupquote{filename}} \textendash{} The name of the file to load.

\item {} 
\sphinxAtStartPar
\sphinxstyleliteralstrong{\sphinxupquote{pagesize}} \textendash{} The size of the pages.

\end{itemize}

\sphinxlineitem{Returns}
\sphinxAtStartPar
A tuple containing the loaded bitstream, bitstream size, and total size.

\end{description}\end{quote}

\end{fulllineitems}

\index{module@\spxentry{module}!management\_pll@\spxentry{management\_pll}}\index{management\_pll@\spxentry{management\_pll}!module@\spxentry{module}}\index{module@\spxentry{module}!mcu\_update@\spxentry{mcu\_update}}\index{mcu\_update@\spxentry{mcu\_update}!module@\spxentry{module}}\index{loadBitstream() (in module mcu\_update)@\spxentry{loadBitstream()}\spxextra{in module mcu\_update}}\phantomsection\label{\detokenize{cplddocs:module-management_pll}}\phantomsection\label{\detokenize{cplddocs:module-mcu_update}}

\begin{fulllineitems}
\phantomsection\label{\detokenize{cplddocs:mcu_update.loadBitstream}}
\pysigstartsignatures
\pysiglinewithargsret{\sphinxcode{\sphinxupquote{mcu\_update.}}\sphinxbfcode{\sphinxupquote{loadBitstream}}}{\sphinxparam{\DUrole{n}{filename}}\sphinxparamcomma \sphinxparam{\DUrole{n}{pagesize}}}{}
\pysigstopsignatures
\sphinxAtStartPar
Load the bitstream from a file.
\begin{quote}\begin{description}
\sphinxlineitem{Parameters}\begin{itemize}
\item {} 
\sphinxAtStartPar
\sphinxstyleliteralstrong{\sphinxupquote{filename}} (\sphinxstyleliteralemphasis{\sphinxupquote{str}}) \textendash{} The path to the bitstream file.

\item {} 
\sphinxAtStartPar
\sphinxstyleliteralstrong{\sphinxupquote{pagesize}} (\sphinxstyleliteralemphasis{\sphinxupquote{int}}) \textendash{} The size of the pages.

\end{itemize}

\sphinxlineitem{Returns}
\sphinxAtStartPar
The formatted bitstream.
int: The bitstream size.
int: The total size.

\sphinxlineitem{Return type}
\sphinxAtStartPar
bytearray

\end{description}\end{quote}

\end{fulllineitems}

\index{module@\spxentry{module}!mng\_update@\spxentry{mng\_update}}\index{mng\_update@\spxentry{mng\_update}!module@\spxentry{module}}\phantomsection\label{\detokenize{cplddocs:module-mng_update}}
\sphinxAtStartPar
Test TPM script.

\sphinxAtStartPar
\_\_author\_\_ = “Bubs”
\index{module@\spxentry{module}!phy\_marvell\_88X2222\_init@\spxentry{phy\_marvell\_88X2222\_init}}\index{phy\_marvell\_88X2222\_init@\spxentry{phy\_marvell\_88X2222\_init}!module@\spxentry{module}}\index{cfg\_10g() (in module phy\_marvell\_88X2222\_init)@\spxentry{cfg\_10g()}\spxextra{in module phy\_marvell\_88X2222\_init}}\phantomsection\label{\detokenize{cplddocs:module-phy_marvell_88X2222_init}}

\begin{fulllineitems}
\phantomsection\label{\detokenize{cplddocs:phy_marvell_88X2222_init.cfg_10g}}
\pysigstartsignatures
\pysiglinewithargsret{\sphinxcode{\sphinxupquote{phy\_marvell\_88X2222\_init.}}\sphinxbfcode{\sphinxupquote{cfg\_10g}}}{\sphinxparam{\DUrole{n}{port}\DUrole{o}{=}\DUrole{default_value}{0}}\sphinxparamcomma \sphinxparam{\DUrole{n}{mdio\_mux}\DUrole{o}{=}\DUrole{default_value}{3}}}{}
\pysigstopsignatures
\sphinxAtStartPar
Configure a 10G port.
\begin{quote}\begin{description}
\sphinxlineitem{Parameters}\begin{itemize}
\item {} 
\sphinxAtStartPar
\sphinxstyleliteralstrong{\sphinxupquote{port}} \textendash{} The port number (default is 0).

\item {} 
\sphinxAtStartPar
\sphinxstyleliteralstrong{\sphinxupquote{mdio\_mux}} \textendash{} The MDIO multiplexer value (default is 3).

\end{itemize}

\end{description}\end{quote}

\end{fulllineitems}

\index{decode\_register() (in module phy\_marvell\_88X2222\_init)@\spxentry{decode\_register()}\spxextra{in module phy\_marvell\_88X2222\_init}}

\begin{fulllineitems}
\phantomsection\label{\detokenize{cplddocs:phy_marvell_88X2222_init.decode_register}}
\pysigstartsignatures
\pysiglinewithargsret{\sphinxcode{\sphinxupquote{phy\_marvell\_88X2222\_init.}}\sphinxbfcode{\sphinxupquote{decode\_register}}}{\sphinxparam{\DUrole{n}{port}}\sphinxparamcomma \sphinxparam{\DUrole{n}{reg\_def}}\sphinxparamcomma \sphinxparam{\DUrole{n}{reg\_value}}\sphinxparamcomma \sphinxparam{\DUrole{n}{field}\DUrole{o}{=}\DUrole{default_value}{None}}}{}
\pysigstopsignatures
\sphinxAtStartPar
Decode and print the value of a register.
\begin{quote}\begin{description}
\sphinxlineitem{Parameters}\begin{itemize}
\item {} 
\sphinxAtStartPar
\sphinxstyleliteralstrong{\sphinxupquote{port}} \textendash{} The port number.

\item {} 
\sphinxAtStartPar
\sphinxstyleliteralstrong{\sphinxupquote{reg\_def}} \textendash{} The register definition.

\item {} 
\sphinxAtStartPar
\sphinxstyleliteralstrong{\sphinxupquote{reg\_value}} \textendash{} The value to decode.

\item {} 
\sphinxAtStartPar
\sphinxstyleliteralstrong{\sphinxupquote{field}} \textendash{} The specific field to decode (default is None).

\end{itemize}

\end{description}\end{quote}

\end{fulllineitems}

\index{get\_port\_cfg() (in module phy\_marvell\_88X2222\_init)@\spxentry{get\_port\_cfg()}\spxextra{in module phy\_marvell\_88X2222\_init}}

\begin{fulllineitems}
\phantomsection\label{\detokenize{cplddocs:phy_marvell_88X2222_init.get_port_cfg}}
\pysigstartsignatures
\pysiglinewithargsret{\sphinxcode{\sphinxupquote{phy\_marvell\_88X2222\_init.}}\sphinxbfcode{\sphinxupquote{get\_port\_cfg}}}{\sphinxparam{\DUrole{n}{port}}\sphinxparamcomma \sphinxparam{\DUrole{n}{mdio\_mux}\DUrole{o}{=}\DUrole{default_value}{2}}}{}
\pysigstopsignatures
\sphinxAtStartPar
Get and decode the configuration of a port.
\begin{quote}\begin{description}
\sphinxlineitem{Parameters}\begin{itemize}
\item {} 
\sphinxAtStartPar
\sphinxstyleliteralstrong{\sphinxupquote{port}} \textendash{} The port number.

\item {} 
\sphinxAtStartPar
\sphinxstyleliteralstrong{\sphinxupquote{mdio\_mux}} \textendash{} The MDIO multiplexer value (default is 2).

\end{itemize}

\end{description}\end{quote}

\end{fulllineitems}

\index{get\_switch\_status() (in module phy\_marvell\_88X2222\_init)@\spxentry{get\_switch\_status()}\spxextra{in module phy\_marvell\_88X2222\_init}}

\begin{fulllineitems}
\phantomsection\label{\detokenize{cplddocs:phy_marvell_88X2222_init.get_switch_status}}
\pysigstartsignatures
\pysiglinewithargsret{\sphinxcode{\sphinxupquote{phy\_marvell\_88X2222\_init.}}\sphinxbfcode{\sphinxupquote{get\_switch\_status}}}{}{}
\pysigstopsignatures
\sphinxAtStartPar
Get the status of various switch ports.
\begin{quote}\begin{description}
\sphinxlineitem{Returns}
\sphinxAtStartPar
Dictionary containing the status of each switch port.

\end{description}\end{quote}

\end{fulllineitems}

\index{rd() (in module phy\_marvell\_88X2222\_init)@\spxentry{rd()}\spxextra{in module phy\_marvell\_88X2222\_init}}

\begin{fulllineitems}
\phantomsection\label{\detokenize{cplddocs:phy_marvell_88X2222_init.rd}}
\pysigstartsignatures
\pysiglinewithargsret{\sphinxcode{\sphinxupquote{phy\_marvell\_88X2222\_init.}}\sphinxbfcode{\sphinxupquote{rd}}}{\sphinxparam{\DUrole{n}{address}}}{}
\pysigstopsignatures
\sphinxAtStartPar
Read a value from a given address.
\begin{quote}\begin{description}
\sphinxlineitem{Parameters}
\sphinxAtStartPar
\sphinxstyleliteralstrong{\sphinxupquote{address}} \textendash{} The address to read from.

\sphinxlineitem{Returns}
\sphinxAtStartPar
The value read from the address.

\end{description}\end{quote}

\end{fulllineitems}

\index{read22() (in module phy\_marvell\_88X2222\_init)@\spxentry{read22()}\spxextra{in module phy\_marvell\_88X2222\_init}}

\begin{fulllineitems}
\phantomsection\label{\detokenize{cplddocs:phy_marvell_88X2222_init.read22}}
\pysigstartsignatures
\pysiglinewithargsret{\sphinxcode{\sphinxupquote{phy\_marvell\_88X2222\_init.}}\sphinxbfcode{\sphinxupquote{read22}}}{\sphinxparam{\DUrole{n}{mux}}\sphinxparamcomma \sphinxparam{\DUrole{n}{phy\_adr}}\sphinxparamcomma \sphinxparam{\DUrole{n}{register}}}{}
\pysigstopsignatures
\sphinxAtStartPar
Read a value from a 22\sphinxhyphen{}bit register.
\begin{quote}\begin{description}
\sphinxlineitem{Parameters}\begin{itemize}
\item {} 
\sphinxAtStartPar
\sphinxstyleliteralstrong{\sphinxupquote{mux}} \textendash{} The multiplexer value.

\item {} 
\sphinxAtStartPar
\sphinxstyleliteralstrong{\sphinxupquote{phy\_adr}} \textendash{} The physical address.

\item {} 
\sphinxAtStartPar
\sphinxstyleliteralstrong{\sphinxupquote{register}} \textendash{} The register to read from.

\end{itemize}

\sphinxlineitem{Returns}
\sphinxAtStartPar
The value read from the register.

\end{description}\end{quote}

\end{fulllineitems}

\index{read45() (in module phy\_marvell\_88X2222\_init)@\spxentry{read45()}\spxextra{in module phy\_marvell\_88X2222\_init}}

\begin{fulllineitems}
\phantomsection\label{\detokenize{cplddocs:phy_marvell_88X2222_init.read45}}
\pysigstartsignatures
\pysiglinewithargsret{\sphinxcode{\sphinxupquote{phy\_marvell\_88X2222\_init.}}\sphinxbfcode{\sphinxupquote{read45}}}{\sphinxparam{\DUrole{n}{mux}}\sphinxparamcomma \sphinxparam{\DUrole{n}{phy\_adr}}\sphinxparamcomma \sphinxparam{\DUrole{n}{device}}\sphinxparamcomma \sphinxparam{\DUrole{n}{register}}}{}
\pysigstopsignatures
\sphinxAtStartPar
Read a value from a 45\sphinxhyphen{}bit register.
\begin{quote}\begin{description}
\sphinxlineitem{Parameters}\begin{itemize}
\item {} 
\sphinxAtStartPar
\sphinxstyleliteralstrong{\sphinxupquote{mux}} \textendash{} The multiplexer value.

\item {} 
\sphinxAtStartPar
\sphinxstyleliteralstrong{\sphinxupquote{phy\_adr}} \textendash{} The physical address.

\item {} 
\sphinxAtStartPar
\sphinxstyleliteralstrong{\sphinxupquote{device}} \textendash{} The device.

\item {} 
\sphinxAtStartPar
\sphinxstyleliteralstrong{\sphinxupquote{register}} \textendash{} The register to read from.

\end{itemize}

\sphinxlineitem{Returns}
\sphinxAtStartPar
The value read from the register.

\end{description}\end{quote}

\end{fulllineitems}

\index{read\_and\_decode() (in module phy\_marvell\_88X2222\_init)@\spxentry{read\_and\_decode()}\spxextra{in module phy\_marvell\_88X2222\_init}}

\begin{fulllineitems}
\phantomsection\label{\detokenize{cplddocs:phy_marvell_88X2222_init.read_and_decode}}
\pysigstartsignatures
\pysiglinewithargsret{\sphinxcode{\sphinxupquote{phy\_marvell\_88X2222\_init.}}\sphinxbfcode{\sphinxupquote{read\_and\_decode}}}{\sphinxparam{\DUrole{n}{port}}\sphinxparamcomma \sphinxparam{\DUrole{n}{reg\_def}}\sphinxparamcomma \sphinxparam{\DUrole{n}{mdio\_mux}\DUrole{o}{=}\DUrole{default_value}{2}}\sphinxparamcomma \sphinxparam{\DUrole{n}{field}\DUrole{o}{=}\DUrole{default_value}{None}}}{}
\pysigstopsignatures
\sphinxAtStartPar
Read and decode a register.
\begin{quote}\begin{description}
\sphinxlineitem{Parameters}\begin{itemize}
\item {} 
\sphinxAtStartPar
\sphinxstyleliteralstrong{\sphinxupquote{port}} \textendash{} The port number.

\item {} 
\sphinxAtStartPar
\sphinxstyleliteralstrong{\sphinxupquote{reg\_def}} \textendash{} The register definition.

\item {} 
\sphinxAtStartPar
\sphinxstyleliteralstrong{\sphinxupquote{mdio\_mux}} \textendash{} The MDIO multiplexer value (default is 2).

\item {} 
\sphinxAtStartPar
\sphinxstyleliteralstrong{\sphinxupquote{field}} \textendash{} The specific field to read (default is None).

\end{itemize}

\end{description}\end{quote}

\end{fulllineitems}

\index{read\_scratch() (in module phy\_marvell\_88X2222\_init)@\spxentry{read\_scratch()}\spxextra{in module phy\_marvell\_88X2222\_init}}

\begin{fulllineitems}
\phantomsection\label{\detokenize{cplddocs:phy_marvell_88X2222_init.read_scratch}}
\pysigstartsignatures
\pysiglinewithargsret{\sphinxcode{\sphinxupquote{phy\_marvell\_88X2222\_init.}}\sphinxbfcode{\sphinxupquote{read\_scratch}}}{\sphinxparam{\DUrole{n}{mux}}\sphinxparamcomma \sphinxparam{\DUrole{n}{offset}}}{}
\pysigstopsignatures
\sphinxAtStartPar
Read a value from the scratch register.
\begin{quote}\begin{description}
\sphinxlineitem{Parameters}\begin{itemize}
\item {} 
\sphinxAtStartPar
\sphinxstyleliteralstrong{\sphinxupquote{mux}} \textendash{} The MDIO multiplexer value.

\item {} 
\sphinxAtStartPar
\sphinxstyleliteralstrong{\sphinxupquote{offset}} \textendash{} The offset within the scratch register.

\end{itemize}

\sphinxlineitem{Returns}
\sphinxAtStartPar
The value read from the scratch register.

\end{description}\end{quote}

\end{fulllineitems}

\index{read\_wis() (in module phy\_marvell\_88X2222\_init)@\spxentry{read\_wis()}\spxextra{in module phy\_marvell\_88X2222\_init}}

\begin{fulllineitems}
\phantomsection\label{\detokenize{cplddocs:phy_marvell_88X2222_init.read_wis}}
\pysigstartsignatures
\pysiglinewithargsret{\sphinxcode{\sphinxupquote{phy\_marvell\_88X2222\_init.}}\sphinxbfcode{\sphinxupquote{read\_wis}}}{\sphinxparam{\DUrole{n}{mdio\_mux}\DUrole{o}{=}\DUrole{default_value}{3}}}{}
\pysigstopsignatures
\sphinxAtStartPar
Read and print the WIS device identifier.
\begin{quote}\begin{description}
\sphinxlineitem{Parameters}
\sphinxAtStartPar
\sphinxstyleliteralstrong{\sphinxupquote{mdio\_mux}} \textendash{} The MDIO multiplexer value (default is 3).

\end{description}\end{quote}

\end{fulllineitems}

\index{readmodifywrite() (in module phy\_marvell\_88X2222\_init)@\spxentry{readmodifywrite()}\spxextra{in module phy\_marvell\_88X2222\_init}}

\begin{fulllineitems}
\phantomsection\label{\detokenize{cplddocs:phy_marvell_88X2222_init.readmodifywrite}}
\pysigstartsignatures
\pysiglinewithargsret{\sphinxcode{\sphinxupquote{phy\_marvell\_88X2222\_init.}}\sphinxbfcode{\sphinxupquote{readmodifywrite}}}{\sphinxparam{\DUrole{n}{mux}}\sphinxparamcomma \sphinxparam{\DUrole{n}{phy\_adr}}\sphinxparamcomma \sphinxparam{\DUrole{n}{device}}\sphinxparamcomma \sphinxparam{\DUrole{n}{register}}\sphinxparamcomma \sphinxparam{\DUrole{n}{value}}\sphinxparamcomma \sphinxparam{\DUrole{n}{select}}}{}
\pysigstopsignatures
\sphinxAtStartPar
Perform read\sphinxhyphen{}modify\sphinxhyphen{}write operation on a 45\sphinxhyphen{}bit register.
\begin{quote}\begin{description}
\sphinxlineitem{Parameters}\begin{itemize}
\item {} 
\sphinxAtStartPar
\sphinxstyleliteralstrong{\sphinxupquote{mux}} \textendash{} The multiplexer value.

\item {} 
\sphinxAtStartPar
\sphinxstyleliteralstrong{\sphinxupquote{phy\_adr}} \textendash{} The physical address.

\item {} 
\sphinxAtStartPar
\sphinxstyleliteralstrong{\sphinxupquote{device}} \textendash{} The device.

\item {} 
\sphinxAtStartPar
\sphinxstyleliteralstrong{\sphinxupquote{register}} \textendash{} The register to read from.

\item {} 
\sphinxAtStartPar
\sphinxstyleliteralstrong{\sphinxupquote{value}} \textendash{} The value to write.

\item {} 
\sphinxAtStartPar
\sphinxstyleliteralstrong{\sphinxupquote{select}} \textendash{} The selection mask.

\end{itemize}

\end{description}\end{quote}

\end{fulllineitems}

\index{set\_SFP() (in module phy\_marvell\_88X2222\_init)@\spxentry{set\_SFP()}\spxextra{in module phy\_marvell\_88X2222\_init}}

\begin{fulllineitems}
\phantomsection\label{\detokenize{cplddocs:phy_marvell_88X2222_init.set_SFP}}
\pysigstartsignatures
\pysiglinewithargsret{\sphinxcode{\sphinxupquote{phy\_marvell\_88X2222\_init.}}\sphinxbfcode{\sphinxupquote{set\_SFP}}}{\sphinxparam{\DUrole{n}{mdio\_mux}\DUrole{o}{=}\DUrole{default_value}{2}}}{}
\pysigstopsignatures
\sphinxAtStartPar
Configure the SFP module.
\begin{quote}\begin{description}
\sphinxlineitem{Parameters}
\sphinxAtStartPar
\sphinxstyleliteralstrong{\sphinxupquote{mdio\_mux}} \textendash{} The MDIO multiplexer value (default is 2).

\end{description}\end{quote}

\end{fulllineitems}

\index{set\_field() (in module phy\_marvell\_88X2222\_init)@\spxentry{set\_field()}\spxextra{in module phy\_marvell\_88X2222\_init}}

\begin{fulllineitems}
\phantomsection\label{\detokenize{cplddocs:phy_marvell_88X2222_init.set_field}}
\pysigstartsignatures
\pysiglinewithargsret{\sphinxcode{\sphinxupquote{phy\_marvell\_88X2222\_init.}}\sphinxbfcode{\sphinxupquote{set\_field}}}{\sphinxparam{\DUrole{n}{port}}\sphinxparamcomma \sphinxparam{\DUrole{n}{reg\_def}}\sphinxparamcomma \sphinxparam{\DUrole{n}{field\_name}}\sphinxparamcomma \sphinxparam{\DUrole{n}{field\_value}}}{}
\pysigstopsignatures
\sphinxAtStartPar
Set a field in a register.
\begin{quote}\begin{description}
\sphinxlineitem{Parameters}\begin{itemize}
\item {} 
\sphinxAtStartPar
\sphinxstyleliteralstrong{\sphinxupquote{port}} \textendash{} The port number.

\item {} 
\sphinxAtStartPar
\sphinxstyleliteralstrong{\sphinxupquote{reg\_def}} \textendash{} The register definition.

\item {} 
\sphinxAtStartPar
\sphinxstyleliteralstrong{\sphinxupquote{field\_name}} \textendash{} The name of the field to set.

\item {} 
\sphinxAtStartPar
\sphinxstyleliteralstrong{\sphinxupquote{field\_value}} \textendash{} The value to set in the field.

\end{itemize}

\end{description}\end{quote}

\end{fulllineitems}

\index{wr() (in module phy\_marvell\_88X2222\_init)@\spxentry{wr()}\spxextra{in module phy\_marvell\_88X2222\_init}}

\begin{fulllineitems}
\phantomsection\label{\detokenize{cplddocs:phy_marvell_88X2222_init.wr}}
\pysigstartsignatures
\pysiglinewithargsret{\sphinxcode{\sphinxupquote{phy\_marvell\_88X2222\_init.}}\sphinxbfcode{\sphinxupquote{wr}}}{\sphinxparam{\DUrole{n}{address}}\sphinxparamcomma \sphinxparam{\DUrole{n}{value}}}{}
\pysigstopsignatures
\sphinxAtStartPar
Write a value to a given address.
\begin{quote}\begin{description}
\sphinxlineitem{Parameters}\begin{itemize}
\item {} 
\sphinxAtStartPar
\sphinxstyleliteralstrong{\sphinxupquote{address}} \textendash{} The address to write to.

\item {} 
\sphinxAtStartPar
\sphinxstyleliteralstrong{\sphinxupquote{value}} \textendash{} The value to write.

\end{itemize}

\end{description}\end{quote}

\end{fulllineitems}

\index{write22() (in module phy\_marvell\_88X2222\_init)@\spxentry{write22()}\spxextra{in module phy\_marvell\_88X2222\_init}}

\begin{fulllineitems}
\phantomsection\label{\detokenize{cplddocs:phy_marvell_88X2222_init.write22}}
\pysigstartsignatures
\pysiglinewithargsret{\sphinxcode{\sphinxupquote{phy\_marvell\_88X2222\_init.}}\sphinxbfcode{\sphinxupquote{write22}}}{\sphinxparam{\DUrole{n}{mux}}\sphinxparamcomma \sphinxparam{\DUrole{n}{phy\_adr}}\sphinxparamcomma \sphinxparam{\DUrole{n}{register}}\sphinxparamcomma \sphinxparam{\DUrole{n}{value}}}{}
\pysigstopsignatures
\sphinxAtStartPar
Write a value to a 22\sphinxhyphen{}bit register.
\begin{quote}\begin{description}
\sphinxlineitem{Parameters}\begin{itemize}
\item {} 
\sphinxAtStartPar
\sphinxstyleliteralstrong{\sphinxupquote{mux}} \textendash{} The multiplexer value.

\item {} 
\sphinxAtStartPar
\sphinxstyleliteralstrong{\sphinxupquote{phy\_adr}} \textendash{} The physical address.

\item {} 
\sphinxAtStartPar
\sphinxstyleliteralstrong{\sphinxupquote{register}} \textendash{} The register to write to.

\item {} 
\sphinxAtStartPar
\sphinxstyleliteralstrong{\sphinxupquote{value}} \textendash{} The value to write.

\end{itemize}

\end{description}\end{quote}

\end{fulllineitems}

\index{write22\_reg() (in module phy\_marvell\_88X2222\_init)@\spxentry{write22\_reg()}\spxextra{in module phy\_marvell\_88X2222\_init}}

\begin{fulllineitems}
\phantomsection\label{\detokenize{cplddocs:phy_marvell_88X2222_init.write22_reg}}
\pysigstartsignatures
\pysiglinewithargsret{\sphinxcode{\sphinxupquote{phy\_marvell\_88X2222\_init.}}\sphinxbfcode{\sphinxupquote{write22\_reg}}}{\sphinxparam{\DUrole{n}{port}}\sphinxparamcomma \sphinxparam{\DUrole{n}{reg\_def}}\sphinxparamcomma \sphinxparam{\DUrole{n}{reg\_value}}}{}
\pysigstopsignatures
\sphinxAtStartPar
Write a value to a 22\sphinxhyphen{}bit register.
\begin{quote}\begin{description}
\sphinxlineitem{Parameters}\begin{itemize}
\item {} 
\sphinxAtStartPar
\sphinxstyleliteralstrong{\sphinxupquote{port}} \textendash{} The port number.

\item {} 
\sphinxAtStartPar
\sphinxstyleliteralstrong{\sphinxupquote{reg\_def}} \textendash{} The register definition.

\item {} 
\sphinxAtStartPar
\sphinxstyleliteralstrong{\sphinxupquote{reg\_value}} \textendash{} The value to write.

\end{itemize}

\end{description}\end{quote}

\end{fulllineitems}

\index{write45() (in module phy\_marvell\_88X2222\_init)@\spxentry{write45()}\spxextra{in module phy\_marvell\_88X2222\_init}}

\begin{fulllineitems}
\phantomsection\label{\detokenize{cplddocs:phy_marvell_88X2222_init.write45}}
\pysigstartsignatures
\pysiglinewithargsret{\sphinxcode{\sphinxupquote{phy\_marvell\_88X2222\_init.}}\sphinxbfcode{\sphinxupquote{write45}}}{\sphinxparam{\DUrole{n}{mux}}\sphinxparamcomma \sphinxparam{\DUrole{n}{phy\_adr}}\sphinxparamcomma \sphinxparam{\DUrole{n}{device}}\sphinxparamcomma \sphinxparam{\DUrole{n}{register}}\sphinxparamcomma \sphinxparam{\DUrole{n}{value}}}{}
\pysigstopsignatures
\sphinxAtStartPar
Write a value to a 45\sphinxhyphen{}bit register.
\begin{quote}\begin{description}
\sphinxlineitem{Parameters}\begin{itemize}
\item {} 
\sphinxAtStartPar
\sphinxstyleliteralstrong{\sphinxupquote{mux}} \textendash{} The multiplexer value.

\item {} 
\sphinxAtStartPar
\sphinxstyleliteralstrong{\sphinxupquote{phy\_adr}} \textendash{} The physical address.

\item {} 
\sphinxAtStartPar
\sphinxstyleliteralstrong{\sphinxupquote{device}} \textendash{} The device.

\item {} 
\sphinxAtStartPar
\sphinxstyleliteralstrong{\sphinxupquote{register}} \textendash{} The register to write to.

\item {} 
\sphinxAtStartPar
\sphinxstyleliteralstrong{\sphinxupquote{value}} \textendash{} The value to write.

\end{itemize}

\end{description}\end{quote}

\end{fulllineitems}

\index{write\_scratch() (in module phy\_marvell\_88X2222\_init)@\spxentry{write\_scratch()}\spxextra{in module phy\_marvell\_88X2222\_init}}

\begin{fulllineitems}
\phantomsection\label{\detokenize{cplddocs:phy_marvell_88X2222_init.write_scratch}}
\pysigstartsignatures
\pysiglinewithargsret{\sphinxcode{\sphinxupquote{phy\_marvell\_88X2222\_init.}}\sphinxbfcode{\sphinxupquote{write\_scratch}}}{\sphinxparam{\DUrole{n}{mux}}\sphinxparamcomma \sphinxparam{\DUrole{n}{offset}}\sphinxparamcomma \sphinxparam{\DUrole{n}{value}}}{}
\pysigstopsignatures
\sphinxAtStartPar
Write a value to the scratch register.
\begin{quote}\begin{description}
\sphinxlineitem{Parameters}\begin{itemize}
\item {} 
\sphinxAtStartPar
\sphinxstyleliteralstrong{\sphinxupquote{mux}} \textendash{} The MDIO multiplexer value.

\item {} 
\sphinxAtStartPar
\sphinxstyleliteralstrong{\sphinxupquote{offset}} \textendash{} The offset within the scratch register.

\item {} 
\sphinxAtStartPar
\sphinxstyleliteralstrong{\sphinxupquote{value}} \textendash{} The value to write.

\end{itemize}

\end{description}\end{quote}

\end{fulllineitems}

\index{module@\spxentry{module}!reg@\spxentry{reg}}\index{reg@\spxentry{reg}!module@\spxentry{module}}\index{get\_max\_width() (in module reg)@\spxentry{get\_max\_width()}\spxextra{in module reg}}\phantomsection\label{\detokenize{cplddocs:module-reg}}

\begin{fulllineitems}
\phantomsection\label{\detokenize{cplddocs:reg.get_max_width}}
\pysigstartsignatures
\pysiglinewithargsret{\sphinxcode{\sphinxupquote{reg.}}\sphinxbfcode{\sphinxupquote{get\_max\_width}}}{\sphinxparam{\DUrole{n}{table1}}\sphinxparamcomma \sphinxparam{\DUrole{n}{index1}}}{}
\pysigstopsignatures
\sphinxAtStartPar
Get the maximum width of the given column index

\end{fulllineitems}

\index{pprint\_table() (in module reg)@\spxentry{pprint\_table()}\spxextra{in module reg}}

\begin{fulllineitems}
\phantomsection\label{\detokenize{cplddocs:reg.pprint_table}}
\pysigstartsignatures
\pysiglinewithargsret{\sphinxcode{\sphinxupquote{reg.}}\sphinxbfcode{\sphinxupquote{pprint\_table}}}{\sphinxparam{\DUrole{n}{table}}}{}
\pysigstopsignatures
\sphinxAtStartPar
Prints out a table of data.
@param table: The table to print. A list of lists.
Each row must have the same number of columns.

\end{fulllineitems}

\index{module@\spxentry{module}!rmp@\spxentry{rmp}}\index{rmp@\spxentry{rmp}!module@\spxentry{module}}\phantomsection\label{\detokenize{cplddocs:module-rmp}}
\sphinxAtStartPar
@package rmp
UDP socket management and RMP packet encoding/decoding

\sphinxAtStartPar
This package provides functions for network initializing and basic 32\sphinxhyphen{}bit read/write
operations on the network\sphinxhyphen{}attached device using RMP protocol. This is rough and minimal code
not exploiting all the RMP protocol features.
\index{module@\spxentry{module}!management\_bsp@\spxentry{management\_bsp}}\index{management\_bsp@\spxentry{management\_bsp}!module@\spxentry{module}}\index{module@\spxentry{module}!management\_flash@\spxentry{management\_flash}}\index{management\_flash@\spxentry{management\_flash}!module@\spxentry{module}}\index{MngProgFlash (class in management\_flash)@\spxentry{MngProgFlash}\spxextra{class in management\_flash}}\phantomsection\label{\detokenize{cplddocs:module-management_bsp}}\phantomsection\label{\detokenize{cplddocs:module-management_flash}}

\begin{fulllineitems}
\phantomsection\label{\detokenize{cplddocs:management_flash.MngProgFlash}}
\pysigstartsignatures
\pysiglinewithargsret{\sphinxbfcode{\sphinxupquote{class\DUrole{w}{ }}}\sphinxcode{\sphinxupquote{management\_flash.}}\sphinxbfcode{\sphinxupquote{MngProgFlash}}}{\sphinxparam{\DUrole{n}{board}}\sphinxparamcomma \sphinxparam{\DUrole{n}{rmp}}}{}
\pysigstopsignatures
\sphinxAtStartPar
Management Class for CPLD and FPGA SPI Flash bitfile storage/access class.

\sphinxAtStartPar
Attributes:
\sphinxhyphen{} rmp: Pointer to RMP instance.
\sphinxhyphen{} board: Pointer to board instance.
\sphinxhyphen{} add4bytemode (bool): Flag indicating whether 4\sphinxhyphen{}byte addressing mode is enabled.
\index{DeviceErase() (management\_flash.MngProgFlash method)@\spxentry{DeviceErase()}\spxextra{management\_flash.MngProgFlash method}}

\begin{fulllineitems}
\phantomsection\label{\detokenize{cplddocs:management_flash.MngProgFlash.DeviceErase}}
\pysigstartsignatures
\pysiglinewithargsret{\sphinxbfcode{\sphinxupquote{DeviceErase}}}{\sphinxparam{\DUrole{n}{flashdeviceindedx}}\sphinxparamcomma \sphinxparam{\DUrole{n}{address}}\sphinxparamcomma \sphinxparam{\DUrole{n}{size}}}{}
\pysigstopsignatures
\sphinxAtStartPar
Erase a specified range on the flash device.
\begin{quote}\begin{description}
\sphinxlineitem{Parameters}\begin{itemize}
\item {} 
\sphinxAtStartPar
\sphinxstyleliteralstrong{\sphinxupquote{self}} \textendash{} The object instance.

\item {} 
\sphinxAtStartPar
\sphinxstyleliteralstrong{\sphinxupquote{flashdeviceindedx}} (\sphinxstyleliteralemphasis{\sphinxupquote{int}}) \textendash{} Index of the flash device.

\item {} 
\sphinxAtStartPar
\sphinxstyleliteralstrong{\sphinxupquote{address}} (\sphinxstyleliteralemphasis{\sphinxupquote{int}}) \textendash{} Address in the flash to start erasing.

\item {} 
\sphinxAtStartPar
\sphinxstyleliteralstrong{\sphinxupquote{size}} (\sphinxstyleliteralemphasis{\sphinxupquote{int}}) \textendash{} Size of the range to erase.

\end{itemize}

\end{description}\end{quote}

\end{fulllineitems}

\index{DeviceEraseChip() (management\_flash.MngProgFlash method)@\spxentry{DeviceEraseChip()}\spxextra{management\_flash.MngProgFlash method}}

\begin{fulllineitems}
\phantomsection\label{\detokenize{cplddocs:management_flash.MngProgFlash.DeviceEraseChip}}
\pysigstartsignatures
\pysiglinewithargsret{\sphinxbfcode{\sphinxupquote{DeviceEraseChip}}}{\sphinxparam{\DUrole{n}{flashdeviceindedx}}}{}
\pysigstopsignatures
\sphinxAtStartPar
Erase the entire flash chip.
\begin{quote}\begin{description}
\sphinxlineitem{Parameters}\begin{itemize}
\item {} 
\sphinxAtStartPar
\sphinxstyleliteralstrong{\sphinxupquote{self}} \textendash{} The object instance.

\item {} 
\sphinxAtStartPar
\sphinxstyleliteralstrong{\sphinxupquote{flashdeviceindedx}} (\sphinxstyleliteralemphasis{\sphinxupquote{int}}) \textendash{} Index of the flash device.

\end{itemize}

\end{description}\end{quote}

\end{fulllineitems}

\index{DeviceGetID() (management\_flash.MngProgFlash method)@\spxentry{DeviceGetID()}\spxextra{management\_flash.MngProgFlash method}}

\begin{fulllineitems}
\phantomsection\label{\detokenize{cplddocs:management_flash.MngProgFlash.DeviceGetID}}
\pysigstartsignatures
\pysiglinewithargsret{\sphinxbfcode{\sphinxupquote{DeviceGetID}}}{\sphinxparam{\DUrole{n}{flashdeviceindedx}}}{}
\pysigstopsignatures
\sphinxAtStartPar
Get the identification of the flash device.
\begin{quote}\begin{description}
\sphinxlineitem{Parameters}\begin{itemize}
\item {} 
\sphinxAtStartPar
\sphinxstyleliteralstrong{\sphinxupquote{self}} \textendash{} The object instance.

\item {} 
\sphinxAtStartPar
\sphinxstyleliteralstrong{\sphinxupquote{flashdeviceindedx}} (\sphinxstyleliteralemphasis{\sphinxupquote{int}}) \textendash{} Index of the flash device.

\end{itemize}

\sphinxlineitem{Returns}
\sphinxAtStartPar
Identification of the flash device.

\sphinxlineitem{Return type}
\sphinxAtStartPar
int

\end{description}\end{quote}

\end{fulllineitems}

\index{DeviceGetInfo() (management\_flash.MngProgFlash method)@\spxentry{DeviceGetInfo()}\spxextra{management\_flash.MngProgFlash method}}

\begin{fulllineitems}
\phantomsection\label{\detokenize{cplddocs:management_flash.MngProgFlash.DeviceGetInfo}}
\pysigstartsignatures
\pysiglinewithargsret{\sphinxbfcode{\sphinxupquote{DeviceGetInfo}}}{\sphinxparam{\DUrole{n}{flashdeviceindedx}}}{}
\pysigstopsignatures
\sphinxAtStartPar
Get information about the flash device.
\begin{quote}\begin{description}
\sphinxlineitem{Parameters}\begin{itemize}
\item {} 
\sphinxAtStartPar
\sphinxstyleliteralstrong{\sphinxupquote{self}} \textendash{} The object instance.

\item {} 
\sphinxAtStartPar
\sphinxstyleliteralstrong{\sphinxupquote{flashdeviceindedx}} (\sphinxstyleliteralemphasis{\sphinxupquote{int}}) \textendash{} Index of the flash device.

\end{itemize}

\sphinxlineitem{Returns}
\sphinxAtStartPar
Information about the flash device.

\sphinxlineitem{Return type}
\sphinxAtStartPar
FlashDevice

\end{description}\end{quote}

\end{fulllineitems}

\index{DeviceWrite() (management\_flash.MngProgFlash method)@\spxentry{DeviceWrite()}\spxextra{management\_flash.MngProgFlash method}}

\begin{fulllineitems}
\phantomsection\label{\detokenize{cplddocs:management_flash.MngProgFlash.DeviceWrite}}
\pysigstartsignatures
\pysiglinewithargsret{\sphinxbfcode{\sphinxupquote{DeviceWrite}}}{\sphinxparam{\DUrole{n}{flashdeviceindedx}}\sphinxparamcomma \sphinxparam{\DUrole{n}{address}}\sphinxparamcomma \sphinxparam{\DUrole{n}{txbuff}}\sphinxparamcomma \sphinxparam{\DUrole{n}{size}}}{}
\pysigstopsignatures
\sphinxAtStartPar
Write data to the flash device.
\begin{quote}\begin{description}
\sphinxlineitem{Parameters}\begin{itemize}
\item {} 
\sphinxAtStartPar
\sphinxstyleliteralstrong{\sphinxupquote{self}} \textendash{} The object instance.

\item {} 
\sphinxAtStartPar
\sphinxstyleliteralstrong{\sphinxupquote{flashdeviceindedx}} (\sphinxstyleliteralemphasis{\sphinxupquote{int}}) \textendash{} Index of the flash device.

\item {} 
\sphinxAtStartPar
\sphinxstyleliteralstrong{\sphinxupquote{address}} (\sphinxstyleliteralemphasis{\sphinxupquote{int}}) \textendash{} Address in the flash where to start writing.

\item {} 
\sphinxAtStartPar
\sphinxstyleliteralstrong{\sphinxupquote{txbuff}} (\sphinxstyleliteralemphasis{\sphinxupquote{bytes}}) \textendash{} Data to be written.

\item {} 
\sphinxAtStartPar
\sphinxstyleliteralstrong{\sphinxupquote{size}} (\sphinxstyleliteralemphasis{\sphinxupquote{int}}) \textendash{} Size of the data to write.

\end{itemize}

\end{description}\end{quote}

\end{fulllineitems}

\index{FlashDevice\_Enter4byteAddMode() (management\_flash.MngProgFlash method)@\spxentry{FlashDevice\_Enter4byteAddMode()}\spxextra{management\_flash.MngProgFlash method}}

\begin{fulllineitems}
\phantomsection\label{\detokenize{cplddocs:management_flash.MngProgFlash.FlashDevice_Enter4byteAddMode}}
\pysigstartsignatures
\pysiglinewithargsret{\sphinxbfcode{\sphinxupquote{FlashDevice\_Enter4byteAddMode}}}{\sphinxparam{\DUrole{n}{device}}}{}
\pysigstopsignatures
\sphinxAtStartPar
Enter 4\sphinxhyphen{}byte addressing mode for Flash device.
\begin{quote}\begin{description}
\sphinxlineitem{Parameters}
\sphinxAtStartPar
\sphinxstyleliteralstrong{\sphinxupquote{device}} \textendash{} Flash device.

\end{description}\end{quote}

\end{fulllineitems}

\index{FlashDevice\_Exit4byteAddMode() (management\_flash.MngProgFlash method)@\spxentry{FlashDevice\_Exit4byteAddMode()}\spxextra{management\_flash.MngProgFlash method}}

\begin{fulllineitems}
\phantomsection\label{\detokenize{cplddocs:management_flash.MngProgFlash.FlashDevice_Exit4byteAddMode}}
\pysigstartsignatures
\pysiglinewithargsret{\sphinxbfcode{\sphinxupquote{FlashDevice\_Exit4byteAddMode}}}{\sphinxparam{\DUrole{n}{device}}}{}
\pysigstopsignatures
\sphinxAtStartPar
Exit 4\sphinxhyphen{}byte addressing mode for Flash device.
\begin{quote}\begin{description}
\sphinxlineitem{Parameters}
\sphinxAtStartPar
\sphinxstyleliteralstrong{\sphinxupquote{device}} \textendash{} Flash device.

\end{description}\end{quote}

\end{fulllineitems}

\index{FlashDevice\_chiperase() (management\_flash.MngProgFlash method)@\spxentry{FlashDevice\_chiperase()}\spxextra{management\_flash.MngProgFlash method}}

\begin{fulllineitems}
\phantomsection\label{\detokenize{cplddocs:management_flash.MngProgFlash.FlashDevice_chiperase}}
\pysigstartsignatures
\pysiglinewithargsret{\sphinxbfcode{\sphinxupquote{FlashDevice\_chiperase}}}{\sphinxparam{\DUrole{n}{device}}}{}
\pysigstopsignatures
\sphinxAtStartPar
Erase the entire Flash chip.
\begin{quote}\begin{description}
\sphinxlineitem{Parameters}
\sphinxAtStartPar
\sphinxstyleliteralstrong{\sphinxupquote{device}} \textendash{} Flash device.

\end{description}\end{quote}

\end{fulllineitems}

\index{FlashDevice\_erase() (management\_flash.MngProgFlash method)@\spxentry{FlashDevice\_erase()}\spxextra{management\_flash.MngProgFlash method}}

\begin{fulllineitems}
\phantomsection\label{\detokenize{cplddocs:management_flash.MngProgFlash.FlashDevice_erase}}
\pysigstartsignatures
\pysiglinewithargsret{\sphinxbfcode{\sphinxupquote{FlashDevice\_erase}}}{\sphinxparam{\DUrole{n}{device}}\sphinxparamcomma \sphinxparam{\DUrole{n}{address}}\sphinxparamcomma \sphinxparam{\DUrole{n}{size}}}{}
\pysigstopsignatures
\sphinxAtStartPar
Erase a range of memory in the Flash device.
\begin{quote}\begin{description}
\sphinxlineitem{Parameters}\begin{itemize}
\item {} 
\sphinxAtStartPar
\sphinxstyleliteralstrong{\sphinxupquote{device}} \textendash{} Flash device.

\item {} 
\sphinxAtStartPar
\sphinxstyleliteralstrong{\sphinxupquote{address}} \textendash{} Starting address of the memory range to erase.

\item {} 
\sphinxAtStartPar
\sphinxstyleliteralstrong{\sphinxupquote{size}} \textendash{} Size of the memory range to erase.

\end{itemize}

\end{description}\end{quote}

\end{fulllineitems}

\index{FlashDevice\_eraseSector() (management\_flash.MngProgFlash method)@\spxentry{FlashDevice\_eraseSector()}\spxextra{management\_flash.MngProgFlash method}}

\begin{fulllineitems}
\phantomsection\label{\detokenize{cplddocs:management_flash.MngProgFlash.FlashDevice_eraseSector}}
\pysigstartsignatures
\pysiglinewithargsret{\sphinxbfcode{\sphinxupquote{FlashDevice\_eraseSector}}}{\sphinxparam{\DUrole{n}{device}}\sphinxparamcomma \sphinxparam{\DUrole{n}{address}}}{}
\pysigstopsignatures
\sphinxAtStartPar
Erase a sector in the Flash device.
\begin{quote}\begin{description}
\sphinxlineitem{Parameters}\begin{itemize}
\item {} 
\sphinxAtStartPar
\sphinxstyleliteralstrong{\sphinxupquote{device}} \textendash{} Flash device.

\item {} 
\sphinxAtStartPar
\sphinxstyleliteralstrong{\sphinxupquote{address}} \textendash{} Memory address of the sector to erase.

\end{itemize}

\end{description}\end{quote}

\end{fulllineitems}

\index{FlashDevice\_prepareCommand() (management\_flash.MngProgFlash method)@\spxentry{FlashDevice\_prepareCommand()}\spxextra{management\_flash.MngProgFlash method}}

\begin{fulllineitems}
\phantomsection\label{\detokenize{cplddocs:management_flash.MngProgFlash.FlashDevice_prepareCommand}}
\pysigstartsignatures
\pysiglinewithargsret{\sphinxbfcode{\sphinxupquote{FlashDevice\_prepareCommand}}}{\sphinxparam{\DUrole{n}{command}}\sphinxparamcomma \sphinxparam{\DUrole{n}{address}}\sphinxparamcomma \sphinxparam{\DUrole{n}{device}}}{}
\pysigstopsignatures
\sphinxAtStartPar
Prepare a Flash device command.
\begin{quote}\begin{description}
\sphinxlineitem{Parameters}\begin{itemize}
\item {} 
\sphinxAtStartPar
\sphinxstyleliteralstrong{\sphinxupquote{command}} \textendash{} Command code.

\item {} 
\sphinxAtStartPar
\sphinxstyleliteralstrong{\sphinxupquote{address}} \textendash{} Flash memory address.

\item {} 
\sphinxAtStartPar
\sphinxstyleliteralstrong{\sphinxupquote{device}} \textendash{} Flash device.

\end{itemize}

\sphinxlineitem{Returns}
\sphinxAtStartPar
Prepared command buffer.

\end{description}\end{quote}

\end{fulllineitems}

\index{FlashDevice\_readIdentification() (management\_flash.MngProgFlash method)@\spxentry{FlashDevice\_readIdentification()}\spxextra{management\_flash.MngProgFlash method}}

\begin{fulllineitems}
\phantomsection\label{\detokenize{cplddocs:management_flash.MngProgFlash.FlashDevice_readIdentification}}
\pysigstartsignatures
\pysiglinewithargsret{\sphinxbfcode{\sphinxupquote{FlashDevice\_readIdentification}}}{\sphinxparam{\DUrole{n}{device}}}{}
\pysigstopsignatures
\sphinxAtStartPar
Read identification from the Flash device.
\begin{quote}\begin{description}
\sphinxlineitem{Parameters}
\sphinxAtStartPar
\sphinxstyleliteralstrong{\sphinxupquote{device}} \textendash{} Flash device.

\sphinxlineitem{Returns}
\sphinxAtStartPar
Identification value.

\end{description}\end{quote}

\end{fulllineitems}

\index{FlashDevice\_readPage() (management\_flash.MngProgFlash method)@\spxentry{FlashDevice\_readPage()}\spxextra{management\_flash.MngProgFlash method}}

\begin{fulllineitems}
\phantomsection\label{\detokenize{cplddocs:management_flash.MngProgFlash.FlashDevice_readPage}}
\pysigstartsignatures
\pysiglinewithargsret{\sphinxbfcode{\sphinxupquote{FlashDevice\_readPage}}}{\sphinxparam{\DUrole{n}{device}}\sphinxparamcomma \sphinxparam{\DUrole{n}{address}}\sphinxparamcomma \sphinxparam{\DUrole{n}{size}}}{}
\pysigstopsignatures
\sphinxAtStartPar
Read a page from the Flash device.
\begin{quote}\begin{description}
\sphinxlineitem{Parameters}\begin{itemize}
\item {} 
\sphinxAtStartPar
\sphinxstyleliteralstrong{\sphinxupquote{device}} \textendash{} Flash device.

\item {} 
\sphinxAtStartPar
\sphinxstyleliteralstrong{\sphinxupquote{address}} \textendash{} Memory address to read from.

\item {} 
\sphinxAtStartPar
\sphinxstyleliteralstrong{\sphinxupquote{size}} \textendash{} Size of the page to read.

\end{itemize}

\sphinxlineitem{Returns}
\sphinxAtStartPar
Read buffer.

\end{description}\end{quote}

\end{fulllineitems}

\index{FlashDevice\_readReg() (management\_flash.MngProgFlash method)@\spxentry{FlashDevice\_readReg()}\spxextra{management\_flash.MngProgFlash method}}

\begin{fulllineitems}
\phantomsection\label{\detokenize{cplddocs:management_flash.MngProgFlash.FlashDevice_readReg}}
\pysigstartsignatures
\pysiglinewithargsret{\sphinxbfcode{\sphinxupquote{FlashDevice\_readReg}}}{\sphinxparam{\DUrole{n}{device}}\sphinxparamcomma \sphinxparam{\DUrole{n}{reg}}}{}
\pysigstopsignatures
\sphinxAtStartPar
Read from Flash device register.
\begin{quote}\begin{description}
\sphinxlineitem{Parameters}\begin{itemize}
\item {} 
\sphinxAtStartPar
\sphinxstyleliteralstrong{\sphinxupquote{device}} \textendash{} Flash device.

\item {} 
\sphinxAtStartPar
\sphinxstyleliteralstrong{\sphinxupquote{reg}} \textendash{} Register to read.

\end{itemize}

\sphinxlineitem{Returns}
\sphinxAtStartPar
Register value.

\end{description}\end{quote}

\end{fulllineitems}

\index{FlashDevice\_readsector() (management\_flash.MngProgFlash method)@\spxentry{FlashDevice\_readsector()}\spxextra{management\_flash.MngProgFlash method}}

\begin{fulllineitems}
\phantomsection\label{\detokenize{cplddocs:management_flash.MngProgFlash.FlashDevice_readsector}}
\pysigstartsignatures
\pysiglinewithargsret{\sphinxbfcode{\sphinxupquote{FlashDevice\_readsector}}}{\sphinxparam{\DUrole{n}{device}}\sphinxparamcomma \sphinxparam{\DUrole{n}{address}}}{}
\pysigstopsignatures
\sphinxAtStartPar
Read a sector from the Flash device.
\begin{quote}\begin{description}
\sphinxlineitem{Parameters}\begin{itemize}
\item {} 
\sphinxAtStartPar
\sphinxstyleliteralstrong{\sphinxupquote{device}} \textendash{} Flash device.

\item {} 
\sphinxAtStartPar
\sphinxstyleliteralstrong{\sphinxupquote{address}} \textendash{} Memory address to read from.

\end{itemize}

\sphinxlineitem{Returns}
\sphinxAtStartPar
Read buffer.

\end{description}\end{quote}

\end{fulllineitems}

\index{FlashDevice\_waitTillReady() (management\_flash.MngProgFlash method)@\spxentry{FlashDevice\_waitTillReady()}\spxextra{management\_flash.MngProgFlash method}}

\begin{fulllineitems}
\phantomsection\label{\detokenize{cplddocs:management_flash.MngProgFlash.FlashDevice_waitTillReady}}
\pysigstartsignatures
\pysiglinewithargsret{\sphinxbfcode{\sphinxupquote{FlashDevice\_waitTillReady}}}{\sphinxparam{\DUrole{n}{device}}}{}
\pysigstopsignatures
\sphinxAtStartPar
Wait until the Flash device is ready.
\begin{quote}\begin{description}
\sphinxlineitem{Parameters}
\sphinxAtStartPar
\sphinxstyleliteralstrong{\sphinxupquote{device}} \textendash{} Flash device.

\end{description}\end{quote}

\end{fulllineitems}

\index{FlashDevice\_writeDisable() (management\_flash.MngProgFlash method)@\spxentry{FlashDevice\_writeDisable()}\spxextra{management\_flash.MngProgFlash method}}

\begin{fulllineitems}
\phantomsection\label{\detokenize{cplddocs:management_flash.MngProgFlash.FlashDevice_writeDisable}}
\pysigstartsignatures
\pysiglinewithargsret{\sphinxbfcode{\sphinxupquote{FlashDevice\_writeDisable}}}{\sphinxparam{\DUrole{n}{device}}}{}
\pysigstopsignatures
\sphinxAtStartPar
Disable write for the Flash device.
\begin{quote}\begin{description}
\sphinxlineitem{Parameters}
\sphinxAtStartPar
\sphinxstyleliteralstrong{\sphinxupquote{device}} \textendash{} Flash device.

\end{description}\end{quote}

\end{fulllineitems}

\index{FlashDevice\_writeEnable() (management\_flash.MngProgFlash method)@\spxentry{FlashDevice\_writeEnable()}\spxextra{management\_flash.MngProgFlash method}}

\begin{fulllineitems}
\phantomsection\label{\detokenize{cplddocs:management_flash.MngProgFlash.FlashDevice_writeEnable}}
\pysigstartsignatures
\pysiglinewithargsret{\sphinxbfcode{\sphinxupquote{FlashDevice\_writeEnable}}}{\sphinxparam{\DUrole{n}{device}}}{}
\pysigstopsignatures
\sphinxAtStartPar
Enable write for the Flash device.
\begin{quote}\begin{description}
\sphinxlineitem{Parameters}
\sphinxAtStartPar
\sphinxstyleliteralstrong{\sphinxupquote{device}} \textendash{} Flash device.

\end{description}\end{quote}

\end{fulllineitems}

\index{FlashDevice\_writePage() (management\_flash.MngProgFlash method)@\spxentry{FlashDevice\_writePage()}\spxextra{management\_flash.MngProgFlash method}}

\begin{fulllineitems}
\phantomsection\label{\detokenize{cplddocs:management_flash.MngProgFlash.FlashDevice_writePage}}
\pysigstartsignatures
\pysiglinewithargsret{\sphinxbfcode{\sphinxupquote{FlashDevice\_writePage}}}{\sphinxparam{\DUrole{n}{device}}\sphinxparamcomma \sphinxparam{\DUrole{n}{address}}\sphinxparamcomma \sphinxparam{\DUrole{n}{size}}\sphinxparamcomma \sphinxparam{\DUrole{n}{buffer}}}{}
\pysigstopsignatures
\sphinxAtStartPar
Write a page to the Flash device.
\begin{quote}\begin{description}
\sphinxlineitem{Parameters}\begin{itemize}
\item {} 
\sphinxAtStartPar
\sphinxstyleliteralstrong{\sphinxupquote{device}} \textendash{} Flash device.

\item {} 
\sphinxAtStartPar
\sphinxstyleliteralstrong{\sphinxupquote{address}} \textendash{} Memory address to write to.

\item {} 
\sphinxAtStartPar
\sphinxstyleliteralstrong{\sphinxupquote{size}} \textendash{} Size of the data to write.

\item {} 
\sphinxAtStartPar
\sphinxstyleliteralstrong{\sphinxupquote{buffer}} \textendash{} Data buffer to write.

\end{itemize}

\sphinxlineitem{Returns}
\sphinxAtStartPar
Read buffer.

\end{description}\end{quote}

\end{fulllineitems}

\index{FlashDevice\_writeReg() (management\_flash.MngProgFlash method)@\spxentry{FlashDevice\_writeReg()}\spxextra{management\_flash.MngProgFlash method}}

\begin{fulllineitems}
\phantomsection\label{\detokenize{cplddocs:management_flash.MngProgFlash.FlashDevice_writeReg}}
\pysigstartsignatures
\pysiglinewithargsret{\sphinxbfcode{\sphinxupquote{FlashDevice\_writeReg}}}{\sphinxparam{\DUrole{n}{device}}\sphinxparamcomma \sphinxparam{\DUrole{n}{reg}}\sphinxparamcomma \sphinxparam{\DUrole{n}{value}\DUrole{o}{=}\DUrole{default_value}{None}}}{}
\pysigstopsignatures
\sphinxAtStartPar
Write to Flash device register.
\begin{quote}\begin{description}
\sphinxlineitem{Parameters}\begin{itemize}
\item {} 
\sphinxAtStartPar
\sphinxstyleliteralstrong{\sphinxupquote{device}} \textendash{} Flash device.

\item {} 
\sphinxAtStartPar
\sphinxstyleliteralstrong{\sphinxupquote{reg}} \textendash{} Register to write.

\item {} 
\sphinxAtStartPar
\sphinxstyleliteralstrong{\sphinxupquote{value}} \textendash{} Value to write.

\end{itemize}

\end{description}\end{quote}

\end{fulllineitems}

\index{FlashDevice\_writesector() (management\_flash.MngProgFlash method)@\spxentry{FlashDevice\_writesector()}\spxextra{management\_flash.MngProgFlash method}}

\begin{fulllineitems}
\phantomsection\label{\detokenize{cplddocs:management_flash.MngProgFlash.FlashDevice_writesector}}
\pysigstartsignatures
\pysiglinewithargsret{\sphinxbfcode{\sphinxupquote{FlashDevice\_writesector}}}{\sphinxparam{\DUrole{n}{device}}\sphinxparamcomma \sphinxparam{\DUrole{n}{address}}\sphinxparamcomma \sphinxparam{\DUrole{n}{buffer}}}{}
\pysigstopsignatures
\sphinxAtStartPar
Write a sector to the Flash device.
\begin{quote}\begin{description}
\sphinxlineitem{Parameters}\begin{itemize}
\item {} 
\sphinxAtStartPar
\sphinxstyleliteralstrong{\sphinxupquote{device}} \textendash{} Flash device.

\item {} 
\sphinxAtStartPar
\sphinxstyleliteralstrong{\sphinxupquote{address}} \textendash{} Memory address to write to.

\item {} 
\sphinxAtStartPar
\sphinxstyleliteralstrong{\sphinxupquote{buffer}} \textendash{} Data buffer to write.

\end{itemize}

\end{description}\end{quote}

\end{fulllineitems}

\index{SPITransaction() (management\_flash.MngProgFlash method)@\spxentry{SPITransaction()}\spxextra{management\_flash.MngProgFlash method}}

\begin{fulllineitems}
\phantomsection\label{\detokenize{cplddocs:management_flash.MngProgFlash.SPITransaction}}
\pysigstartsignatures
\pysiglinewithargsret{\sphinxbfcode{\sphinxupquote{SPITransaction}}}{\sphinxparam{\DUrole{n}{device}}\sphinxparamcomma \sphinxparam{\DUrole{n}{TxBuffer}}\sphinxparamcomma \sphinxparam{\DUrole{n}{cmd}}\sphinxparamcomma \sphinxparam{\DUrole{n}{size}}}{}
\pysigstopsignatures
\sphinxAtStartPar
Perform an SPI transaction.
\begin{quote}\begin{description}
\sphinxlineitem{Parameters}\begin{itemize}
\item {} 
\sphinxAtStartPar
\sphinxstyleliteralstrong{\sphinxupquote{device}} \textendash{} Flash device.

\item {} 
\sphinxAtStartPar
\sphinxstyleliteralstrong{\sphinxupquote{TxBuffer}} \textendash{} Transmit buffer.

\item {} 
\sphinxAtStartPar
\sphinxstyleliteralstrong{\sphinxupquote{cmd}} \textendash{} Command to be sent.

\item {} 
\sphinxAtStartPar
\sphinxstyleliteralstrong{\sphinxupquote{size}} \textendash{} Size of the SPI transaction.

\end{itemize}

\sphinxlineitem{Returns}
\sphinxAtStartPar
Received buffer.

\end{description}\end{quote}

\end{fulllineitems}

\index{firmwareProgram() (management\_flash.MngProgFlash method)@\spxentry{firmwareProgram()}\spxextra{management\_flash.MngProgFlash method}}

\begin{fulllineitems}
\phantomsection\label{\detokenize{cplddocs:management_flash.MngProgFlash.firmwareProgram}}
\pysigstartsignatures
\pysiglinewithargsret{\sphinxbfcode{\sphinxupquote{firmwareProgram}}}{\sphinxparam{\DUrole{n}{flashdeviceindedx}}\sphinxparamcomma \sphinxparam{\DUrole{n}{bitstreamFilename}}\sphinxparamcomma \sphinxparam{\DUrole{n}{address}}\sphinxparamcomma \sphinxparam{\DUrole{n}{dumpFilename}\DUrole{o}{=}\DUrole{default_value}{None}}\sphinxparamcomma \sphinxparam{\DUrole{n}{erase\_all}\DUrole{o}{=}\DUrole{default_value}{False}}\sphinxparamcomma \sphinxparam{\DUrole{n}{erase\_size}\DUrole{o}{=}\DUrole{default_value}{None}}\sphinxparamcomma \sphinxparam{\DUrole{n}{add\_len}\DUrole{o}{=}\DUrole{default_value}{False}}}{}
\pysigstopsignatures
\sphinxAtStartPar
Program the firmware onto the flash device.
\begin{quote}\begin{description}
\sphinxlineitem{Parameters}\begin{itemize}
\item {} 
\sphinxAtStartPar
\sphinxstyleliteralstrong{\sphinxupquote{self}} \textendash{} The object instance.

\item {} 
\sphinxAtStartPar
\sphinxstyleliteralstrong{\sphinxupquote{flashdeviceindedx}} (\sphinxstyleliteralemphasis{\sphinxupquote{int}}) \textendash{} Index of the flash device.

\item {} 
\sphinxAtStartPar
\sphinxstyleliteralstrong{\sphinxupquote{bitstreamFilename}} (\sphinxstyleliteralemphasis{\sphinxupquote{str}}) \textendash{} Filename of the bitstream.

\item {} 
\sphinxAtStartPar
\sphinxstyleliteralstrong{\sphinxupquote{address}} (\sphinxstyleliteralemphasis{\sphinxupquote{int}}) \textendash{} Address in the flash where the bitstream will be written.

\item {} 
\sphinxAtStartPar
\sphinxstyleliteralstrong{\sphinxupquote{dumpFilename}} (\sphinxstyleliteralemphasis{\sphinxupquote{str}}\sphinxstyleliteralemphasis{\sphinxupquote{, }}\sphinxstyleliteralemphasis{\sphinxupquote{optional}}) \textendash{} Filename for dumping flash content. Defaults to None.

\item {} 
\sphinxAtStartPar
\sphinxstyleliteralstrong{\sphinxupquote{erase\_all}} (\sphinxstyleliteralemphasis{\sphinxupquote{bool}}\sphinxstyleliteralemphasis{\sphinxupquote{, }}\sphinxstyleliteralemphasis{\sphinxupquote{optional}}) \textendash{} Flag to indicate whether to erase the entire flash. Defaults to False.

\item {} 
\sphinxAtStartPar
\sphinxstyleliteralstrong{\sphinxupquote{erase\_size}} (\sphinxstyleliteralemphasis{\sphinxupquote{int}}\sphinxstyleliteralemphasis{\sphinxupquote{, }}\sphinxstyleliteralemphasis{\sphinxupquote{optional}}) \textendash{} Size to erase, if specified. Defaults to None.

\item {} 
\sphinxAtStartPar
\sphinxstyleliteralstrong{\sphinxupquote{add\_len}} (\sphinxstyleliteralemphasis{\sphinxupquote{bool}}\sphinxstyleliteralemphasis{\sphinxupquote{, }}\sphinxstyleliteralemphasis{\sphinxupquote{optional}}) \textendash{} Flag to prepend bitstream size during writing. Defaults to False.

\end{itemize}

\end{description}\end{quote}

\end{fulllineitems}

\index{firmwareRead() (management\_flash.MngProgFlash method)@\spxentry{firmwareRead()}\spxextra{management\_flash.MngProgFlash method}}

\begin{fulllineitems}
\phantomsection\label{\detokenize{cplddocs:management_flash.MngProgFlash.firmwareRead}}
\pysigstartsignatures
\pysiglinewithargsret{\sphinxbfcode{\sphinxupquote{firmwareRead}}}{\sphinxparam{\DUrole{n}{flashdeviceindedx}}\sphinxparamcomma \sphinxparam{\DUrole{n}{address}}\sphinxparamcomma \sphinxparam{\DUrole{n}{size}}\sphinxparamcomma \sphinxparam{\DUrole{n}{dumpFilename}}}{}
\pysigstopsignatures
\sphinxAtStartPar
Read firmware from the flash device.
\begin{quote}\begin{description}
\sphinxlineitem{Parameters}\begin{itemize}
\item {} 
\sphinxAtStartPar
\sphinxstyleliteralstrong{\sphinxupquote{self}} \textendash{} The object instance.

\item {} 
\sphinxAtStartPar
\sphinxstyleliteralstrong{\sphinxupquote{flashdeviceindedx}} (\sphinxstyleliteralemphasis{\sphinxupquote{int}}) \textendash{} Index of the flash device.

\item {} 
\sphinxAtStartPar
\sphinxstyleliteralstrong{\sphinxupquote{address}} (\sphinxstyleliteralemphasis{\sphinxupquote{int}}) \textendash{} Address in the flash from where to start reading.

\item {} 
\sphinxAtStartPar
\sphinxstyleliteralstrong{\sphinxupquote{size}} (\sphinxstyleliteralemphasis{\sphinxupquote{int}}) \textendash{} Size of the data to read.

\item {} 
\sphinxAtStartPar
\sphinxstyleliteralstrong{\sphinxupquote{dumpFilename}} (\sphinxstyleliteralemphasis{\sphinxupquote{str}}) \textendash{} Filename for dumping the read data.

\end{itemize}

\end{description}\end{quote}

\end{fulllineitems}

\index{loadBitstream() (management\_flash.MngProgFlash method)@\spxentry{loadBitstream()}\spxextra{management\_flash.MngProgFlash method}}

\begin{fulllineitems}
\phantomsection\label{\detokenize{cplddocs:management_flash.MngProgFlash.loadBitstream}}
\pysigstartsignatures
\pysiglinewithargsret{\sphinxbfcode{\sphinxupquote{loadBitstream}}}{\sphinxparam{\DUrole{n}{filename}}\sphinxparamcomma \sphinxparam{\DUrole{n}{sectorSize}}}{}
\pysigstopsignatures
\sphinxAtStartPar
Load a bitstream file into memory.
\begin{quote}\begin{description}
\sphinxlineitem{Parameters}\begin{itemize}
\item {} 
\sphinxAtStartPar
\sphinxstyleliteralstrong{\sphinxupquote{filename}} \textendash{} Path to the bitstream file.

\item {} 
\sphinxAtStartPar
\sphinxstyleliteralstrong{\sphinxupquote{sectorSize}} \textendash{} Size of Flash sectors.

\end{itemize}

\sphinxlineitem{Returns}
\sphinxAtStartPar
Tuple containing the loaded bitstream, bitstream size, and the size of the allocated memory block.

\end{description}\end{quote}

\end{fulllineitems}

\index{saveBitstream() (management\_flash.MngProgFlash method)@\spxentry{saveBitstream()}\spxextra{management\_flash.MngProgFlash method}}

\begin{fulllineitems}
\phantomsection\label{\detokenize{cplddocs:management_flash.MngProgFlash.saveBitstream}}
\pysigstartsignatures
\pysiglinewithargsret{\sphinxbfcode{\sphinxupquote{saveBitstream}}}{\sphinxparam{\DUrole{n}{filename}}\sphinxparamcomma \sphinxparam{\DUrole{n}{memblock}}\sphinxparamcomma \sphinxparam{\DUrole{n}{bitstreamSize}}}{}
\pysigstopsignatures
\sphinxAtStartPar
Save a bitstream from memory to a file.
\begin{quote}\begin{description}
\sphinxlineitem{Parameters}\begin{itemize}
\item {} 
\sphinxAtStartPar
\sphinxstyleliteralstrong{\sphinxupquote{filename}} \textendash{} Path to the output bitstream file.

\item {} 
\sphinxAtStartPar
\sphinxstyleliteralstrong{\sphinxupquote{memblock}} \textendash{} Data to be saved.

\item {} 
\sphinxAtStartPar
\sphinxstyleliteralstrong{\sphinxupquote{bitstreamSize}} \textendash{} Size of the bitstream data.

\end{itemize}

\end{description}\end{quote}

\end{fulllineitems}

\index{spi\_chipselect() (management\_flash.MngProgFlash method)@\spxentry{spi\_chipselect()}\spxextra{management\_flash.MngProgFlash method}}

\begin{fulllineitems}
\phantomsection\label{\detokenize{cplddocs:management_flash.MngProgFlash.spi_chipselect}}
\pysigstartsignatures
\pysiglinewithargsret{\sphinxbfcode{\sphinxupquote{spi\_chipselect}}}{\sphinxparam{\DUrole{n}{isactive}}}{}
\pysigstopsignatures
\sphinxAtStartPar
Set or clear the SPI Chip Select.
\begin{quote}\begin{description}
\sphinxlineitem{Parameters}
\sphinxAtStartPar
\sphinxstyleliteralstrong{\sphinxupquote{isactive}} \textendash{} True to activate the Chip Select, False to deactivate.

\end{description}\end{quote}

\end{fulllineitems}

\index{spi\_config() (management\_flash.MngProgFlash method)@\spxentry{spi\_config()}\spxextra{management\_flash.MngProgFlash method}}

\begin{fulllineitems}
\phantomsection\label{\detokenize{cplddocs:management_flash.MngProgFlash.spi_config}}
\pysigstartsignatures
\pysiglinewithargsret{\sphinxbfcode{\sphinxupquote{spi\_config}}}{\sphinxparam{\DUrole{n}{spi\_cs\_ow}}}{}
\pysigstopsignatures
\sphinxAtStartPar
Configure SPI.
\begin{quote}\begin{description}
\sphinxlineitem{Parameters}
\sphinxAtStartPar
\sphinxstyleliteralstrong{\sphinxupquote{spi\_cs\_ow}} \textendash{} 1 to enable, 0 to disable the SPI Chip Select One\sphinxhyphen{}Wire.

\end{description}\end{quote}

\end{fulllineitems}

\index{spi\_mux\_selection() (management\_flash.MngProgFlash method)@\spxentry{spi\_mux\_selection()}\spxextra{management\_flash.MngProgFlash method}}

\begin{fulllineitems}
\phantomsection\label{\detokenize{cplddocs:management_flash.MngProgFlash.spi_mux_selection}}
\pysigstartsignatures
\pysiglinewithargsret{\sphinxbfcode{\sphinxupquote{spi\_mux\_selection}}}{\sphinxparam{\DUrole{n}{slaveid}}}{}
\pysigstopsignatures
\sphinxAtStartPar
Select the SPI MUX.
\begin{quote}\begin{description}
\sphinxlineitem{Parameters}
\sphinxAtStartPar
\sphinxstyleliteralstrong{\sphinxupquote{slaveid}} \textendash{} Slave ID for SPI MUX selection.

\end{description}\end{quote}

\end{fulllineitems}

\index{spi\_rx\_available() (management\_flash.MngProgFlash method)@\spxentry{spi\_rx\_available()}\spxextra{management\_flash.MngProgFlash method}}

\begin{fulllineitems}
\phantomsection\label{\detokenize{cplddocs:management_flash.MngProgFlash.spi_rx_available}}
\pysigstartsignatures
\pysiglinewithargsret{\sphinxbfcode{\sphinxupquote{spi\_rx\_available}}}{}{}
\pysigstopsignatures
\sphinxAtStartPar
Get the number of available SPI receive bytes.
\begin{quote}\begin{description}
\sphinxlineitem{Returns}
\sphinxAtStartPar
Number of available SPI receive bytes.

\end{description}\end{quote}

\end{fulllineitems}

\index{spi\_sync() (management\_flash.MngProgFlash method)@\spxentry{spi\_sync()}\spxextra{management\_flash.MngProgFlash method}}

\begin{fulllineitems}
\phantomsection\label{\detokenize{cplddocs:management_flash.MngProgFlash.spi_sync}}
\pysigstartsignatures
\pysiglinewithargsret{\sphinxbfcode{\sphinxupquote{spi\_sync}}}{\sphinxparam{\DUrole{n}{slaveid}}\sphinxparamcomma \sphinxparam{\DUrole{n}{txBuffer}}\sphinxparamcomma \sphinxparam{\DUrole{n}{cmd}}\sphinxparamcomma \sphinxparam{\DUrole{n}{length}}}{}
\pysigstopsignatures
\sphinxAtStartPar
Perform a synchronous SPI transaction.
\begin{quote}\begin{description}
\sphinxlineitem{Parameters}\begin{itemize}
\item {} 
\sphinxAtStartPar
\sphinxstyleliteralstrong{\sphinxupquote{slaveid}} \textendash{} Slave ID for SPI communication.

\item {} 
\sphinxAtStartPar
\sphinxstyleliteralstrong{\sphinxupquote{txBuffer}} \textendash{} Transmit buffer.

\item {} 
\sphinxAtStartPar
\sphinxstyleliteralstrong{\sphinxupquote{cmd}} \textendash{} Command to be sent.

\item {} 
\sphinxAtStartPar
\sphinxstyleliteralstrong{\sphinxupquote{length}} \textendash{} Length of SPI transaction.

\end{itemize}

\sphinxlineitem{Returns}
\sphinxAtStartPar
Received buffer.

\end{description}\end{quote}

\end{fulllineitems}

\index{spi\_trigger() (management\_flash.MngProgFlash method)@\spxentry{spi\_trigger()}\spxextra{management\_flash.MngProgFlash method}}

\begin{fulllineitems}
\phantomsection\label{\detokenize{cplddocs:management_flash.MngProgFlash.spi_trigger}}
\pysigstartsignatures
\pysiglinewithargsret{\sphinxbfcode{\sphinxupquote{spi\_trigger}}}{\sphinxparam{\DUrole{n}{length}}}{}
\pysigstopsignatures
\sphinxAtStartPar
Trigger SPI transmission with the specified length.
\begin{quote}\begin{description}
\sphinxlineitem{Parameters}
\sphinxAtStartPar
\sphinxstyleliteralstrong{\sphinxupquote{length}} \textendash{} Length of SPI transmission.

\end{description}\end{quote}

\end{fulllineitems}

\index{spi\_tx\_remaining() (management\_flash.MngProgFlash method)@\spxentry{spi\_tx\_remaining()}\spxextra{management\_flash.MngProgFlash method}}

\begin{fulllineitems}
\phantomsection\label{\detokenize{cplddocs:management_flash.MngProgFlash.spi_tx_remaining}}
\pysigstartsignatures
\pysiglinewithargsret{\sphinxbfcode{\sphinxupquote{spi\_tx\_remaining}}}{}{}
\pysigstopsignatures
\sphinxAtStartPar
Get the number of remaining SPI transmit bytes.
\begin{quote}\begin{description}
\sphinxlineitem{Returns}
\sphinxAtStartPar
Number of remaining SPI transmit bytes.

\end{description}\end{quote}

\end{fulllineitems}


\end{fulllineitems}

\index{spiregisters (class in management\_flash)@\spxentry{spiregisters}\spxextra{class in management\_flash}}

\begin{fulllineitems}
\phantomsection\label{\detokenize{cplddocs:management_flash.spiregisters}}
\pysigstartsignatures
\pysigline{\sphinxbfcode{\sphinxupquote{class\DUrole{w}{ }}}\sphinxcode{\sphinxupquote{management\_flash.}}\sphinxbfcode{\sphinxupquote{spiregisters}}}
\pysigstopsignatures
\sphinxAtStartPar
Class representing SPI registers.

\sphinxAtStartPar
Attributes:
\sphinxhyphen{} spi\_cs\_ow (int): SPI Chip Select One\sphinxhyphen{}Wire register address.
\sphinxhyphen{} spi\_cs0 (int): SPI Chip Select 0 register address.
\sphinxhyphen{} spi\_tx\_byte (int): SPI Transmit Byte register address.
\sphinxhyphen{} spi\_rx\_byte (int): SPI Receive Byte register address.
\sphinxhyphen{} spi\_tx\_buf\_len (int): SPI Transmit Buffer Length register address.
\sphinxhyphen{} spi\_rx\_buf\_len (int): SPI Receive Buffer Length register address.
\sphinxhyphen{} spi\_fifo\_addr (int): SPI FIFO Address register address.
\sphinxhyphen{} spi\_mux (int): SPI Multiplexer register address.
\sphinxhyphen{} spi\_rxtxbuffer (int): SPI Receive/Transmit Buffer register address.

\end{fulllineitems}

\index{module@\spxentry{module}!management\_mcu\_uart@\spxentry{management\_mcu\_uart}}\index{management\_mcu\_uart@\spxentry{management\_mcu\_uart}!module@\spxentry{module}}\index{MngMcuUart (class in management\_mcu\_uart)@\spxentry{MngMcuUart}\spxextra{class in management\_mcu\_uart}}\phantomsection\label{\detokenize{cplddocs:module-management_mcu_uart}}

\begin{fulllineitems}
\phantomsection\label{\detokenize{cplddocs:management_mcu_uart.MngMcuUart}}
\pysigstartsignatures
\pysiglinewithargsret{\sphinxbfcode{\sphinxupquote{class\DUrole{w}{ }}}\sphinxcode{\sphinxupquote{management\_mcu\_uart.}}\sphinxbfcode{\sphinxupquote{MngMcuUart}}}{\sphinxparam{\DUrole{n}{board}}\sphinxparamcomma \sphinxparam{\DUrole{n}{rmp}}}{}
\pysigstopsignatures
\sphinxAtStartPar
Management Class for CPLD2MCU Uart control for MCU update
\index{reset\_mcu() (management\_mcu\_uart.MngMcuUart method)@\spxentry{reset\_mcu()}\spxextra{management\_mcu\_uart.MngMcuUart method}}

\begin{fulllineitems}
\phantomsection\label{\detokenize{cplddocs:management_mcu_uart.MngMcuUart.reset_mcu}}
\pysigstartsignatures
\pysiglinewithargsret{\sphinxbfcode{\sphinxupquote{reset\_mcu}}}{}{}
\pysigstopsignatures
\sphinxAtStartPar
Resets the MCU.

\end{fulllineitems}

\index{start\_mcu\_sam\_ba\_monitor() (management\_mcu\_uart.MngMcuUart method)@\spxentry{start\_mcu\_sam\_ba\_monitor()}\spxextra{management\_mcu\_uart.MngMcuUart method}}

\begin{fulllineitems}
\phantomsection\label{\detokenize{cplddocs:management_mcu_uart.MngMcuUart.start_mcu_sam_ba_monitor}}
\pysigstartsignatures
\pysiglinewithargsret{\sphinxbfcode{\sphinxupquote{start\_mcu\_sam\_ba\_monitor}}}{}{}
\pysigstopsignatures
\sphinxAtStartPar
Initiates the MCU SAM\sphinxhyphen{}BA monitor.
\begin{quote}\begin{description}
\sphinxlineitem{Returns}
\sphinxAtStartPar
The operation status (0 for success, 1 for timeout).

\end{description}\end{quote}

\end{fulllineitems}

\index{uart\_receive\_byte() (management\_mcu\_uart.MngMcuUart method)@\spxentry{uart\_receive\_byte()}\spxextra{management\_mcu\_uart.MngMcuUart method}}

\begin{fulllineitems}
\phantomsection\label{\detokenize{cplddocs:management_mcu_uart.MngMcuUart.uart_receive_byte}}
\pysigstartsignatures
\pysiglinewithargsret{\sphinxbfcode{\sphinxupquote{uart\_receive\_byte}}}{}{}
\pysigstopsignatures
\sphinxAtStartPar
Receives a single byte over UART from the MCU.
\begin{quote}\begin{description}
\sphinxlineitem{Returns}
\sphinxAtStartPar
A tuple containing the received byte and the operation status
(0 for success, 1 for timeout).

\end{description}\end{quote}

\end{fulllineitems}

\index{uart\_send\_buffer() (management\_mcu\_uart.MngMcuUart method)@\spxentry{uart\_send\_buffer()}\spxextra{management\_mcu\_uart.MngMcuUart method}}

\begin{fulllineitems}
\phantomsection\label{\detokenize{cplddocs:management_mcu_uart.MngMcuUart.uart_send_buffer}}
\pysigstartsignatures
\pysiglinewithargsret{\sphinxbfcode{\sphinxupquote{uart\_send\_buffer}}}{\sphinxparam{\DUrole{n}{databuff}}}{}
\pysigstopsignatures
\sphinxAtStartPar
Sends a buffer of data over UART to the MCU.
\begin{quote}\begin{description}
\sphinxlineitem{Parameters}
\sphinxAtStartPar
\sphinxstyleliteralstrong{\sphinxupquote{databuff}} \textendash{} The data buffer to be sent.

\sphinxlineitem{Returns}
\sphinxAtStartPar
The operation status (0 for success, 1 for timeout).

\end{description}\end{quote}

\end{fulllineitems}

\index{uart\_send\_buffer\_wrx() (management\_mcu\_uart.MngMcuUart method)@\spxentry{uart\_send\_buffer\_wrx()}\spxextra{management\_mcu\_uart.MngMcuUart method}}

\begin{fulllineitems}
\phantomsection\label{\detokenize{cplddocs:management_mcu_uart.MngMcuUart.uart_send_buffer_wrx}}
\pysigstartsignatures
\pysiglinewithargsret{\sphinxbfcode{\sphinxupquote{uart\_send\_buffer\_wrx}}}{\sphinxparam{\DUrole{n}{databuff}}}{}
\pysigstopsignatures
\sphinxAtStartPar
Sends a buffer of data over UART to the MCU and receives a response.
\begin{quote}\begin{description}
\sphinxlineitem{Parameters}
\sphinxAtStartPar
\sphinxstyleliteralstrong{\sphinxupquote{databuff}} \textendash{} The data buffer to be sent.

\sphinxlineitem{Returns}
\sphinxAtStartPar
A tuple containing the operation status (0 for success, 1 for timeout)
and the received data buffer.

\end{description}\end{quote}

\end{fulllineitems}

\index{uart\_send\_byte() (management\_mcu\_uart.MngMcuUart method)@\spxentry{uart\_send\_byte()}\spxextra{management\_mcu\_uart.MngMcuUart method}}

\begin{fulllineitems}
\phantomsection\label{\detokenize{cplddocs:management_mcu_uart.MngMcuUart.uart_send_byte}}
\pysigstartsignatures
\pysiglinewithargsret{\sphinxbfcode{\sphinxupquote{uart\_send\_byte}}}{\sphinxparam{\DUrole{n}{dataw}}}{}
\pysigstopsignatures
\sphinxAtStartPar
Sends a single byte over UART to the MCU.
\begin{quote}\begin{description}
\sphinxlineitem{Parameters}
\sphinxAtStartPar
\sphinxstyleliteralstrong{\sphinxupquote{dataw}} \textendash{} The byte of data to be sent.

\sphinxlineitem{Returns}
\sphinxAtStartPar
The operation status (0 for success, 1 for timeout).

\end{description}\end{quote}

\end{fulllineitems}


\end{fulllineitems}

\index{module@\spxentry{module}!management\_spi@\spxentry{management\_spi}}\index{management\_spi@\spxentry{management\_spi}!module@\spxentry{module}}\phantomsection\label{\detokenize{cplddocs:module-management_spi}}

\begin{fulllineitems}

\pysigstartsignatures
\pysiglinewithargsret{\sphinxbfcode{\sphinxupquote{class\DUrole{w}{ }}}\sphinxcode{\sphinxupquote{management.}}\sphinxbfcode{\sphinxupquote{MANAGEMENT}}}{\sphinxparam{\DUrole{o}{**}\DUrole{n}{kwargs}}}{}
\pysigstopsignatures
\sphinxAtStartPar
Class representing a MANAGEMENT instance.


\begin{fulllineitems}

\pysigstartsignatures
\pysiglinewithargsret{\sphinxbfcode{\sphinxupquote{checkLoad}}}{}{}
\pysigstopsignatures
\sphinxAtStartPar
Check if the registers are loaded.
\begin{quote}\begin{description}
\sphinxlineitem{Raises}
\sphinxAtStartPar
\sphinxstyleliteralstrong{\sphinxupquote{NameError}} \textendash{} If registers are not loaded or the board is not connected.

\end{description}\end{quote}

\end{fulllineitems}



\begin{fulllineitems}

\pysigstartsignatures
\pysiglinewithargsret{\sphinxbfcode{\sphinxupquote{disconnect}}}{}{}
\pysigstopsignatures
\sphinxAtStartPar
Disconnect from the network.

\sphinxAtStartPar
This function closes the network connection and sets the state to “Unconnected”.
\begin{quote}\begin{description}
\sphinxlineitem{Raises}
\sphinxAtStartPar
\sphinxstyleliteralstrong{\sphinxupquote{None}} \textendash{} 

\end{description}\end{quote}

\end{fulllineitems}



\begin{fulllineitems}

\pysigstartsignatures
\pysiglinewithargsret{\sphinxbfcode{\sphinxupquote{find\_register}}}{\sphinxparam{\DUrole{n}{register\_name}}\sphinxparamcomma \sphinxparam{\DUrole{n}{display}\DUrole{o}{=}\DUrole{default_value}{False}}}{}
\pysigstopsignatures
\sphinxAtStartPar
Find register information for provided search string.
:param register\_name: Regular expression to search against
:param display: True to output result to console
:return: List of found registers

\end{fulllineitems}



\begin{fulllineitems}

\pysigstartsignatures
\pysiglinewithargsret{\sphinxbfcode{\sphinxupquote{find\_register\_names}}}{\sphinxparam{\DUrole{n}{register\_name}}\sphinxparamcomma \sphinxparam{\DUrole{n}{display}\DUrole{o}{=}\DUrole{default_value}{False}}}{}
\pysigstopsignatures
\sphinxAtStartPar
Find register names for the provided search string.
\begin{quote}\begin{description}
\sphinxlineitem{Parameters}\begin{itemize}
\item {} 
\sphinxAtStartPar
\sphinxstyleliteralstrong{\sphinxupquote{register\_name}} (\sphinxstyleliteralemphasis{\sphinxupquote{str}}) \textendash{} Regular expression to search against.

\item {} 
\sphinxAtStartPar
\sphinxstyleliteralstrong{\sphinxupquote{display}} (\sphinxstyleliteralemphasis{\sphinxupquote{bool}}\sphinxstyleliteralemphasis{\sphinxupquote{, }}\sphinxstyleliteralemphasis{\sphinxupquote{optional}}) \textendash{} True to output result to console. Defaults to False.

\end{itemize}

\sphinxlineitem{Returns}
\sphinxAtStartPar
List of found register names.

\sphinxlineitem{Return type}
\sphinxAtStartPar
list

\end{description}\end{quote}

\end{fulllineitems}



\begin{fulllineitems}

\pysigstartsignatures
\pysiglinewithargsret{\sphinxbfcode{\sphinxupquote{get\_bios}}}{}{}
\pysigstopsignatures
\sphinxAtStartPar
Generate a BIOS string based on specific register values.
\begin{quote}\begin{description}
\sphinxlineitem{Returns}
\sphinxAtStartPar
BIOS information string.

\sphinxlineitem{Return type}
\sphinxAtStartPar
str

\end{description}\end{quote}

\end{fulllineitems}



\begin{fulllineitems}

\pysigstartsignatures
\pysiglinewithargsret{\sphinxbfcode{\sphinxupquote{get\_board}}}{\sphinxparam{\DUrole{n}{ext\_info\_offset}\DUrole{o}{=}\DUrole{default_value}{16}}}{}
\pysigstopsignatures
\sphinxAtStartPar
Get the board name from the extended info string.
\begin{quote}\begin{description}
\sphinxlineitem{Parameters}
\sphinxAtStartPar
\sphinxstyleliteralstrong{\sphinxupquote{ext\_info\_offset}} (\sphinxstyleliteralemphasis{\sphinxupquote{int}}\sphinxstyleliteralemphasis{\sphinxupquote{, }}\sphinxstyleliteralemphasis{\sphinxupquote{optional}}) \textendash{} Offset for extended info. Defaults to 0x10.

\sphinxlineitem{Returns}
\sphinxAtStartPar
The extracted board name.

\sphinxlineitem{Return type}
\sphinxAtStartPar
str

\sphinxlineitem{Raises}
\sphinxAtStartPar
\sphinxstyleliteralstrong{\sphinxupquote{NameError}} \textendash{} If the field BOARD doesn’t exist in the extended info.

\end{description}\end{quote}

\end{fulllineitems}



\begin{fulllineitems}

\pysigstartsignatures
\pysiglinewithargsret{\sphinxbfcode{\sphinxupquote{get\_board\_info}}}{}{}
\pysigstopsignatures
\sphinxAtStartPar
Retrieve information about the board.
\begin{quote}\begin{description}
\sphinxlineitem{Returns}
\sphinxAtStartPar
Board information as a dictionary.

\sphinxlineitem{Return type}
\sphinxAtStartPar
dict

\end{description}\end{quote}

\end{fulllineitems}



\begin{fulllineitems}

\pysigstartsignatures
\pysiglinewithargsret{\sphinxbfcode{\sphinxupquote{get\_extended\_info}}}{\sphinxparam{\DUrole{n}{ext\_info\_offset}\DUrole{o}{=}\DUrole{default_value}{16}}}{}
\pysigstopsignatures
\sphinxAtStartPar
Get the extended info string from the board.

\end{fulllineitems}



\begin{fulllineitems}

\pysigstartsignatures
\pysiglinewithargsret{\sphinxbfcode{\sphinxupquote{get\_mac}}}{}{}
\pysigstopsignatures
\sphinxAtStartPar
Retrieve the MAC address and format it as a string.
\begin{quote}\begin{description}
\sphinxlineitem{Returns}
\sphinxAtStartPar
MAC address string.

\sphinxlineitem{Return type}
\sphinxAtStartPar
str

\end{description}\end{quote}

\end{fulllineitems}



\begin{fulllineitems}

\pysigstartsignatures
\pysiglinewithargsret{\sphinxbfcode{\sphinxupquote{get\_register\_name\_by\_address}}}{\sphinxparam{\DUrole{n}{add}}}{}
\pysigstopsignatures
\sphinxAtStartPar
Get register name by address.
\begin{quote}\begin{description}
\sphinxlineitem{Parameters}
\sphinxAtStartPar
\sphinxstyleliteralstrong{\sphinxupquote{add}} (\sphinxstyleliteralemphasis{\sphinxupquote{int}}) \textendash{} Register address.

\sphinxlineitem{Returns}
\sphinxAtStartPar
Register name or “Unknown register address” if not found.

\sphinxlineitem{Return type}
\sphinxAtStartPar
str

\end{description}\end{quote}

\end{fulllineitems}



\begin{fulllineitems}

\pysigstartsignatures
\pysiglinewithargsret{\sphinxbfcode{\sphinxupquote{get\_xml\_from\_board}}}{\sphinxparam{\DUrole{n}{xml\_map\_offset}}}{}
\pysigstopsignatures
\sphinxAtStartPar
Get XML data from the board.

\end{fulllineitems}



\begin{fulllineitems}

\pysigstartsignatures
\pysiglinewithargsret{\sphinxbfcode{\sphinxupquote{list\_register\_names}}}{}{}
\pysigstopsignatures
\sphinxAtStartPar
List register names.
\begin{quote}\begin{description}
\sphinxlineitem{Raises}
\sphinxAtStartPar
\sphinxstyleliteralstrong{\sphinxupquote{None}} \textendash{} 

\end{description}\end{quote}

\end{fulllineitems}



\begin{fulllineitems}

\pysigstartsignatures
\pysiglinewithargsret{\sphinxbfcode{\sphinxupquote{load\_firmware\_blocking}}}{\sphinxparam{\DUrole{n}{Device}}\sphinxparamcomma \sphinxparam{\DUrole{n}{path\_to\_xml\_file}\DUrole{o}{=}\DUrole{default_value}{\textquotesingle{}\textquotesingle{}}}\sphinxparamcomma \sphinxparam{\DUrole{n}{xml\_map\_offset}\DUrole{o}{=}\DUrole{default_value}{8}}}{}
\pysigstopsignatures
\sphinxAtStartPar
Load firmware blocking.

\end{fulllineitems}



\begin{fulllineitems}

\pysigstartsignatures
\pysiglinewithargsret{\sphinxbfcode{\sphinxupquote{pll\_calib}}}{}{}
\pysigstopsignatures
\sphinxAtStartPar
Calibrate PLL.

\sphinxAtStartPar
This function performs calibration by writing specific values to SPI addresses.
\begin{quote}\begin{description}
\sphinxlineitem{Raises}
\sphinxAtStartPar
\sphinxstyleliteralstrong{\sphinxupquote{None}} \textendash{} 

\end{description}\end{quote}

\end{fulllineitems}



\begin{fulllineitems}

\pysigstartsignatures
\pysiglinewithargsret{\sphinxbfcode{\sphinxupquote{pll\_dumpcfg}}}{\sphinxparam{\DUrole{n}{cfg\_filename}}}{}
\pysigstopsignatures
\sphinxAtStartPar
Dump PLL configuration to a file.
\begin{quote}\begin{description}
\sphinxlineitem{Parameters}
\sphinxAtStartPar
\sphinxstyleliteralstrong{\sphinxupquote{cfg\_filename}} (\sphinxstyleliteralemphasis{\sphinxupquote{str}}) \textendash{} The path to the configuration file.

\end{description}\end{quote}

\end{fulllineitems}



\begin{fulllineitems}

\pysigstartsignatures
\pysiglinewithargsret{\sphinxbfcode{\sphinxupquote{pll\_ioupdate}}}{}{}
\pysigstopsignatures
\sphinxAtStartPar
Update PLL IO.

\sphinxAtStartPar
This function updates PLL IO by writing a specific value to the SPI address.
\begin{quote}\begin{description}
\sphinxlineitem{Raises}
\sphinxAtStartPar
\sphinxstyleliteralstrong{\sphinxupquote{None}} \textendash{} 

\end{description}\end{quote}

\end{fulllineitems}



\begin{fulllineitems}

\pysigstartsignatures
\pysiglinewithargsret{\sphinxbfcode{\sphinxupquote{pll\_ldcfg}}}{\sphinxparam{\DUrole{n}{cfg\_filename}}}{}
\pysigstopsignatures
\sphinxAtStartPar
Load PLL configuration from a file.
\begin{quote}\begin{description}
\sphinxlineitem{Parameters}
\sphinxAtStartPar
\sphinxstyleliteralstrong{\sphinxupquote{cfg\_filename}} (\sphinxstyleliteralemphasis{\sphinxupquote{str}}) \textendash{} The path to the configuration file.

\end{description}\end{quote}

\end{fulllineitems}



\begin{fulllineitems}

\pysigstartsignatures
\pysiglinewithargsret{\sphinxbfcode{\sphinxupquote{pll\_read\_with\_update}}}{\sphinxparam{\DUrole{n}{address}}}{}
\pysigstopsignatures
\sphinxAtStartPar
Read from SPI with PLL update.
\begin{quote}\begin{description}
\sphinxlineitem{Parameters}
\sphinxAtStartPar
\sphinxstyleliteralstrong{\sphinxupquote{address}} (\sphinxstyleliteralemphasis{\sphinxupquote{int}}) \textendash{} The address to read from.

\sphinxlineitem{Returns}
\sphinxAtStartPar
The read value.

\sphinxlineitem{Return type}
\sphinxAtStartPar
int

\end{description}\end{quote}

\end{fulllineitems}



\begin{fulllineitems}

\pysigstartsignatures
\pysiglinewithargsret{\sphinxbfcode{\sphinxupquote{pll\_write\_with\_update}}}{\sphinxparam{\DUrole{n}{address}}\sphinxparamcomma \sphinxparam{\DUrole{n}{value}}}{}
\pysigstopsignatures
\sphinxAtStartPar
Write to SPI with PLL update.
\begin{quote}\begin{description}
\sphinxlineitem{Parameters}\begin{itemize}
\item {} 
\sphinxAtStartPar
\sphinxstyleliteralstrong{\sphinxupquote{address}} (\sphinxstyleliteralemphasis{\sphinxupquote{int}}) \textendash{} The address to write to.

\item {} 
\sphinxAtStartPar
\sphinxstyleliteralstrong{\sphinxupquote{value}} (\sphinxstyleliteralemphasis{\sphinxupquote{int}}) \textendash{} The value to write.

\end{itemize}

\end{description}\end{quote}

\end{fulllineitems}



\begin{fulllineitems}

\pysigstartsignatures
\pysiglinewithargsret{\sphinxbfcode{\sphinxupquote{read\_register}}}{\sphinxparam{\DUrole{n}{register}}\sphinxparamcomma \sphinxparam{\DUrole{n}{n}\DUrole{o}{=}\DUrole{default_value}{1}}\sphinxparamcomma \sphinxparam{\DUrole{n}{offset}\DUrole{o}{=}\DUrole{default_value}{0}}\sphinxparamcomma \sphinxparam{\DUrole{n}{device}\DUrole{o}{=}\DUrole{default_value}{None}}}{}
\pysigstopsignatures
\sphinxAtStartPar
Get register value
:param register: Register name
:param n: Number of words to read
:param offset: Memory address offset to read from
:param device: Device/node can be explicitly specified
:return: Values

\end{fulllineitems}



\begin{fulllineitems}

\pysigstartsignatures
\pysiglinewithargsret{\sphinxbfcode{\sphinxupquote{read\_spi}}}{\sphinxparam{\DUrole{n}{address}}}{}
\pysigstopsignatures
\sphinxAtStartPar
Read SPI data from the specified address.
\begin{quote}\begin{description}
\sphinxlineitem{Parameters}
\sphinxAtStartPar
\sphinxstyleliteralstrong{\sphinxupquote{address}} (\sphinxstyleliteralemphasis{\sphinxupquote{int}}) \textendash{} The address to read from.

\sphinxlineitem{Returns}
\sphinxAtStartPar
The read value.

\sphinxlineitem{Return type}
\sphinxAtStartPar
int

\end{description}\end{quote}

\end{fulllineitems}



\begin{fulllineitems}

\pysigstartsignatures
\pysiglinewithargsret{\sphinxbfcode{\sphinxupquote{write\_register}}}{\sphinxparam{\DUrole{n}{register}}\sphinxparamcomma \sphinxparam{\DUrole{n}{values}}\sphinxparamcomma \sphinxparam{\DUrole{n}{offset}\DUrole{o}{=}\DUrole{default_value}{0}}\sphinxparamcomma \sphinxparam{\DUrole{n}{device}\DUrole{o}{=}\DUrole{default_value}{None}}}{}
\pysigstopsignatures
\sphinxAtStartPar
Set register value
:param register: Register name
:param values: Values to write
:param offset: Memory address offset to write to
:param device: Device/node can be explicitly specified

\end{fulllineitems}



\begin{fulllineitems}

\pysigstartsignatures
\pysiglinewithargsret{\sphinxbfcode{\sphinxupquote{write\_spi}}}{\sphinxparam{\DUrole{n}{address}}\sphinxparamcomma \sphinxparam{\DUrole{n}{value}}}{}
\pysigstopsignatures
\sphinxAtStartPar
Write SPI data to the specified address.
\begin{quote}\begin{description}
\sphinxlineitem{Parameters}\begin{itemize}
\item {} 
\sphinxAtStartPar
\sphinxstyleliteralstrong{\sphinxupquote{address}} (\sphinxstyleliteralemphasis{\sphinxupquote{int}}) \textendash{} The address to write to.

\item {} 
\sphinxAtStartPar
\sphinxstyleliteralstrong{\sphinxupquote{value}} (\sphinxstyleliteralemphasis{\sphinxupquote{int}}) \textendash{} The value to write.

\end{itemize}

\end{description}\end{quote}

\end{fulllineitems}


\end{fulllineitems}



\begin{fulllineitems}

\pysigstartsignatures
\pysiglinewithargsret{\sphinxcode{\sphinxupquote{management.}}\sphinxbfcode{\sphinxupquote{filter\_list\_by\_level}}}{\sphinxparam{\DUrole{n}{reg\_name\_list}}\sphinxparamcomma \sphinxparam{\DUrole{n}{reg\_name}}}{}
\pysigstopsignatures
\sphinxAtStartPar
Filter a list of register names by level.

\end{fulllineitems}



\begin{fulllineitems}

\pysigstartsignatures
\pysiglinewithargsret{\sphinxcode{\sphinxupquote{management.}}\sphinxbfcode{\sphinxupquote{format\_num}}}{\sphinxparam{\DUrole{n}{num}}}{}
\pysigstopsignatures
\sphinxAtStartPar
Convert a number to a string.

\end{fulllineitems}



\begin{fulllineitems}

\pysigstartsignatures
\pysiglinewithargsret{\sphinxcode{\sphinxupquote{management.}}\sphinxbfcode{\sphinxupquote{get\_max\_width}}}{\sphinxparam{\DUrole{n}{table1}}\sphinxparamcomma \sphinxparam{\DUrole{n}{index1}}}{}
\pysigstopsignatures
\sphinxAtStartPar
Get the maximum width of the given column index

\end{fulllineitems}



\begin{fulllineitems}

\pysigstartsignatures
\pysiglinewithargsret{\sphinxcode{\sphinxupquote{management.}}\sphinxbfcode{\sphinxupquote{get\_shift\_from\_mask}}}{\sphinxparam{\DUrole{n}{mask}}}{}
\pysigstopsignatures
\sphinxAtStartPar
Get the shift value from a mask.

\end{fulllineitems}



\begin{fulllineitems}

\pysigstartsignatures
\pysiglinewithargsret{\sphinxcode{\sphinxupquote{management.}}\sphinxbfcode{\sphinxupquote{hexstring2ascii}}}{\sphinxparam{\DUrole{n}{hexstring}}\sphinxparamcomma \sphinxparam{\DUrole{n}{xor}\DUrole{o}{=}\DUrole{default_value}{0}}}{}
\pysigstopsignatures
\sphinxAtStartPar
Convert a hexstring to an ASCII\sphinxhyphen{}String.
\begin{quote}\begin{description}
\sphinxlineitem{Parameters}\begin{itemize}
\item {} 
\sphinxAtStartPar
\sphinxstyleliteralstrong{\sphinxupquote{hexstring}} (\sphinxstyleliteralemphasis{\sphinxupquote{str}}) \textendash{} The input hex string.

\item {} 
\sphinxAtStartPar
\sphinxstyleliteralstrong{\sphinxupquote{xor}} (\sphinxstyleliteralemphasis{\sphinxupquote{int}}) \textendash{} XOR value as an integer (default is 0).

\end{itemize}

\sphinxlineitem{Returns}
\sphinxAtStartPar
The converted ASCII string.

\sphinxlineitem{Return type}
\sphinxAtStartPar
str

\end{description}\end{quote}

\end{fulllineitems}



\begin{fulllineitems}

\pysigstartsignatures
\pysiglinewithargsret{\sphinxcode{\sphinxupquote{management.}}\sphinxbfcode{\sphinxupquote{pprint\_table}}}{\sphinxparam{\DUrole{n}{table}}}{}
\pysigstopsignatures
\sphinxAtStartPar
Prints out a table of data, padded for alignment.
\begin{quote}\begin{description}
\sphinxlineitem{Parameters}
\sphinxAtStartPar
\sphinxstyleliteralstrong{\sphinxupquote{table}} (\sphinxstyleliteralemphasis{\sphinxupquote{list}}) \textendash{} The table to print. A list of lists.
Each row must have the same number of columns.

\end{description}\end{quote}

\end{fulllineitems}


\sphinxstepscope


\chapter{Subrack Tools Documentation}
\label{\detokenize{toolsdocs:module-power_on_tpm}}\label{\detokenize{toolsdocs:subrack-tools-documentation}}\label{\detokenize{toolsdocs::doc}}\index{module@\spxentry{module}!power\_on\_tpm@\spxentry{power\_on\_tpm}}\index{power\_on\_tpm@\spxentry{power\_on\_tpm}!module@\spxentry{module}}\index{usage\_hexample (in module power\_on\_tpm)@\spxentry{usage\_hexample}\spxextra{in module power\_on\_tpm}}

\begin{fulllineitems}
\phantomsection\label{\detokenize{toolsdocs:power_on_tpm.usage_hexample}}
\pysigstartsignatures
\pysigline{\sphinxcode{\sphinxupquote{power\_on\_tpm.}}\sphinxbfcode{\sphinxupquote{usage\_hexample}}\sphinxbfcode{\sphinxupquote{\DUrole{w}{ }\DUrole{p}{=}\DUrole{w}{ }\textquotesingle{}(es power on TPM1 and TPM3). \%prog \sphinxhyphen{}\sphinxhyphen{}t1 \sphinxhyphen{}\sphinxhyphen{}t3\textbackslash{}n(es power on all TPM). \%prog \sphinxhyphen{}\sphinxhyphen{}all\textbackslash{}n\textquotesingle{}}}}
\pysigstopsignatures
\sphinxAtStartPar
TPM Power Control Script

\sphinxAtStartPar
This script provides a command\sphinxhyphen{}line interface for controlling the power state of TPMs (Trusted Platform Modules) in a SKALAB Subrack. It uses the OptionParser module to parse command\sphinxhyphen{}line options for selecting TPMs to power on.
\begin{description}
\sphinxlineitem{Command\sphinxhyphen{}Line Options:}
\sphinxAtStartPar
\textendash{}t1, \textendash{}t2, …, \textendash{}t8: Select individual TPMs (1 to 8) to power on.
\textendash{}all: Select all TPMs to power on.

\sphinxlineitem{Usage Example:}
\sphinxAtStartPar
\$ python script\_name.py \textendash{}t1 \textendash{}t3 \textendash{}all

\end{description}

\sphinxAtStartPar
Note: Replace ‘script\_name.py’ with the actual name of the script file.

\end{fulllineitems}

\index{module@\spxentry{module}!power\_off\_tpm@\spxentry{power\_off\_tpm}}\index{power\_off\_tpm@\spxentry{power\_off\_tpm}!module@\spxentry{module}}\index{usage\_hexample (in module power\_off\_tpm)@\spxentry{usage\_hexample}\spxextra{in module power\_off\_tpm}}\phantomsection\label{\detokenize{toolsdocs:module-power_off_tpm}}

\begin{fulllineitems}
\phantomsection\label{\detokenize{toolsdocs:power_off_tpm.usage_hexample}}
\pysigstartsignatures
\pysigline{\sphinxcode{\sphinxupquote{power\_off\_tpm.}}\sphinxbfcode{\sphinxupquote{usage\_hexample}}\sphinxbfcode{\sphinxupquote{\DUrole{w}{ }\DUrole{p}{=}\DUrole{w}{ }\textquotesingle{}(es power off TPM1 and TPM3). \%prog \sphinxhyphen{}\sphinxhyphen{}t1 \sphinxhyphen{}\sphinxhyphen{}t3\textbackslash{}n(es power off all TPM). \%prog \sphinxhyphen{}\sphinxhyphen{}all\textbackslash{}n\textquotesingle{}}}}
\pysigstopsignatures
\sphinxAtStartPar
TPM Power Control Script

\sphinxAtStartPar
This script provides a command\sphinxhyphen{}line interface for controlling the power state of TPMs (Trusted Platform Modules) in a SKALAB Subrack. It uses the OptionParser module to parse command\sphinxhyphen{}line options for selecting TPMs to power off.
\begin{description}
\sphinxlineitem{Command\sphinxhyphen{}Line Options:}
\sphinxAtStartPar
\textendash{}t1, \textendash{}t2, …, \textendash{}t8: Select individual TPMs (1 to 8) to power off.
\textendash{}all: Select all TPMs to power off.

\sphinxlineitem{Usage Example:}
\sphinxAtStartPar
\$ python script\_name.py \textendash{}t1 \textendash{}t3 \textendash{}all

\end{description}

\sphinxAtStartPar
Note: Replace ‘script\_name.py’ with the actual name of the script file.

\end{fulllineitems}

\index{module@\spxentry{module}!fpga\_reg@\spxentry{fpga\_reg}}\index{fpga\_reg@\spxentry{fpga\_reg}!module@\spxentry{module}}\index{module@\spxentry{module}!fpga\_i2c\_reg@\spxentry{fpga\_i2c\_reg}}\index{fpga\_i2c\_reg@\spxentry{fpga\_i2c\_reg}!module@\spxentry{module}}\index{module@\spxentry{module}!i2c\_reg@\spxentry{i2c\_reg}}\index{i2c\_reg@\spxentry{i2c\_reg}!module@\spxentry{module}}\index{get\_dev\_add() (in module i2c\_reg)@\spxentry{get\_dev\_add()}\spxextra{in module i2c\_reg}}\phantomsection\label{\detokenize{toolsdocs:module-fpga_reg}}\phantomsection\label{\detokenize{toolsdocs:module-fpga_i2c_reg}}\phantomsection\label{\detokenize{toolsdocs:module-i2c_reg}}

\begin{fulllineitems}
\phantomsection\label{\detokenize{toolsdocs:i2c_reg.get_dev_add}}
\pysigstartsignatures
\pysiglinewithargsret{\sphinxcode{\sphinxupquote{i2c\_reg.}}\sphinxbfcode{\sphinxupquote{get\_dev\_add}}}{\sphinxparam{\DUrole{n}{devname}}}{}
\pysigstopsignatures
\sphinxAtStartPar
Get the I2C device address based on the device name.
\begin{quote}\begin{description}
\sphinxlineitem{Parameters}
\sphinxAtStartPar
\sphinxstyleliteralstrong{\sphinxupquote{devname}} (\sphinxstyleliteralemphasis{\sphinxupquote{str}}) \textendash{} The name of the I2C device.

\sphinxlineitem{Returns}
\sphinxAtStartPar
The I2C device address.

\sphinxlineitem{Return type}
\sphinxAtStartPar
int

\sphinxlineitem{Raises}
\sphinxAtStartPar
\sphinxstyleliteralstrong{\sphinxupquote{SystemExit}} \textendash{} If the provided device name is incorrect.

\end{description}\end{quote}

\end{fulllineitems}



\chapter{Indices and tables}
\label{\detokenize{index:indices-and-tables}}\begin{itemize}
\item {} 
\sphinxAtStartPar
\DUrole{xref,std,std-ref}{genindex}

\item {} 
\sphinxAtStartPar
\DUrole{xref,std,std-ref}{modindex}

\item {} 
\sphinxAtStartPar
\DUrole{xref,std,std-ref}{search}

\end{itemize}


\renewcommand{\indexname}{Python Module Index}
\begin{sphinxtheindex}
\let\bigletter\sphinxstyleindexlettergroup
\bigletter{b}
\item\relax\sphinxstyleindexentry{backplane}\sphinxstyleindexpageref{apidocs:\detokenize{module-backplane}}
\indexspace
\bigletter{c}
\item\relax\sphinxstyleindexentry{config\_ip}\sphinxstyleindexpageref{cplddocs:\detokenize{module-config_ip}}
\item\relax\sphinxstyleindexentry{cpld2mcu\_serial\_ctrl\_2}\sphinxstyleindexpageref{cplddocs:\detokenize{module-cpld2mcu_serial_ctrl_2}}
\indexspace
\bigletter{f}
\item\relax\sphinxstyleindexentry{fpga\_i2c\_reg}\sphinxstyleindexpageref{toolsdocs:\detokenize{module-fpga_i2c_reg}}
\item\relax\sphinxstyleindexentry{fpga\_reg}\sphinxstyleindexpageref{toolsdocs:\detokenize{module-fpga_reg}}
\indexspace
\bigletter{h}
\item\relax\sphinxstyleindexentry{HardwareBaseClass}\sphinxstyleindexpageref{webserverdocs:\detokenize{module-HardwareBaseClass}}
\item\relax\sphinxstyleindexentry{HardwareThreadedClass}\sphinxstyleindexpageref{webserverdocs:\detokenize{module-HardwareThreadedClass}}
\indexspace
\bigletter{i}
\item\relax\sphinxstyleindexentry{i2c\_reg}\sphinxstyleindexpageref{toolsdocs:\detokenize{module-i2c_reg}}
\indexspace
\bigletter{m}
\item\relax\sphinxstyleindexentry{management}\sphinxstyleindexpageref{apidocs:\detokenize{module-management}}
\item\relax\sphinxstyleindexentry{management\_bsp}\sphinxstyleindexpageref{cplddocs:\detokenize{module-management_bsp}}
\item\relax\sphinxstyleindexentry{management\_flash}\sphinxstyleindexpageref{cplddocs:\detokenize{module-management_flash}}
\item\relax\sphinxstyleindexentry{management\_mcu\_uart}\sphinxstyleindexpageref{cplddocs:\detokenize{module-management_mcu_uart}}
\item\relax\sphinxstyleindexentry{management\_pll}\sphinxstyleindexpageref{cplddocs:\detokenize{module-management_pll}}
\item\relax\sphinxstyleindexentry{management\_spi}\sphinxstyleindexpageref{cplddocs:\detokenize{module-management_spi}}
\item\relax\sphinxstyleindexentry{mcu\_update}\sphinxstyleindexpageref{cplddocs:\detokenize{module-mcu_update}}
\item\relax\sphinxstyleindexentry{mng\_update}\sphinxstyleindexpageref{cplddocs:\detokenize{module-mng_update}}
\indexspace
\bigletter{p}
\item\relax\sphinxstyleindexentry{phy\_marvell\_88X2222\_init}\sphinxstyleindexpageref{cplddocs:\detokenize{module-phy_marvell_88X2222_init}}
\item\relax\sphinxstyleindexentry{power\_off\_tpm}\sphinxstyleindexpageref{toolsdocs:\detokenize{module-power_off_tpm}}
\item\relax\sphinxstyleindexentry{power\_on\_tpm}\sphinxstyleindexpageref{toolsdocs:\detokenize{module-power_on_tpm}}
\indexspace
\bigletter{r}
\item\relax\sphinxstyleindexentry{reg}\sphinxstyleindexpageref{cplddocs:\detokenize{module-reg}}
\item\relax\sphinxstyleindexentry{rmp}\sphinxstyleindexpageref{cplddocs:\detokenize{module-rmp}}
\indexspace
\bigletter{s}
\item\relax\sphinxstyleindexentry{subrack\_hardware}\sphinxstyleindexpageref{webserverdocs:\detokenize{module-subrack_hardware}}
\item\relax\sphinxstyleindexentry{subrack\_management\_board}\sphinxstyleindexpageref{apidocs:\detokenize{module-subrack_management_board}}
\item\relax\sphinxstyleindexentry{subrack\_monitoring\_point\_lookup}\sphinxstyleindexpageref{apidocs:\detokenize{module-subrack_monitoring_point_lookup}}
\indexspace
\bigletter{w}
\item\relax\sphinxstyleindexentry{web\_server}\sphinxstyleindexpageref{webserverdocs:\detokenize{module-web_server}}
\end{sphinxtheindex}

\renewcommand{\indexname}{Index}
\printindex
\end{document}